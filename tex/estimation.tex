\chapter{目标参数估计}

在完成信号检测之后,雷达系统还需要进一步提取目标的关键参数,例如方位角、距离、多普勒频率以及极化特征等。这些参数的准确估计不仅关系到目标的定位与跟踪精度,也是后续态势感知与目标识别的重要基础。从算法发展的角度来看,目标参数估计大体经历了几个阶段:最初的经典波束形成方法,随后发展到最小方差处理方法,再到子空间方法和稀疏表示方法。近年来,随着多维观测手段的兴起,多域联合估计方法逐渐受到关注。这类方法能够同时利用空间、多普勒甚至极化等多维信息,从而实现更强的参数解耦和更高的估计精度。

在\cref{sec_radar_signal_model}中,我们介绍了雷达的接收信号模型。对于一个阵列多普勒雷达系统,其接收信号可以表示为一个张量:
\[
    \begin{split}
        \mathcal{X} = \mathcal{I}_K \times_1 \mathbf{R} \times_2 \mathbf{A} \times_3 \mathbf{B} \times_4 \mathbf{P},
    \end{split}
\]
其中,\( K \)为目标数,而矩阵\( \mathbf{R} \)、\( \mathbf{A} \)、\( \mathbf{B} \)和\( \mathbf{P} \)分别包含了目标的距离、方位角、俯仰角、多普勒信息。利用脉冲压缩技术,可以获取目标的距离信息,而本章的重点则是如何进一步从数据中提取目标的方位角和多普勒频率等参数。

\section{波束形成方法}
考虑均匀线性阵列或单阵元在多脉冲下的观测情形,此时,接收信号退化为一个矩阵:
\[
    \mathbf{X} = \mathbf{R} \mathbf{A}^{\mathrm{T}} \in \mathbb{C}^{L \times M},
\]
其中\( \mathbf{R} = \begin{bsmallmatrix} \bm{r}_1 & \bm{r}_2 & \cdots & \bm{r}_K \end{bsmallmatrix} \in \mathbb{C}^{L \times K} \)为目标信号矩阵,\( \mathbf{A} = \begin{bsmallmatrix} \bm{a}_{\omega_1} & \bm{a}_{\omega_2} & \cdots & \bm{a}_{\omega_K} \end{bsmallmatrix} \in \mathbb{C}^{M \times K} \)为阵列流形矩阵(导向矢量矩阵),其列向量为目标的空间或多普勒流形向量,其第\( k \)列通常有如下形式:
\[
    \bm{a}_{\omega_k} = \begin{bmatrix}
        1 & e^{j 2 \pi \omega_k} & \cdots & e^{j 2 \pi (M-1) \omega_k }
    \end{bmatrix}^{\mathrm{T}}.
\]
其中,参数 $\omega_k$ 的物理意义取决于雷达体制:若为阵列雷达,则 $\omega_k = \frac{d}{\lambda}\sin \theta_k$,与目标的方位角 $\theta_k$ 相关;若为单阵元多脉冲雷达,则 $\omega_k = \frac{2 T_r v_k}{\lambda}$,与目标的径向速度 $v_k$ 相关。

波束形成(Beamforming)是最经典的参数估计方法,其基本思想是通过对接收信号进行加权求和,形成一个指向特定方向的空间滤波器或指定速度的多普勒滤波器,从而增强该方向或速度上的信号分量。对于给定滤波器\( \bm{w}_{\omega} \),波束形成的输出可以表示为:
\[
    \bm{y}_{\omega} = \mathbf{X} \bm{w}_{\omega} \in \mathbb{C}^{L \times 1},
\]
可以看作是某个方向或速度上接收到的回波。

\subsection{经典波束形成}

经典波束形成(Conventional Beamforming, CBF)方法利用了不同频率复指数信号在维度充分大时趋于正交的性质,即当 $M$ 足够大时,有
\[
    \frac{1}{M}\bm{a}_{\omega_i}^{\mathrm{H}} \bm{a}_{\omega_j} \approx
    \begin{cases}
        1, & \omega_i = \omega_j,     \\
        0, & \omega_i \neq \omega_j .
    \end{cases}
\]
因此,将滤波器权向量选为与目标流形向量相匹配的形式:
\[
    \bm{w}_{\omega} = \frac{1}{M} \overline{\bm{a}}
    = \frac{1}{M} \begin{bmatrix}
        1 & e^{j 2 \pi \omega} & \cdots & e^{j 2 \pi (M-1) \omega}
    \end{bmatrix}^{\mathrm{H}},
\]
即可有效提取出对应参数值 $\omega$ 的信号分量。该方法又称为 \textbf{Bartlett 波束形成},是最经典的一种波束形成策略。

进一步地,通过扫描不同的参数值\( \omega \),将原始信号投影到不同的方向或速度上,则可以分别获得不同方向或速度上的信号分量。具体而言,对于阵列信号,可以构建如下矩阵:
\[
    \mathbf{W} = \frac{1}{M}\begin{bmatrix}
        1                                                   & 1                                                    & \cdots & 1                                                   \\
        e^{- j 2 \pi \frac{d \sin \theta_1}{\lambda}}       & e^{- j 2 \pi \frac{d \sin \theta_2}{\lambda}}        & \cdots & e^{- j 2 \pi \frac{d \sin \theta_N}{\lambda}}       \\
        \vdots                                              & \vdots                                               & \cdots & \vdots                                              \\
        e^{- j 2 \pi \frac{(M-1) d \sin \theta_1}{\lambda}} & e^{- j 2 \pi \frac{ (M-1) d \sin \theta_2}{\lambda}} & \cdots & e^{- j 2 \pi \frac{(M-1) d \sin \theta_N}{\lambda}}
    \end{bmatrix}\in \mathbb{C}^{M \times N},
\]
其中,\( \theta_1, \theta_2, \cdots, \theta_N \)为扫描的角度值。对于多脉冲信号,可以构建类似的矩阵:
\[
    \mathbf{W} = \frac{1}{M}\begin{bmatrix}
        1                                             & 1                                             & \cdots & 1                                             \\
        e^{- j 2 \pi \frac{2 T_r v_1}{\lambda}}       & e^{- j 2 \pi \frac{2 T_r v_2}{\lambda}}       & \cdots & e^{- j 2 \pi \frac{2 T_r v_N}{\lambda}}       \\
        \vdots                                        & \vdots                                        & \cdots & \vdots                                        \\
        e^{- j 2 \pi \frac{2 (M-1) T_r v_1}{\lambda}} & e^{- j 2 \pi \frac{2 (M-1) T_r v_2}{\lambda}} & \cdots & e^{- j 2 \pi \frac{2 (M-1) T_r v_N}{\lambda}}
    \end{bmatrix} \in \mathbb{C}^{M \times N},
\]
其中,\( v_1, v_2, \cdots, v_N \)为扫描的速度值。对应的波束形成可以表示为如下的矩阵乘法:
\[
    \mathbf{Y} = \mathbf{X} \mathbf{W} \in \mathbb{C}^{L \times N}.
\]
有趣的是,对于多脉冲信号,当所选扫描速度 $v_n$ 满足
\[
    \frac{2 T_r v_n}{\lambda} = \frac{n-1}{M}, \quad n = 1, 2, \cdots, N
\]
时,波束形成矩阵 $\mathbf{W}$ 正好对应于一个 $M \times M$ 的离散傅里叶变换(Discrete Fourier Transform,DFT)矩阵的前 $N$ 行。特别地,当 $M = N$ 时,$\mathbf{W}$ 即为标准的 DFT 矩阵。在这种情况下,波束形成过程等价于对接收信号在参数维度(慢时间维)上进行离散傅里叶变换,从而实现对多普勒频率的估计。

此外,根据\cref{apx.conv-corr-mat}中的结论,对 $\mathbf{X}$ 的列向量 $\bm{x}_i$ 进行脉冲压缩同样可以写成矩阵乘法形式:
\[
    \bm{y}_i = \mathbf{S}^{\mathrm{H}}\bm{x}_i,
\]
其中,$\mathbf{S}$ 为由参考向量构造的矩阵,有如下形式:
\[
    \mathbf{S} =
    \begin{bmatrix}
        s_1    & 0       & \cdots & 0      \\
        s_2    & s_1     & \cdots & 0      \\
        \vdots & \vdots  & \ddots & \vdots \\
        s_W    & s_{W-1} & \ddots & s_1    \\
        0      & s_W     & \ddots & s_2    \\
        \vdots & \vdots  & \ddots & \vdots \\
        0      & 0       & \cdots & s_W
    \end{bmatrix}.
\]
由此可见,当同时对接收信号矩阵 $\mathbf{X}$ 的行向量与列向量分别实施波束形成与脉冲压缩时,其结果可统一表示为
\[
    \mathbf{Z} = \mathbf{S}^{\mathrm{H}} \mathbf{X}\mathbf{W}.
\]

在矩阵 $\mathbf{Z}$ 中,目标表现为显著的峰值,其位置对应目标的距离及其他参数(方位或径向速度)。换言之,波束形成可视为在参数维度上的匹配滤波,即对目标信号执行``参数域脉冲压缩''。结合距离维与参数维的双重脉冲压缩,便可实现目标的定位与参数估计。

\begin{example}
    对\cref{fig_data_x}中的接收数据矩阵\( \mathbf{X} \)进行脉冲压缩和波束形成处理。

    \begin{figure}[htb!]
        \centering
        % \includegraphics[width=.4\textwidth]{./img}
        \begin{tikzpicture}
            \begin{axis}[
                    xlabel={参数维},
                    ylabel={时间维},
                    enlargelimits=false,
                    width=4.5cm, height=4.5cm,
                    ytick=\empty,
                    xtick=\empty,
                    ticklabel style={font=\small},
                    label style={font=\small},
                    axis on top
                ]
                \addplot graphics [
                        xmin=-1, xmax=1, ymin=-1, ymax=1,
                    ] {./img/estimation/est_1.png};
            \end{axis}
        \end{tikzpicture}
        \caption{接收数据矩阵\( \mathbf{X} \)(仅绘制了实部)}
        \label{fig_data_x}
    \end{figure}
\end{example}

\begin{solution}
    处理结果如\cref{fig_compressed}所示。\cref{fig_compressed_1} 给出了对 \( \mathbf{X} \) 的行向量进行脉冲压缩的结果:原本具有一定展宽的目标回波在压缩后转变为尖锐的脉冲,在图像中表现为水平直线;同时,由于两个目标对应的直线具有不同的频率起伏,说明其参数值(方位或速度)存在差异。

    \cref{fig_compressed_2} 展示了对 \( \mathbf{X} \) 的行向量进行波束形成的结果:两个目标在参数维上也被压缩为尖锐的脉冲,在图像中对应两条垂直直线,并且这两条直线明显呈现出 Chirp 信号的特征。

    \cref{fig_compressed_3} 则给出了在行向量和列向量上分别进行脉冲压缩与波束形成的联合结果:此时,两个目标在距离维和参数维均被压缩为尖锐脉冲,在图像中表现为两个亮点,其位置正好对应目标的距离和参数值。

    \begin{figure}[htb!]
        \centering
        \begin{subfigure}{.3\textwidth}
            \centering
            \begin{tikzpicture}
                \begin{axis}[
                        xlabel={参数维},
                        ylabel={时间维},
                        enlargelimits=false,
                        width=4.5cm, height=4.5cm,
                        ytick=\empty,
                        xtick=\empty,
                        ticklabel style={font=\small},
                        label style={font=\small},
                        axis on top
                    ]
                    \addplot graphics [
                            xmin=-1, xmax=1, ymin=-1, ymax=1,
                        ] {./img/estimation/est_2.png};
                \end{axis}
            \end{tikzpicture}
            \caption{\( \mathbf{S}\mathbf{X} \)实部}
            \label{fig_compressed_1}
        \end{subfigure}
        \begin{subfigure}{.3\textwidth}
            \centering
            \begin{tikzpicture}
                \begin{axis}[
                        xlabel={参数维},
                        ylabel={时间维},
                        enlargelimits=false,
                        width=4.5cm, height=4.5cm,
                        ytick=\empty,
                        xtick=\empty,
                        ticklabel style={font=\small},
                        label style={font=\small},
                        axis on top
                    ]
                    \addplot graphics [
                            xmin=-1, xmax=1, ymin=-1, ymax=1,
                        ] {./img/estimation/est_3.png};
                \end{axis}
            \end{tikzpicture}
            \caption{\( \mathbf{X}\mathbf{W} \)实部}
            \label{fig_compressed_2}
        \end{subfigure}
        \begin{subfigure}{.3\textwidth}
            \centering
            \begin{tikzpicture}
                \begin{axis}[
                        xlabel={参数维},
                        ylabel={时间维},
                        enlargelimits=false,
                        width=4.5cm, height=4.5cm,
                        ytick=\empty,
                        xtick=\empty,
                        ticklabel style={font=\small},
                        label style={font=\small},
                        axis on top
                    ]
                    \addplot graphics [
                            xmin=-1, xmax=1, ymin=-1, ymax=1,
                        ] {./img/estimation/est_4.png};
                \end{axis}
            \end{tikzpicture}
            \caption{\(  \mathbf{S}\mathbf{X}\mathbf{W} \)幅度}
            \label{fig_compressed_3}
        \end{subfigure}
        \caption{距离维和参数维二重脉冲压缩示意图}
        \label{fig_compressed}
    \end{figure}
\end{solution}

在某些情况下,我们可能并不关心目标的距离信息,而仅希望获得其参数信息(如方位或速度)。此时,可以直接利用波束形成后的输出计算参数维度上的功率谱。具体地,对于参数值 $\omega$,对应输出的功率可表示为
\[
    p_{\omega} = \frac{1}{L}\|\bm{y}_{\omega}\|_2^2
    = \frac{1}{L}\|\mathbf{X}\bm{w}_{\omega}\|_2^2
    = \frac{1}{L}\bm{w}_{\omega}^{\mathrm{H}}\mathbf{X}^{\mathrm{H}}\mathbf{X}\bm{w}_{\omega}
    = \bm{w}_{\omega}^{\mathrm{H}} \mathbf{\Sigma}_{\mathbf{X}} \bm{w}_{\omega},
\]
其中,$\mathbf{\Sigma}_{\mathbf{X}} = \frac{1}{L}\mathbf{X}^{\mathrm{H}}\mathbf{X}$ 为接收信号矩阵的协方差矩阵(一般假设接收信号均值为零)。通过对不同参数值 $\omega$ 进行扫描,即可得到参数维度的功率谱估计,从而实现目标参数的直接估计。

对\cref{fig_data_x}中的数据进行波束形成处理,并计算参数维度的功率谱,结果如\cref{fig_spectrum_cbf}所示。可以看到,功率谱在目标参数值处出现了明显的峰值,并且峰值在参数维度上的位置与\cref{fig_compressed_3}中的目标水平位置一致。通过对功率谱进行峰值检测,即可实现目标参数的估计。

\begin{figure}[htb!]
    \centering
    \begin{tikzpicture}
        \begin{axis}[
                xlabel=参数维, ylabel=归一化功率谱,
                ticklabel style={font=\small},
                label style={font=\small},
                % xtick=\empty,ytick=\empty,
                xmin = -0.5, xmax=0.5, ymin=0, ymax=1.2,
                grid, width=8cm, height=4cm,
                legend cell align=left,
                legend style={
                        anchor=north east,
                        font=\tiny,
                        draw=none,
                        fill=none
                    }
            ]
            \addplot[
                c1,
                thick,
            ] table[x=w, y=p, col sep=comma] {./img/estimation/cbf.csv};
        \end{axis}
    \end{tikzpicture}
    \caption{波束形成功率谱估计}
    \label{fig_spectrum_cbf}
\end{figure}

对于阵列信号而言,波束形成的一大优势是可以直接通过移相器硬件实现。通过控制各阵元的相位,对原始模拟信号加权求和,即可在空间上实现对特定方向的滤波。这样只需一路模数转换器对波束形成后的信号进行采样与处理,便能显著节省硬件资源。参数扫描则通过调节移相器的相位来完成,对应的雷达系统称为无源相控阵雷达(Passive Phased Array Radar)。这种方式的缺点在于无法同时观测多个方向的信号分量,因此在目标数量较多或干扰复杂的环境下,性能容易受限。相比之下,若对各阵元的信号分别采样并数字化,再通过数字处理实现波束合成,则能够同时获得多个方向的信号分量,从而提升系统的灵活性与抗干扰能力,但也会增加硬件开销。这类系统即为有源相控阵雷达(Active Phased Array Radar)。

尽管波束形成方法简单直观,但其分辨率和抗干扰能力有限,尤其在目标间距较近或存在强干扰时,性能会显著下降。对于一个特定的参数值\( \omega \),波束形成的输出可以进一步表示为:
\[
    \bm{y}_{\omega} = \mathbf{X} \bm{w}_{\omega} = \sum_{k=1}^{K} \bm{r}_k \bm{a}_{\omega_k}^{\mathrm{T}} \bm{w}_{\omega} = \sum_{k=1}^{K} \bm{r}_k \frac{\bm{a}_{\omega_k}^{\mathrm{T}} \overline{\bm{a}}_{\omega}}{M}.
\]
从中可以发现,波束形成的输出实际上是各个目标回波信号的加权和,对应的权重为\( \frac{\bm{a}_{\omega_k}^{\mathrm{T}} \overline{\bm{a}}_{\omega}}{M} \)。理想情况下,当目标参数值与滤波器参数值相匹配时,该权重接近1,否则趋于0。但在实际应用中,\( M \)的数值有限,此时即便\( \omega_k \neq \omega \)时,对应的权重也可能较大,导致提取出来的信号中混入了其他目标的成分,从而影响估计精度。

注意到
\[
    \begin{split}
        \frac{\bm{a}_{\omega_k}^{\mathrm{T}} \overline{\bm{a}}_{\omega}}{M} & = \frac{1}{M} \sum_{m=0}^{M-1} e^{j 2 \pi m (\omega_k - \omega)} = \frac{1}{M} \frac{1 - e^{j 2 \pi M (\omega_k - \omega)}}{1 - e^{j 2 \pi (\omega_k - \omega)}} \\
                                                                            & = \frac{1}{M} \frac{\sin(\pi M (\omega_k - \omega))}{\sin(\pi (\omega_k - \omega))} e^{j \pi (M-1)(\omega_k - \omega)}.
    \end{split}
\]
可以发现该权重的取值仅和\( M \)以及参数差值\( \omega_k - \omega \)有关。如\cref{fig_weight}所示,随着\( M \)的增大,权重函数的主瓣变窄,旁瓣降低,从而提升了参数分辨能力。

\begin{figure}[htb!]
    \centering
    \begin{tikzpicture}
        \begin{axis}[
                xlabel=$ \omega_k - \omega $, ylabel=权重幅度,
                ticklabel style={font=\small},
                label style={font=\small},
                grid, smooth,
                xmin = -0.5, xmax=0.5, ymin=0, ymax=1,
                width=10cm, height=5cm,
                legend cell align=left,
                clip=false,
                legend style={
                        anchor=north east,
                        font=\tiny,
                        draw=none,
                        fill=none
                    }
            ]
            \addplot[
                c1,
                thick,
            ] table[x=w, y=a1, col sep=comma] {./img/estimation/weight.csv};
            \addlegendentry{\( M=11 \)};
            \addplot[
                c2,
                thick,
            ] table[x=w, y=a2, col sep=comma] {./img/estimation/weight.csv};
            \addlegendentry{\( M=21 \)};
            \addplot[
                c3,
                thick,
            ] table[x=w, y=a3, col sep=comma] {./img/estimation/weight.csv};
            \addlegendentry{\( M=51 \)};
        \end{axis}
    \end{tikzpicture}
    \caption{不同\( M \)值下波束形成权重函数}
    \label{fig_weight}
\end{figure}

\subsection{和差比幅测角}
无源相控阵节约了硬件开销,但其无法同时观测多个方向的信号分量,有源相控阵虽然提升了灵活性,但硬件成本大大增加。为此,雷达系统中常采用一种折衷的方案,即和差比幅测角(Sum–Difference Amplitude Comparison)方法。 该方法通过将阵列划分为多个子阵,并对各子阵的输出进行加权组合,从而在一定程度上兼顾了硬件复杂度与参数估计性能。

不妨考虑一个均匀线性阵列,其阵元总数为 $2M$,并将其划分为两个子阵,每个子阵包含 $M$ 个阵元。对于给定方位角 $\theta$,和差比幅测角方法构造如下两组滤波器:
\[
    \bm{w}_{\text{sum}} = \frac{1}{2M}
    \begin{bmatrix}
        \bm{w}_1 \\
        \bm{w}_2
    \end{bmatrix}, \quad  \bm{w}_{\text{diff}} = \frac{1}{2M}
    \begin{bmatrix}
        \bm{w}_1 \\
        -\bm{w}_2
    \end{bmatrix},
\]
其中,\( \bm{w}_1 \) 和 \( \bm{w}_2 \) 分别为前\( M \) 个阵元和后\( M \) 个阵元的权向量,有如下形式:
\[
    \bm{w}_1 = \begin{bmatrix}
        1                                           \\
        e^{- j 2 \pi \frac{d \sin \theta}{\lambda}} \\
        \vdots                                      \\
        e^{- j 2 \pi \frac{(M-1) d \sin \theta}{\lambda}}
    \end{bmatrix}, \quad
    \bm{w}_2 = \begin{bmatrix}
        e^{- j 2 \pi \frac{M d \sin \theta}{\lambda}}     \\
        e^{- j 2 \pi \frac{(M+1) d \sin \theta}{\lambda}} \\
        \vdots                                            \\
        e^{- j 2 \pi \frac{(2M-1) d \sin \theta}{\lambda}}
    \end{bmatrix}.
\]
记 $\bm{w}_{\text{sum}}$ 为两个子阵的和波束,$\bm{w}_{\text{diff}}$ 为两个子阵的差波束。对于位于方位角 $\theta_k$ 的目标,其波束形成输出分别为
\[
    \bm{y}_{\text{sum}} =  \bm{r}_k \bm{a}_{\omega_k}^{\mathrm{T}} \bm{w}_{\text{sum}},
    \quad
    \bm{y}_{\text{diff}} = \bm{r}_k \bm{a}_{\omega_k}^{\mathrm{T}} \bm{w}_{\text{diff}}.
\]
可以看出,两者在时域波形上保持完全一致,仅在复数比例因子上有所不同,分别为$\bm{a}_{\omega_k}^{\mathrm{T}} \bm{w}_{\text{sum}}$ 和 $\bm{a}_{\omega_k}^{\mathrm{T}} \bm{w}_{\text{diff}}$。该比例因子同时包含幅度与相位信息,其比值正是和差比幅测角所利用的参数。

为了分析这两个幅度的关系,同样将\( \bm{a}_{\omega_k} \) 拆分为两部分:
\[
    \bm{a}_{\omega_k} =
    \begin{bmatrix}
        \bm{a}_{\omega_k,1} \\
        \bm{a}_{\omega_k,2}
    \end{bmatrix},
\]
其中,
\[
    \bm{a}_{\omega_k,1} = \begin{bmatrix}
        1                                           \\
        e^{j 2 \pi \frac{d \sin \theta_k}{\lambda}} \\
        \vdots                                      \\
        e^{j 2 \pi \frac{(M-1) d \sin \theta_k}{\lambda}}
    \end{bmatrix}, \quad
    \bm{a}_{\omega_k,2} = \begin{bmatrix}
        e^{j 2 \pi \frac{M d \sin \theta_k}{\lambda}}     \\
        e^{j 2 \pi \frac{(M+1) d \sin \theta_k}{\lambda}} \\
        \vdots                                            \\
        e^{j 2 \pi \frac{(2M-1) d \sin \theta_k}{\lambda}}
    \end{bmatrix}.
\]
那么,对于和波束,我们有
\[
    \begin{split}
        \bm{a}_{\omega_k}^{\mathrm{T}} \bm{w}_{\text{sum}} & = \frac{1}{2M}\begin{bmatrix}
                                                                               \bm{a}_{\omega_k,1}^{\mathrm{T}} & \bm{a}_{\omega_k,2}^{\mathrm{T}}
                                                                           \end{bmatrix}
        \begin{bmatrix}
            \bm{w}_1 \\
            \bm{w}_2
        \end{bmatrix}                                                                                                                                                                                                                                                                                  \\
                                                           & = \frac{1}{2M} \bm{a}_{\omega_k,1}^{\mathrm{T}} \bm{w}_1 + \frac{1}{2M} \bm{a}_{\omega_k,2}^{\mathrm{T}} \bm{w}_2                                                                                                                          \\
                                                           & = \frac{1}{2M} \bm{a}_{\omega_k,1}^{\mathrm{T}} \bm{w}_1 + \frac{1}{2M} \left( e^{j 2 \pi \frac{M d \sin \theta_k}{\lambda}} \bm{a}_{\omega_k,1}^{\mathrm{T}} \right) \left( e^{-j 2 \pi \frac{M d \sin \theta}{\lambda}} \bm{w}_1 \right) \\
                                                           & = \frac{1 + e^{j 2\pi \frac{M d (\sin \theta_k - \sin \theta)}{\lambda}}}{2M} \bm{a}_{\omega_k,1}^{\mathrm{T}} \bm{w}_1.
    \end{split}
\]
类似地,对于差波束,有
\[
    \begin{split}
        \bm{a}_{\omega_k}^{\mathrm{T}} \bm{w}_{\text{diff}} & = \frac{1}{2M}\begin{bmatrix}
                                                                                \bm{a}_{\omega_k,1}^{\mathrm{T}} & \bm{a}_{\omega_k,2}^{\mathrm{T}}
                                                                            \end{bmatrix}
        \begin{bmatrix}
            \bm{w}_1 \\
            -\bm{w}_2
        \end{bmatrix}                                                                                                                                                                                                                                                                                     \\
                                                            & = \frac{1}{2M} \bm{a}_{\omega_k,1}^{\mathrm{T}} \bm{w}_1 - \frac{1}{2M} \bm{a}_{\omega_k,2}^{\mathrm{T}} \bm{w}_2                                                                                                                            \\
                                                            & = \frac{1}{2M} \bm{a}_{\omega_k,1}^{\mathrm{T}} \bm{w}_1 - \frac{1}{2M} \left( e^{j 2 \pi \frac{M d \sin \theta_k}{\lambda}} \bm{a}_{\omega_k,1}^{\mathrm{T}} \right) \left( e^{-j 2 \pi \frac{M d \sin \theta_k}{\lambda}} \bm{w}_1 \right) \\
                                                            & = = \frac{1 - e^{j 2\pi \frac{M d (\sin \theta_k - \sin \theta)}{\lambda}}}{2M} \bm{a}_{\omega_k,1}^{\mathrm{T}} \bm{w}_1.
    \end{split}
\]

不妨令\( y_{\text{sum}} \)和\( y_{\text{diff}} \)分别为对\( \bm{y}_{\text{sum}} \)和\( \bm{y}_{\text{diff}} \)进行脉冲压缩后的峰值,则有
\[
    \frac{y_{\text{diff}}}{y_{\text{sum}}} = \frac{1 - e^{j 2\pi \frac{M d (\sin \theta_k - \sin \theta)}{\lambda}}}{1 + e^{j 2\pi \frac{M d (\sin \theta_k - \sin \theta)}{\lambda}}}.
\]
记\( \phi = \pi \frac{M d (\sin \theta_k - \sin \theta)}{\lambda} \),则上式可化简为
\[
    \frac{y_{\text{diff}}}{y_{\text{sum}}} = \frac{1 - e^{j 2 \phi}}{1 + e^{j 2 \phi}} = \frac{e^{-j \phi} - e^{j \phi}}{e^{-j \phi} + e^{j \phi}} = \frac{-2j \sin \phi}{2 \cos \phi} =-j \tan \phi.
\]
因此,两个峰值的虚部之比为
\[
    r = \operatorname{Imag}\left( \frac{y_{\text{diff}}}{y_{\text{sum}}} \right) = - \tan \phi = - \tan \left( \pi \frac{M d (\sin \theta_k - \sin \theta)}{\lambda} \right).
\]

注意到,在\( \theta_k \)与\( \theta \)接近的情况下,有
\[
    \sin \theta_k - \sin \theta = 2 \cos \left( \frac{\theta_k + \theta}{2} \right) \sin \left( \frac{\theta_k - \theta}{2} \right) \approx 2 \cos \theta  \frac{\theta_k - \theta}{2} = \cos \theta (\theta_k - \theta).
\]
将其代入上式,得到
\[
    r \approx - \tan \left( \pi \frac{M d \cos \theta}{\lambda} (\theta_k - \theta) \right),
\]
进一步推导,有
\[
    \theta_k \approx \theta + \frac{\lambda}{\pi M d \cos \theta} \arctan (-r).
\]

\cref{fig_ac}展示了\( \theta \)等于0时,目标角度\( \theta_k \)和和差比幅值\( r \)之间的关系曲线。可以看到,随着目标角度偏离扫描角度,和差比幅值呈现出非线性变化的趋势。通过测量和差比幅值\( r \),并结合上述关系式,即可实现对目标方位角\( \theta_k \)的估计。

\begin{figure}[htb!]
    \centering
    \begin{tikzpicture}
        \begin{axis}[
                xlabel=\( \theta_k(\text{度}) \), ylabel=\( r \),
                ticklabel style={font=\small},
                label style={font=\small},
                grid, xmin=-6, xmax=6,
                width=6cm, height=6cm,
                legend cell align=left,
                legend style={
                        anchor=north east,
                        font=\tiny,
                        draw=none,
                        fill=none
                    }
            ]
            \addplot[
                c1,
                thick,
            ] table[x=x, y=y, col sep=comma] {./img/estimation/ac.csv};
        \end{axis}
    \end{tikzpicture}
    \caption{目标角度与和差比幅值关系曲线}
    \label{fig_ac}
\end{figure}

相较于无源相控阵雷达,和差比幅法需要额外增加一路模数转换器。但这一硬件代价带来的好处是,在主瓣范围内,目标角度能够得到更为精确的测量。此外,和差比幅法的计算复杂度也较低,主要涉及简单的加减运算和脉冲压缩处理,适合实时应用。

\subsection{最小方差无失真响应}
经典波束形成方法并未考虑噪声与干扰的统计特性,因此在复杂环境下性能往往受到限制。为此,最小方差无失真响应(Minimum Variance Distortionless Response, MVDR)方法在设计滤波器时引入了噪声与干扰的协方差矩阵,通过优化权向量以最小化输出功率,同时保证对目标信号无失真响应。该方法的发明者为 Jack Capon,因此有时也称为 Capon 算法。

设 \( \bm{w}_{\omega} \) 为待设计的滤波器权向量,则对应的输出为
\[
    \bm{y}_{\omega} = \mathbf{X} \bm{w}_{\omega}.
\]
直观地,我们希望输出 \(\bm{y}_{\omega}\) 尽可能只包含参数值为 \(\omega\) 的目标信号,同时抑制其他信号与噪声干扰。注意到目标信号对应的导向向量为 \(\bm{a}_{\omega}\),因此目标分量在输出中的占比与 \(|\bm{a}_{\omega}^{\mathrm{T}} \bm{w}_{\omega}|\) 成正比。由此可构造如下优化问题:
\[
    \max_{\bm{w}_{\omega}} \frac{|\bm{a}_{\omega}^{\mathrm{T}} \bm{w}_{\omega}|^2}{ \frac{1}{L} \|\bm{y}_{\omega}\|_2^2},
\]
即在保证输出功率尽可能小的前提下,使目标信号成分最大化。该问题可转化为如下形式的约束优化问题:
\[
    \begin{cases}
        \min_{\bm{w}_{\omega}} \quad & \frac{1}{L} \|\bm{y}_{\omega}\|_2^2               \\
        \text{s.t.} \quad            & \bm{a}_{\omega}^{\mathrm{T}} \bm{w}_{\omega} = 1,
    \end{cases}
\]
其中约束条件保证了对目标信号的无失真响应。

记接收信号的协方差矩阵为
\[
    \mathbf{\Sigma}_{\mathbf{X}} = \frac{1}{L} \mathbf{X}^{\mathrm{H}} \mathbf{X},
\]
则目标函数可重写为
\[
    \frac{1}{L} \|\bm{y}_{\omega}\|_2^2
    = \frac{1}{L} \bm{w}_{\omega}^{\mathrm{H}} \mathbf{X}^{\mathrm{H}} \mathbf{X} \bm{w}_{\omega}
    = \bm{w}_{\omega}^{\mathrm{H}} \mathbf{\Sigma}_{\mathbf{X}} \bm{w}_{\omega}.
\]
利用拉格朗日乘子法可得该优化问题的闭式解为
\[
    \bm{w}_{\omega}
    = \frac{\mathbf{\Sigma}_{\mathbf{X}}^{-1} \overline{\bm{a}}_{\omega}}
    {\overline{\bm{a}}_{\omega}^{\mathrm{H}} \mathbf{\Sigma}_{\mathbf{X}}^{-1} \overline{\bm{a}}_{\omega}}.
\]

遍历所有参数值 \(\omega\),即可构造滤波器组矩阵
\[
    \mathbf{W} = \begin{bmatrix} \bm{w}_{\omega_1} & \bm{w}_{\omega_2} & \cdots \end{bmatrix},
\]
并由此得到波束形成的输出矩阵
\[
    \mathbf{Y} = \mathbf{X} \mathbf{W}.
\]
与经典波束形成类似,可通过计算各列向量的功率来获得参数维度的功率谱:
\[
    p_{\omega}
    = \frac{1}{L} \|\bm{y}_{\omega}\|_2^2
    = \bm{w}_{\omega}^{\mathrm{H}} \mathbf{\Sigma}_{\mathbf{X}} \bm{w}_{\omega}
    = \frac{1}{\overline{\bm{a}}_{\omega}^{\mathrm{H}} \mathbf{\Sigma}_{\mathbf{X}}^{-1} \overline{\bm{a}}_{\omega}}.
\]

\begin{example}
    对\cref{fig_noise_capon}中的包含噪声的数据分别进行 CBF 和 Capon 方法处理,计算对应的二维脉压结果和参数维度的功率谱估计。
    \begin{figure}[htb!]
        \centering
        \begin{tikzpicture}
            \begin{axis}[
                    xlabel={参数维},
                    ylabel={时间维},
                    enlargelimits=false,
                    width=5cm, height=5cm,
                    ytick=\empty,
                    xtick=\empty,
                    ticklabel style={font=\small},
                    label style={font=\small},
                    axis on top
                ]
                \addplot graphics [
                        xmin=-1, xmax=1, ymin=-1, ymax=1,
                    ] {./img/estimation/capon_1.png};
            \end{axis}
        \end{tikzpicture}
        \caption{有噪声数据矩阵\( \mathbf{X} \)(仅绘制了实部)}
        \label{fig_noise_capon}
    \end{figure}
\end{example}
\begin{solution}
    二维脉压结果如\cref{fig_compressed}所示。与经典波束形成相比,Capon 方法在参数维度上展现出更高的分辨率,两个目标在参数维上被压缩为更尖锐的脉冲,并且旁瓣水平显著降低,从而提升了目标的可分辨性和检测性能。
    \begin{figure}[htb!]
        \centering
        \begin{subfigure}{.4\textwidth}
            \centering
            \begin{tikzpicture}
                \begin{axis}[
                        xlabel={参数维},
                        ylabel={时间维},
                        enlargelimits=false,
                        width=5cm, height=5cm,
                        ytick=\empty,
                        xtick=\empty,
                        ticklabel style={font=\small},
                        label style={font=\small},
                        axis on top
                    ]
                    \addplot graphics [
                            xmin=-1, xmax=1, ymin=-1, ymax=1,
                        ] {./img/estimation/capon_2.png};
                \end{axis}
            \end{tikzpicture}
            \caption{CBF}
            \label{fig_compressed_capon_1}
        \end{subfigure}
        \begin{subfigure}{.4\textwidth}
            \centering
            \begin{tikzpicture}
                \begin{axis}[
                        xlabel={参数维},
                        ylabel={时间维},
                        enlargelimits=false,
                        width=5cm, height=5cm,
                        ytick=\empty,
                        xtick=\empty,
                        ticklabel style={font=\small},
                        label style={font=\small},
                        axis on top
                    ]
                    \addplot graphics [
                            xmin=-1, xmax=1, ymin=-1, ymax=1,
                        ] {./img/estimation/capon_3.png};
                \end{axis}
            \end{tikzpicture}
            \caption{Capon}
            \label{fig_compressed_capon_2}
        \end{subfigure}
        \caption{CBF 和 Capon 方法处理后的数据矩阵的幅度图}
        \label{fig_compressed_capon}
    \end{figure}

    参数维度的功率谱估计如\cref{fig_compressed_capon2}所示。可以看到,Capon 方法在参数维度上展现出更高的分辨率,两个目标在参数维上被压缩为更尖锐的脉冲,并且旁瓣水平显著降低,有效提升了目标的可分辨性和检测性能。
    \begin{figure}[htb!]
        \centering
        \begin{tikzpicture}
            \begin{axis}[
                    xlabel={参数维}, ylabel={归一化功率谱},
                    ticklabel style={font=\small},
                    xmin=-0.5, xmax=0.5, ymin=0, ymax=1.2,
                    % xtick=\empty,ytick=\empty,
                    width=8cm, height=4cm,
                    label style={font=\small},
                    grid,
                    legend cell align=left,
                    legend style={
                            anchor=north east,
                            font=\tiny,
                            draw=none,
                            fill=none
                        }
                ]
                \addplot[
                    c1,
                    thick,
                ] table[x=w, y=p1, col sep=comma] {./img/estimation/capon.csv};
                \addlegendentry{CBF}
                \addplot[
                    c2,
                    thick,
                ] table[x=w, y=p2, col sep=comma] {./img/estimation/capon.csv};
                \addlegendentry{Capon}
            \end{axis}
        \end{tikzpicture}
        \caption{CBF 和 Capon 方法处理后的数据矩阵的幅度图}
        \label{fig_compressed_capon2}
    \end{figure}
\end{solution}


在实际应用中,由于接收信号的样本数量有限,协方差矩阵的估计往往不够准确,甚至可能不可逆。为缓解这一问题,可以借鉴岭回归(Ridge Regression)的思想,在目标函数中加入权重向量的范数惩罚项,从而得到如下优化模型:
\[
    \begin{aligned}
        \min_{\bm{w}_{\omega}} \quad &
        \bm{w}_{\omega}^{\mathrm{H}} \mathbf{\Sigma}_{\mathbf{X}} \bm{w}_{\omega}
        + \delta \|\bm{w}_{\omega}\|_2^2                                                 \\
        \text{s.t.} \quad            & \bm{a}_{\omega}^{\mathrm{T}} \bm{w}_{\omega} = 1,
    \end{aligned}
\]
其中 \(\delta > 0\) 为正则化参数。利用拉格朗日乘子法,可得其闭式解为
\[
    \bm{w}_{\omega}
    = \frac{(\mathbf{\Sigma}_{\mathbf{X}} + \delta \mathbf{I})^{-1} \,\overline{\bm{a}}_{\omega}}
    {\overline{\bm{a}}_{\omega}^{\mathrm{H}} (\mathbf{\Sigma}_{\mathbf{X}} + \delta \mathbf{I})^{-1} \,\overline{\bm{a}}_{\omega}}.
\]

由此可见,该方法等价于将原协方差矩阵替换为 \(\mathbf{\Sigma}_{\mathbf{X}} + \delta \mathbf{I}\),在保证无失真约束的同时显著改善了矩阵的条件数,从而增强了算法的鲁棒性。这一技巧通常被称为对角加载(Diagonal Loading)。

\section{子空间方法}
从滤波的角度来看,经典波束形成与 Capon 算法本质上都是通过设计权向量,对接收信号矩阵进行加权叠加,以实现对特定参数下的目标回波的增强,以及对噪声和干扰的抑制。不同之处在于,Capon 算法利用接收信号的协方差矩阵,在保证无失真响应的同时最小化输出功率,因此在噪声和干扰抑制方面较经典波束形成更为有效。然而,Capon 算法依然受限于“滤波”这一框架:对于白噪声而言,无论滤波器如何设计,其输出噪声功率的下限始终等于噪声本身。因此,即便在高信噪比条件下,Capon 谱的分辨率仍受限于通道数\( M \),难以满足高精度参数估计的需求。

为突破 Capon 算法的分辨率瓶颈,研究者提出了基于子空间的高分辨率方法。其核心思想不再依赖滤波器权向量的设计,而是通过对接收信号协方差矩阵进行特征分解,将信号与噪声划分到互相正交的子空间中,并利用这种正交性实现参数估计。借助这一几何结构,子空间方法能够获得远超传统滤波框架的分辨能力。典型代表包括 MUSIC 和 ESPRIT 方法,它们利用信号子空间与噪声子空间的正交特性,实现了高分辨率的参数估计。

\subsection{MUSIC方法}
考虑含噪接收信号模型
\[
    \mathbf{X} = \mathbf{R}\mathbf{A}^{\mathrm{T}} + \mathbf{N} \in \mathbb{C}^{L \times M},
\]
其中 \(\mathbf{N}\) 为加性噪声。若噪声为零均值、方差为 \(\sigma^2\) 的白噪声,则其协方差矩阵近似为对角矩阵:
\[
    \mathbf{\Sigma}_{\mathbf{N}}
    = \frac{1}{L}\mathbf{N}^{\mathrm{H}}\mathbf{N}
    \approx \sigma^2 \mathbf{I} \in \mathbb{C}^{M \times M}.
\]

对于信号部分,一般假设各目标对应的距离互不相同,即 \(\mathbf{R}\) 的列向量互相正交。此时信号协方差矩阵可写为
\[
    \mathbf{\Sigma}_{\mathbf{S}}
    = \frac{1}{L}\,\overline{\mathbf{A}}\,\mathbf{R}^{\mathrm{H}}\mathbf{R}\,\mathbf{A}^{\mathrm{T}}
    = \overline{\mathbf{A}}
    \left(\frac{1}{L}\mathbf{R}^{\mathrm{H}}\mathbf{R}\right)
    \mathbf{A}^{\mathrm{T}}
    \approx \overline{\mathbf{A}}\,\mathbf{\Sigma}_{\mathbf{R}}\,\mathbf{A}^{\mathrm{T}},
\]
其中
\[
    \mathbf{\Sigma}_{\mathbf{R}}
    = \frac{1}{L}\mathbf{R}^{\mathrm{H}}\mathbf{R}
    = \operatorname{diag}(\sigma_1^2,\sigma_2^2,\cdots,\sigma_K^2),
\]
其对角元素即为各目标的功率。又因为信号与噪声独立,即满足\(\mathbf{R}\mathbf{A}^{\mathrm{T}}\mathbf{N}^{\mathrm{H}} \approx \mathbf{0}\),则整个接收信号的协方差矩阵为
\[
    \begin{split}
        \mathbf{\Sigma}_{\mathbf{X}}
         & = \frac{1}{L}\big(\mathbf{R}\mathbf{A}^{\mathrm{T}}+\mathbf{N}\big)^{\mathrm{H}}
        \big(\mathbf{R}\mathbf{A}^{\mathrm{T}}+\mathbf{N}\big)                                   \\
         & \approx \frac{1}{L}\mathbf{A}\mathbf{R}^{\mathrm{H}}\mathbf{R}\mathbf{A}^{\mathrm{T}}
        + \frac{1}{L}\mathbf{N}^{\mathrm{H}}\mathbf{N}                                           \\
         & = \mathbf{\Sigma}_{\mathbf{S}} + \mathbf{\Sigma}_{\mathbf{N}}                         \\
         & \approx \overline{\mathbf{A}}\,\mathbf{\Sigma}_{\mathbf{R}}\,\mathbf{A}^{\mathrm{T}}
        + \sigma^2 \mathbf{I}.
    \end{split}
\]

考虑到矩阵 \(\mathbf{A}\) 的列向量由不同频率的复指数信号组成,而当阵元数 \(M\) 足够大时,不同频率的复指数信号近似正交。因此,对于协方差矩阵 \(\mathbf{\Sigma}_{\mathbf{X}}\),不难验证 \(\mathbf{A}\) 的列向量的共轭 \(\overline{\bm{a}}_{\omega_k}\) 是其特征向量,对应的特征值为 \(\sigma_k^2 M + \sigma^2\)。具体推导如下:
\[
    \begin{split}
        \mathbf{\Sigma}_{\mathbf{X}} \overline{\bm{a}}_{\omega_k}
         & \approx \left( \overline{\mathbf{A}} \mathbf{\Sigma}_{\mathbf{R}} \mathbf{A}^{\mathrm{T}} + \sigma^2 \mathbf{I} \right) \overline{\bm{a}}_{\omega_k}
        = \overline{\mathbf{A}} \mathbf{\Sigma}_{\mathbf{R}} \left( \mathbf{A}^{\mathrm{T}} \overline{\bm{a}}_{\omega_k} \right) + \sigma^2 \overline{\bm{a}}_{\omega_k}                            \\
         & = \overline{\mathbf{A}} \mathbf{\Sigma}_{\mathbf{R}} \begin{bmatrix}
                                                                    0      \\
                                                                    \vdots \\
                                                                    M      \\
                                                                    \vdots \\
                                                                    0
                                                                \end{bmatrix} + \sigma^2 \overline{\bm{a}}_{\omega_k} = \overline{\mathbf{A}} \begin{bmatrix}
                                                                                                                                                  0            \\
                                                                                                                                                  \vdots       \\
                                                                                                                                                  \sigma_k^2 M \\
                                                                                                                                                  \vdots       \\
                                                                                                                                                  0
                                                                                                                                              \end{bmatrix} + \sigma^2 \overline{\bm{a}}_{\omega_k} \\
         & = \sigma_k^2 M \overline{\bm{a}}_{\omega_k} + \sigma^2 \overline{\bm{a}}_{\omega_k}
        = \left( \sigma_k^2 M + \sigma^2 \right) \overline{\bm{a}}_{\omega_k}.
    \end{split}
\]

另一方面,一般有 \(M > K\),因此 \(\mathbf{\Sigma}_{\mathbf{X}}\) 至少还存在 \(M-K\) 个不在 \(\mathbf{A}\) 列空间内的特征向量,记为 \(\bm{e}_{K+1},\bm{e}_{K+2},\dots,\bm{e}_M\)(方便起见,令\( \bm{e}_k \)的模长与\( \bm{a}_{\omega_k}\) 模长相同,都为\( M \))。由于协方差矩阵为共轭对称矩阵,其特征向量必然正交,因此有:
\[
    \mathbf{\Sigma}_{\mathbf{X}} \bm{e}_i
    \approx \big(\overline{\mathbf{A}}\,\mathbf{\Sigma}_{\mathbf{R}}\,\mathbf{A}^{\mathrm{T}} + \sigma^2 \mathbf{I}\big)\bm{e}_i
    = \sigma^2 \bm{e}_i,\qquad i=K+1,\dots,M.
\]

综上,\(\mathbf{\Sigma}_{\mathbf{X}}\) 的特征结构可分为两类:由 \(\overline{\bm{a}}_{\omega_1},\dots,\overline{\bm{a}}_{\omega_K}\) 张成的信号子空间,对应特征值为\(\sigma_1^2 M+\sigma^2,\dots,\sigma_K^2 M+\sigma^2\);由 \(\bm{e}_{K+1},\dots,\bm{e}_M\) 张成的噪声子空间,对应特征值均为 \(\sigma^2\);且信号子空间与噪声子空间彼此正交。

又因为信号对应的特征值满足\(\sigma_k^2 M + \sigma^2 > \sigma^2\),故可通过对 \(\mathbf{\Sigma}_{\mathbf{X}}\) 进行特征分解,提取最大的 \(K\) 个特征值及其特征向量来构造信号子空间,其余部分即为噪声子空间。具体地,设
\[
    \mathbf{\Sigma}_{\mathbf{X}}
    = \mathbf{V}\mathbf{\Lambda}\mathbf{V}^{\mathrm{H}}
    = \begin{bmatrix}
        \mathbf{V}_{s} & \mathbf{V}_{n}
    \end{bmatrix}
    \begin{bmatrix}
        \mathbf{\Lambda}_{s} & \mathbf{0}           \\
        \mathbf{0}           & \mathbf{\Lambda}_{n}
    \end{bmatrix}
    \begin{bmatrix}
        \mathbf{V}_{s}^{\mathrm{H}} \\
        \mathbf{V}_{n}^{\mathrm{H}}
    \end{bmatrix},
\]
其中,\(\mathbf{V}_{s}\) 包含前 \(K\) 个最大特征值对应的特征向量,构成信号子空间;\(\mathbf{V}_{n}\) 包含剩余 \(M-K\) 个特征值对应的特征向量,构成噪声子空间。

% 考虑到\( \mathbf{A} \)的列向量为不同频率的复指数信号,而不同频率的复指数信号随着\( M \)的增大趋于正交。因此,对于协方差矩阵\( \mathbf{\Sigma}_{\mathbf{X}} \),不难验证,矩阵\( \mathbf{A} \)的列向量的共轭\( \overline{\bm{a}}_{\omega_k} \)为其特征向量,且对应的特征值为\( \sigma_k^2 M + \sigma^2 \):
% \[
%     \begin{split}
%         \mathbf{\Sigma}_{\mathbf{X}} \overline{\bm{a}}_{\omega_k}
%          & \approx \left( \overline{\mathbf{A}} \mathbf{\Sigma}_{\mathbf{R}} \mathbf{A}^{\mathrm{T}} + \sigma^2 \mathbf{I} \right) \overline{\bm{a}}_{\omega_k}
%         = \overline{\mathbf{A}} \mathbf{\Sigma}_{\mathbf{R}} \left( \mathbf{A}^{\mathrm{T}} \overline{\bm{a}}_{\omega_k} \right) + \sigma^2 \overline{\bm{a}}_{\omega_k}                            \\
%          & = \overline{\mathbf{A}} \mathbf{\Sigma}_{\mathbf{R}} \begin{bmatrix}
%                                                                     0      \\
%                                                                     \vdots \\
%                                                                     M      \\
%                                                                     \vdots \\
%                                                                     0
%                                                                 \end{bmatrix} + \sigma^2 \overline{\bm{a}}_{\omega_k} = \overline{\mathbf{A}} \begin{bmatrix}
%                                                                                                                                                   0            \\
%                                                                                                                                                   \vdots       \\
%                                                                                                                                                   \sigma_k^2 M \\
%                                                                                                                                                   \vdots       \\
%                                                                                                                                                   0
%                                                                                                                                               \end{bmatrix} + \sigma^2 \overline{\bm{a}}_{\omega_k} \\
%          & = \sigma_k^2 M \overline{\bm{a}}_{\omega_k} + \sigma^2 \overline{\bm{a}}_{\omega_k}
%         = \left( \sigma_k^2 M + \sigma^2 \right) \overline{\bm{a}}_{\omega_k}.
%     \end{split}
% \]
% 一般来说,\( M > K \),因此矩阵\( \mathbf{\Sigma}_{\mathbf{X}} \)至少有\( M-K \)个特征向量不在\( \mathbf{A} \)的列空间内。记这些特征向量为\( \bm{e}_{K+1}, \bm{e}_{K+2}, \ldots, \bm{e}_M \),且都与\( \mathbf{A} \)的列向量的共轭正交,可以验证,这些特征向量对应的特征值均为\( \sigma^2 \):
% \[
%     \begin{split}
%         \mathbf{\Sigma}_{\mathbf{X}} \bm{e}_i
%          & \approx \left( \overline{\mathbf{A}} \mathbf{\Sigma}_{\mathbf{R}} \mathbf{A}^{\mathrm{T}} + \sigma^2 \mathbf{I} \right) \bm{e}_i
%         = \sigma^2 \bm{e}_i.
%     \end{split}
% \]

% 综上,矩阵\( \mathbf{\Sigma}_{\mathbf{X}} \)的特征向量和特征值可以划分为两组:\( \overline{\bm{a}}_{\omega_1}, \overline{\bm{a}}_{\omega_2}, \cdots, \overline{\bm{a}}_{\omega_K} \)对应的特征值为\( \sigma_1^2 M + \sigma^2, \sigma_2^2 M + \sigma^2, \ldots, \sigma_K^2 M + \sigma^2 \);而\( \bm{e}_{K+1}, \bm{e}_{K+2}, \ldots, \bm{e}_M \)对应的特征值均为\( \sigma^2 \)。这两组特征向量张成的子空间分别被称作信号子空间和噪声子空间。而共轭对称矩阵的特征向量必然正交,因此信号子空间与噪声子空间之间同样是互相正交的。

% 此外,注意到信号对应的特征值\( \sigma^2_k M + \sigma^2 > \sigma^2 \),因此可以通过对协方差矩阵进行特征分解,并提取最大的\( K \)个特征值及其对应的特征向量,用来作为信号子空间,剩余的部分作为噪声子空间。具体地,设协方差矩阵的特征分解为
% \[
%     \mathbf{\Sigma}_{\mathbf{X}} = \mathbf{U} \mathbf{\Lambda} \mathbf{U}^{\mathrm{H}} = \begin{bmatrix}
%         \mathbf{U}_{s} & \mathbf{U}_{n}
%     \end{bmatrix} \begin{bmatrix}
%         \mathbf{\Lambda}_{s} & \mathbf{0}           \\
%         \mathbf{0}           & \mathbf{\Lambda}_{n}
%     \end{bmatrix} \begin{bmatrix}
%         \mathbf{U}_{s}^{\mathrm{H}} \\
%         \mathbf{U}_{n}^{\mathrm{H}}
%     \end{bmatrix},
% \]
% 其中,\( \mathbf{U}_{s} \)包含了前\( K \)个最大的特征值对应的特征向量,张成信号子空间;而\( \mathbf{U}_{n} \)包含了剩余的\( M-K \)个特征值对应的特征向量,张成噪声子空间。

Mutiple Signal Classification (MUSIC) 方法正是利用了信号子空间与噪声子空间的正交性,实现对目标参数的高分辨率估计。给定一个导向矢量\( \bm{a}_{\omega} \),如果其对应的参数值\( \omega \)恰好是某个目标的参数值\( \omega_k \),那么该导向矢量必然位于信号子空间内,因此与噪声子空间正交。反之,如果\( \omega \)不是任何目标的参数值,那么导向矢量\( \bm{a}_{\omega} \)将位于噪声子空间内。

进一步地,可以通过将该导向矢量投影到噪声子空间来衡量其与噪声子空间的正交程度,即\( \|\mathbf{V}_{n}^{\mathrm{H}} \overline{\bm{a}}_{\omega}\|_2^2\)。如果该值接近于零,则说明\( \bm{a}_{\omega} \)与噪声子空间正交,进而推断出\( \omega \)可能是某个目标的参数值,反之亦然。基于这一思路,MUSIC 方法定义了如下的谱函数:
\[
    p_{\omega} = \frac{1}{\|\mathbf{V}_{n}^{\mathrm{H}} \overline{\bm{a}}_{\omega}\|_2^2} = \frac{1}{\overline{\bm{a}}_{\omega}^{\mathrm{H}} \mathbf{V}_{n} \mathbf{V}_{n}^{\mathrm{H}} \overline{\bm{a}}_{\omega}}.
\]
遍历所有参数值\( \omega \),即可得到参数维度的功率谱估计。谱函数在目标参数值处会出现尖锐的峰值,从而实现对目标参数的高分辨率估计。

如\cref{fig_music_eg}所示,在强噪声环境下,CBF、Capon 与 MUSIC 三种方法的功率谱估计结果差异显著。可以观察到,MUSIC 方法的谱峰更加尖锐,旁瓣水平更低。此外,由于引入了信号子空间与噪声子空间,MUSIC 方法能够有效抑制噪声,因此其功率谱中的底噪明显低于 CBF 与 Capon,显著提升了目标的可检测性。
\begin{figure}[htb!]
    \centering
    \begin{tikzpicture}
        \begin{axis}[
                xlabel={参数维}, ylabel={归一化功率谱},
                ticklabel style={font=\small},
                xmin=-0.5, xmax=0.5, ymin=0, ymax=1.2,
                % xtick=\empty,ytick=\empty,
                width=8cm, height=4cm,
                label style={font=\small},
                grid,
                legend cell align=left,
                legend style={
                        anchor=north east,
                        font=\tiny,
                        draw=none,
                        fill=none
                    }
            ]
            \addplot[
                c1,
                thick,
            ] table[x=f, y=p1, col sep=comma] {./img/estimation/music_spectrum.csv};
            \addlegendentry{CBF}
            \addplot[
                c2,
                thick,
            ] table[x=f, y=p2, col sep=comma] {./img/estimation/music_spectrum.csv};
            \addlegendentry{Capon}
            \addplot[
                c3,
                thick,
            ] table[x=f, y=p3, col sep=comma] {./img/estimation/music_spectrum.csv};
            \addlegendentry{MUSIC}
        \end{axis}
    \end{tikzpicture}
    \caption{不同波束形成算法的归一化功率谱比较}
    \label{fig_music_eg}
\end{figure}

与 CBF 和 Capon 方法不同,MUSIC 方法的推导过程中并未涉及滤波器权向量的设计,因此难以直接得到如 \cref{fig_compressed_capon} 所示的二维脉压结果。然而,MUSIC 的核心在于利用接收信号矩阵的低秩结构。基于这一点,可以通过对接收信号矩阵进行截断奇异值分解(Truncated SVD),即仅保留前 \(K\) 个奇异值及其对应的奇异向量,对数据进行重构,从而实现信号成分的提取与噪声的抑制。

设接收信号矩阵的奇异值分解为
\[
    \mathbf{X} = \mathbf{U}\mathbf{\Lambda}\mathbf{V}^{\mathrm{H}},
\]
则其协方差矩阵可表示为
\[
    \mathbf{\Sigma}_{\mathbf{X}}
    = \tfrac{1}{L}\mathbf{X}^{\mathrm{H}}\mathbf{X}
    = \tfrac{1}{L}\mathbf{V}\mathbf{\Lambda}^{\mathrm{H}}\mathbf{\Lambda}\mathbf{V}^{\mathrm{H}}.
\]
其中,\(\mathbf{\Lambda}\in\mathbb{R}^{M\times M}\) 为对角矩阵:
\[
    \mathbf{\Lambda} = \begin{bmatrix}
        \lambda_1 &        &           \\
                  & \ddots &           \\
                  &        & \lambda_M \\
    \end{bmatrix},
\]
从而有
\[
    \mathbf{\Lambda}^{\mathrm{H}} \mathbf{\Lambda} = \begin{bmatrix}
        \lambda_1^2 &        &             \\
                    & \ddots &             \\
                    &        & \lambda_M^2
    \end{bmatrix}.
\]

由此可见,右奇异向量矩阵 \(\mathbf{V}\) 正是协方差矩阵的特征向量,而奇异值的平方 \(\lambda_i^2\) 与协方差矩阵的特征值成正比。因此,前 \(K\) 个最大的奇异值对应于信号子空间,其余则对应于噪声子空间。据此,通过保留前 \(K\) 个奇异值及其对应的奇异向量,可近似提取信号子空间并抑制噪声。降噪后的数据矩阵为
\[
    \hat{\mathbf{X}} = \mathbf{U}_K \mathbf{\Lambda}_K \mathbf{V}_K^{\mathrm{H}},
\]
其中,\(\mathbf{U}_K\) 与 \(\mathbf{V}_K\) 分别由 \(\mathbf{U}\)、\(\mathbf{V}\) 的前 \(K\) 列构成,\(\mathbf{\Lambda}_K\) 为前 \(K\) 个奇异值构成的对角矩阵:
\[
    \mathbf{\Lambda}_K =
    \begin{bmatrix}
        \lambda_1 &           &        &           \\
                  & \lambda_2 &        &           \\
                  &           & \ddots &           \\
                  &           &        & \lambda_K
    \end{bmatrix} \in \mathbb{R}^{K\times K}.
\]

\begin{example}
    对\cref{fig_noise_music}中的包含噪声的数据进行SVD分解,并保留前\( K=2 \)个奇异值及其对应的奇异向量,计算降噪后的数据矩阵\( \hat{\mathbf{X}} \)。
    \begin{figure}[htb!]
        \centering
        \begin{tikzpicture}
            \begin{axis}[
                    xlabel={参数维},
                    ylabel={时间维},
                    enlargelimits=false,
                    width=5cm, height=5cm,
                    ytick=\empty,
                    xtick=\empty,
                    ticklabel style={font=\small},
                    label style={font=\small},
                    axis on top
                ]
                \addplot graphics [
                        xmin=-1, xmax=1, ymin=-1, ymax=1,
                    ] {./img/estimation/music_1.png};
            \end{axis}
        \end{tikzpicture}
        \caption{有噪声数据矩阵\( \mathbf{X} \)(仅绘制了实部)}
        \label{fig_noise_music}
    \end{figure}
\end{example}
\begin{solution}
    降噪后的数据矩阵如\cref{fig_denoised}所示。可以看到,经过截断奇异值分解后,数据矩阵中的噪声成分被有效抑制,目标信号得以更清晰地呈现。
    \begin{figure}[htb!]
        \centering
        \begin{tikzpicture}
            \begin{axis}[
                    xlabel={参数维},
                    ylabel={时间维},
                    enlargelimits=false,
                    width=5cm, height=5cm,
                    ytick=\empty,
                    xtick=\empty,
                    ticklabel style={font=\small},
                    label style={font=\small},
                    axis on top
                ]
                \addplot graphics [
                        xmin=-1, xmax=1, ymin=-1, ymax=1,
                    ] {./img/estimation/music_2.png};
            \end{axis}
        \end{tikzpicture}
        \caption{降噪后的数据矩阵\( \hat{\mathbf{X}} \)(仅绘制了实部)}
        \label{fig_denoised}
    \end{figure}

    可以对降噪后的数据矩阵先后采用匹配滤波和 Capon 算法进行二维脉压处理,从而获得更清晰的目标回波图像。需要注意的是,此时接收信号矩阵的协方差矩阵已不再满秩,因此在实现 Capon 算法时应引入对角加载技术。各算法处理结果的幅度图如\cref{fig_compressed_music}所示。由图可见,经过降噪处理后,目标信号得到更清晰的展现,旁瓣水平显著降低,从而有效提升了目标的分辨能力与检测性能。

    \begin{figure}[htb!]
        \centering
        \begin{subfigure}{.32\textwidth}
            \centering
            \begin{tikzpicture}
                \begin{axis}[
                        xlabel={参数维},
                        ylabel={时间维},
                        enlargelimits=false,
                        width=5cm, height=5cm,
                        ytick=\empty,
                        xtick=\empty,
                        ticklabel style={font=\small},
                        label style={font=\small},
                        axis on top
                    ]
                    \addplot graphics [
                            xmin=-1, xmax=1, ymin=-1, ymax=1,
                        ] {./img/estimation/music_3.png};
                \end{axis}
            \end{tikzpicture}
            \caption{CBF}
            \label{fig_2d_compress_denoise_1}
        \end{subfigure}
        \begin{subfigure}{.32\textwidth}
            \centering
            \begin{tikzpicture}
                \begin{axis}[
                        xlabel={参数维},
                        ylabel={时间维},
                        enlargelimits=false,
                        width=5cm, height=5cm,
                        ytick=\empty,
                        xtick=\empty,
                        ticklabel style={font=\small},
                        label style={font=\small},
                        axis on top
                    ]
                    \addplot graphics [
                            xmin=-1, xmax=1, ymin=-1, ymax=1,
                        ] {./img/estimation/music_4.png};
                \end{axis}
            \end{tikzpicture}
            \caption{Capon}
            \label{fig_2d_compress_denoise_2}
        \end{subfigure}
        \begin{subfigure}{.32\textwidth}
            \centering
            \begin{tikzpicture}
                \begin{axis}[
                        xlabel={参数维},
                        ylabel={时间维},
                        enlargelimits=false,
                        width=5cm, height=5cm,
                        ytick=\empty,
                        xtick=\empty,
                        ticklabel style={font=\small},
                        label style={font=\small},
                        axis on top
                    ]
                    \addplot graphics [
                            xmin=-1, xmax=1, ymin=-1, ymax=1,
                        ] {./img/estimation/music_5.png};
                \end{axis}
            \end{tikzpicture}
            \caption{SVD降噪}
            \label{fig_2d_compress_denoise_3}
        \end{subfigure}
        \caption{不同方法处理后的数据矩阵的幅度图}
        \label{fig_compressed_music}
    \end{figure}
\end{solution}

接下来,我们定性地讨论一下 MUSIC 方法为何优于 Capon 方法。注意到,Capon 方法的功率谱输出为
\[
    p_{\omega}^{\text{Capon}} = \frac{1}{\overline{\bm{a}}_{\omega}^{\mathrm{H}} \mathbf{\Sigma}_{\mathbf{X}}^{-1} \overline{\bm{a}}_{\omega}}.
\]
又因为协方差矩阵有如下特征分解:
\[
    \mathbf{\Sigma}_{\mathbf{X}}
    = \mathbf{V} \mathbf{\Lambda} \mathbf{V}^{\mathrm{H}},
\]
其中
\[
    \begin{split}
        \mathbf{V}       & \approx \frac{1}{\sqrt{M}}\begin{bmatrix}
                                                         \overline{\bm{a}}_1 & \cdots & \overline{\bm{a}}_K & \bm{e}_{K+1} & \cdots & \bm{e}_M
                                                     \end{bmatrix}, \\
        \mathbf{\Lambda} & \approx \operatorname{diag}(\sigma_1^2 M + \sigma^2, \cdots, \sigma_K^2 M + \sigma^2, \underbrace{\sigma^2, \cdots, \sigma^2}_{M-K}).
    \end{split}
\]
基于特征分解,可以方便地得到\( \mathbf{\Sigma}_{\mathbf{X}} \) 的逆为
\[
    \mathbf{\Sigma}_{\mathbf{X}}^{-1} \approx \mathbf{V} \mathbf{\Lambda}^{-1} \mathbf{V}^{\mathrm{H}},
\]
其中
\[
    \mathbf{\Lambda}^{-1}
    = \operatorname{diag}\left(\frac{1}{\sigma_1^2 M + \sigma^2}, \cdots, \frac{1}{\sigma_K^2 M + \sigma^2}, \underbrace{\frac{1}{\sigma^2}, \cdots, \frac{1}{\sigma^2}}_{M-K}\right).
\]

因此,对于目标参数 \(\overline{\bm{a}}_{\omega_k}\),Capon 方法的功率谱值为
\[
    p_{\omega_k}^{\text{Capon}}
    \approx \frac{1}{\tfrac{M}{\sigma_k^2 M + \sigma^2}}
    = \frac{\sigma_k^2 M + \sigma^2}{M},
\]
而对于非目标参数 \(\omega \neq \omega_k\),有
\[
    p_{\omega}^{\text{Capon}}
    \approx \frac{1}{\tfrac{M}{\sigma^2}}
    = \frac{\sigma^2}{M}.
\]
由此可见,Capon 谱峰与底噪的比值为
\[
    \frac{p_{\omega_k}^{\text{Capon}}}{p_{\omega}^{\text{Capon}}}
    \approx \frac{\sigma_k^2 M}{\sigma^2} + 1.
\]

对于 MUSIC 方法,其功率谱函数定义为
\[
    p_{\omega}^{\text{MUSIC}}
    = \frac{1}{\overline{\bm{a}}_{\omega}^{\mathrm{H}}
    \mathbf{V}_n \mathbf{V}_n^{\mathrm{H}}
    \overline{\bm{a}}_{\omega}}.
\]
注意到
\[
    \mathbf{V}_n \mathbf{V}_n^{\mathrm{H}}
    = \mathbf{V}\,\tilde{\mathbf{\Lambda}}\,\mathbf{V}^{\mathrm{H}}, \qquad
    \tilde{\mathbf{\Lambda}}
    = \operatorname{diag}(\underbrace{0,\dots,0}_{K},
    \underbrace{1,\dots,1}_{M-K}),
\]
因此,对于目标参数 \(\overline{\bm{a}}_{\omega_k}\),分母为零,
\[
    p_{\omega_k}^{\text{MUSIC}} \approx +\infty,
\]
而对于非目标参数值 \(\omega \neq \omega_k\),有
\[
    p_{\omega}^{\text{MUSIC}} \approx \frac{1}{M}.
\]
故 MUSIC 谱峰与底噪之比理论上趋于无穷大,所以归一化之后其谱峰更加尖锐。

注意到,Capon 与 MUSIC 的谱函数可统一表示为
\[
    p_{\omega} = \frac{1}{\overline{\bm{a}}_{\omega}^{\mathrm{H}}
        \mathbf{W}\,\overline{\bm{a}}_{\omega}},
\]
其中 Capon 方法对应 \(\mathbf{W} = \mathbf{\Sigma}_{\mathbf{X}}^{-1}\),即对信号与噪声赋予与功率成反比的加权;而 MUSIC 方法对应 \(\mathbf{W} = \mathbf{U}_n \mathbf{U}_n^{\mathrm{H}}\),是噪声子空间的投影矩阵,其权重仅取 \(0/1\),本质上是一种非线性划分。正是这种权重结构的差异,使得 MUSIC 能够实现比 Capon 更高的分辨率。

\subsection{ESPRIT方法}
注意到CBF、Capon与MUSIC方法均需要对参数空间进行遍历,当对参数的估计精度要求较高时,计算量会显著增加。而Estimation of Signal Parameters via Rotational Invariance Techniques (ESPRIT)方法则通过巧妙地构造两个具有平移关系的子阵列,利用其旋转不变性来实现对参数的直接估计,从而避免了参数空间的遍历。

令\( \bm{a}_{\omega_k} \)为第\( k \)个目标的导向矢量,有如下表示
\[
    \bm{a}_{\omega_k}
    = \begin{bmatrix}
        1 & e^{j2\pi\omega_k} & \cdots & e^{j2\pi(M-1)\omega_k}
    \end{bmatrix}^{\mathrm{T}}.
\]
取导向矢量的前\( M-1 \)个元素与后\( M-1 \)个元素,分别构成两个子导向矢量:
\[
    \begin{split}
         & \bm{a}_{\omega_k, 1} = \begin{bmatrix}
                                      1 & e^{j2\pi\omega_k} & \cdots & e^{j2\pi(M-2)\omega_k}
                                  \end{bmatrix}^{\mathrm{T}},                  \\
         & \bm{a}_{\omega_k, 2} = \begin{bmatrix}
                                      e^{j2\pi\omega_k} & e^{j2\pi2\omega_k} & \cdots & e^{j2\pi(M-1)\omega_k}
                                  \end{bmatrix}^{\mathrm{T}}.
    \end{split}
\]
显然,这两个子导向矢量满足如下的旋转不变性关系:
\[
    \bm{a}_{\omega_k, 2} = \bm{a}_{\omega_k, 1} e^{j2\pi\omega_k}.
\]
将所有目标的子导向矢量按列堆叠,分别构成两个子阵列的导向矩阵:
\[
    \mathbf{A}_1 = \begin{bmatrix}
        \bm{a}_{\omega_1, 1} & \bm{a}_{\omega_2, 1} & \cdots & \bm{a}_{\omega_K, 1}
    \end{bmatrix}, \quad
    \mathbf{A}_2 = \begin{bmatrix}
        \bm{a}_{\omega_1, 2} & \bm{a}_{\omega_2, 2} & \cdots & \bm{a}_{\omega_K, 2}
    \end{bmatrix}.
\]
显然,这两个子阵列的导向矩阵同样满足旋转不变性关系:
\[
    \mathbf{A}_2 = \mathbf{A}_1 \mathbf{\Phi},
\]
其中
\[
    \mathbf{\Phi} = \operatorname{diag}\big(e^{j2\pi\omega_1}, e^{j2\pi\omega_2}, \ldots, e^{j2\pi\omega_K}\big).
\]

假设接收信号矩阵\( \mathbf{X} = \mathbf{R} \mathbf{A}^{\mathrm{T}} + \mathbf{N} \)有如下SVD分解
\[
    \mathbf{X} = \mathbf{U} \mathbf{\Lambda} \mathbf{V}^{\mathrm{H}} = \begin{bmatrix}
        \mathbf{U}_s & \mathbf{U}_n
    \end{bmatrix} \begin{bmatrix}
        \mathbf{\Lambda}_s & \mathbf{0}         \\
        \mathbf{0}         & \mathbf{\Lambda}_n
    \end{bmatrix} \begin{bmatrix}
        \mathbf{V}_s^{\mathrm{H}} \\
        \mathbf{V}_n^{\mathrm{H}}
    \end{bmatrix}.
\]
其中\( \mathbf{V}_s \in \mathbb{C}^{M \times K} \) 的列向量张成了信号子空间,理论上与导向矩阵\( \mathbf{A} \)的列空间相同,因此存在一个可逆矩阵\( \mathbf{T} \in \mathbb{C}^{K \times K} \),使得
\[
    \mathbf{V}_s = \mathbf{A} \mathbf{T}.
\]
类似地,取\( \mathbf{V}_s \)的前\( M-1 \)行与后\( M-1 \)行,分别构成两个子矩阵,记作\( \mathbf{V}_1 \in \mathbb{C}^{(M-1) \times K} \)和\( \mathbf{V}_2 \in \mathbb{C}^{(M-1) \times K} \),则有
\[
    \mathbf{V}_1 = \mathbf{A}_1 \mathbf{T}, \quad \mathbf{V}_2 = \mathbf{A}_2 \mathbf{T}.
\]

结合旋转不变性关系\( \mathbf{A}_2 = \mathbf{A}_1 \mathbf{\Phi} \),可得
\[
    \mathbf{V}_2 = \mathbf{A}_1 \mathbf{\Phi} \mathbf{T} = \mathbf{V}_1 \mathbf{T}^{-1} \mathbf{\Phi} \mathbf{T}.
\]
记\( \mathbf{\Psi} = \mathbf{T}^{-1} \mathbf{\Phi} \mathbf{T} \),则有
\[
    \mathbf{V}_2 = \mathbf{V}_1 \mathbf{\Psi},
\]
即SVD分解得到的信号子空间的两个子矩阵同样满足旋转不变性关系。

进一步地,由于矩阵 \( \mathbf{\Phi} \) 与 \( \mathbf{\Psi} \) 相似,因此它们具有完全相同的特征值。又因为 \( \mathbf{\Phi} \) 的特征值正好等于其对角线元素 \( e^{j2\pi\omega_k} \),故可通过计算 \( \mathbf{\Psi} \) 的特征值直接获得目标参数 \( \omega_k \) 的估计。

由此,问题转化为:在已知 \( \mathbf{V}_1 \) 与 \( \mathbf{V}_2 \) 的情况下,如何求取 \( \mathbf{\Psi} \)。一种直接的方法是采用最小二乘法(Least Squares, LS),构建如下优化问题:
\[
    \min_{\mathbf{\Psi}} \|\mathbf{V}_2 - \mathbf{V}_1 \mathbf{\Psi}\|_F^2,
\]
其解析解为
\[
    \hat{\mathbf{\Psi}} = (\mathbf{V}_1^{\mathrm{H}} \mathbf{V}_1)^{-1} \mathbf{V}_1^{\mathrm{H}} \mathbf{V}_2.
\]

然而,上述模型隐含了一个假设:即 \( \mathbf{V}_1 \) 为无噪自变量,而噪声仅存在于因变量 \( \mathbf{V}_2 \) 中。但在实际情形下,\( \mathbf{V}_1 \) 与 \( \mathbf{V}_2 \) 均会受到噪声污染,因此该假设并不严谨。为此,可以引入总最小二乘法(Total Least Squares, TLS),构建如下更合理的模型:
\[
    \begin{cases}
        \min\limits_{\mathbf{N}_1, \mathbf{N}_2}
        \left\|\begin{bmatrix}\mathbf{N}_1 & \mathbf{N}_2\end{bmatrix}\right\|_{\mathrm{F}}^2, \\[6pt]
        \text{s.t.} \quad \mathbf{V}_2 - \mathbf{N}_2 = (\mathbf{V}_1 - \mathbf{N}_1)\mathbf{\Psi}.
    \end{cases}
\]
在该模型中,\(\mathbf{N}_1 \in \mathbb{C}^{(M-1) \times K}\) 与 \(\mathbf{N}_2 \in \mathbb{C}^{(M-1) \times K}\) 分别表示 \( \mathbf{V}_1 \) 与 \( \mathbf{V}_2\) 的噪声项。TLS 的目标是寻找能量最小的噪声修正,使得两者重新满足旋转不变性条件,从而获得更为鲁棒的估计结果。

注意到约束条件可以写成如下的分块矩阵形式:
\[
    \left( \begin{bmatrix}
            \mathbf{V}_1 & \mathbf{V}_2
        \end{bmatrix} -
    \begin{bmatrix}
            \mathbf{N}_1 & \mathbf{N}_2
        \end{bmatrix} \right)
    \begin{bmatrix}
        \mathbf{\Psi} \\
        -\mathbf{I}
    \end{bmatrix} = \mathbf{0}.
\]
记\( \tilde{\mathbf{V}} = \begin{bmatrix} \mathbf{V}_1 & \mathbf{V}_2 \end{bmatrix} \in \mathbb{C}^{(M-1) \times 2K} \),\( \tilde{\mathbf{N}} = \begin{bmatrix} \mathbf{N}_1 & \mathbf{N}_2 \end{bmatrix} \in \mathbb{C}^{(M-1) \times 2K} \),上式表明,矩阵\( \tilde{\mathbf{V}} - \tilde{\mathbf{N}} \)不满秩,存在\( K \)个线性无关的向量\( \begin{bmatrix} \mathbf{\Psi}^{\mathrm{T}} & -\mathbf{I} \end{bmatrix}^{\mathrm{T}} \),对应的特征值为零。

此时,问题可转化为寻找一个 F-范数最小的修正矩阵 \( \tilde{\mathbf{N}} \),使得修正后的矩阵 \( \tilde{\mathbf{V}} - \tilde{\mathbf{N}} \) 的秩为 \( K \)(为方便起见,这里假设 \( M-1 \geq 2K \))。换言之,即求解矩阵 \( \tilde{\mathbf{V}} \) 的最佳低秩近似。

设 \( \tilde{\mathbf{V}} \) 的奇异值分解为
\[
    \tilde{\mathbf{V}}
    = \tilde{\mathbf{U}} \tilde{\mathbf{\Lambda}} \tilde{\mathbf{V}}^{\mathrm{H}}
    = \begin{bmatrix}
        \tilde{\mathbf{U}}_1 & \tilde{\mathbf{U}}_2
    \end{bmatrix}
    \begin{bmatrix}
        \tilde{\mathbf{\Lambda}}_1 & \mathbf{0}                 \\
        \mathbf{0}                 & \tilde{\mathbf{\Lambda}}_2
    \end{bmatrix}
    \begin{bmatrix}
        \tilde{\mathbf{V}}_1^{\mathrm{H}} \\
        \tilde{\mathbf{V}}_2^{\mathrm{H}}
    \end{bmatrix}
    = \tilde{\mathbf{U}}_1 \tilde{\mathbf{\Lambda}}_1 \tilde{\mathbf{V}}_1^{\mathrm{H}}
    + \tilde{\mathbf{U}}_2 \tilde{\mathbf{\Lambda}}_2 \tilde{\mathbf{V}}_2^{\mathrm{H}},
\]
其中,\( \tilde{\mathbf{\Lambda}}_1 \in \mathbb{R}^{K \times K} \) 为前 \( K \) 个最大奇异值构成的对角矩阵,\( \tilde{\mathbf{\Lambda}}_2 \in \mathbb{R}^{K \times K} \) 为后 \( K \) 个最小奇异值构成的对角矩阵。于是,矩阵
\[
    \tilde{\mathbf{V}} - \tilde{\mathbf{N}} = \tilde{\mathbf{U}}_1 \tilde{\mathbf{\Lambda}}_1 \tilde{\mathbf{V}}_1^{\mathrm{H}}
\]
即为 \( \tilde{\mathbf{V}} \) 的最佳秩-\( K \) 近似,而对应的噪声修正为
\[
    \tilde{\mathbf{N}} = \tilde{\mathbf{U}}_2 \tilde{\mathbf{\Lambda}}_2 \tilde{\mathbf{V}}_2^{\mathrm{H}}.
\]

由于 \(\tilde{\mathbf{V}}_1\) 与 \(\tilde{\mathbf{V}}_2\) 的列空间正交,因此有
\[
    \tilde{\mathbf{U}}_1 \tilde{\mathbf{\Lambda}}_1 \tilde{\mathbf{V}}_1^{\mathrm{H}} \tilde{\mathbf{V}}_2 = \mathbf{0}.
\]
这表明 \(\tilde{\mathbf{V}}_2 \in \mathbb{C}^{2K \times K}\) 的列空间等价于矩阵\(\begin{bmatrix} \mathbf{\Psi}^{\mathrm{T}} & -\mathbf{I} \end{bmatrix}^{\mathrm{T}}\)的张成空间。进一步地,将 \(\tilde{\mathbf{V}}_2\) 分块为
\[
    \tilde{\mathbf{V}}_2 =
    \begin{bmatrix}
        \tilde{\mathbf{V}}_{2,1} \\
        \tilde{\mathbf{V}}_{2,2}
    \end{bmatrix},
    \quad \tilde{\mathbf{V}}_{2,1}, \tilde{\mathbf{V}}_{2,2} \in \mathbb{C}^{K \times K},
\]
则有
\[
    -\tilde{\mathbf{V}}_2 \tilde{\mathbf{V}}_{2,2}^{-1}
    = \begin{bmatrix}
        -\tilde{\mathbf{V}}_{2,1} \tilde{\mathbf{V}}_{2,2}^{-1} \\
        -\mathbf{I}
    \end{bmatrix}
    = \begin{bmatrix}
        \mathbf{\Psi} \\
        -\mathbf{I}
    \end{bmatrix}.
\]
由此可得,总最小二乘估计结果为
\[
    \hat{\mathbf{\Psi}} = -\tilde{\mathbf{V}}_{2,1} \tilde{\mathbf{V}}_{2,2}^{-1}.
\]

\begin{example}
    对\cref{fig_noise_esprit}中的包含噪声的数据进行SVD分解,保留前\( K=2 \)个奇异值及其对应的奇异向量,并使用ESPRIT方法估计对应的目标参数。

    \begin{figure}[htb!]
        \centering
        \begin{tikzpicture}
            \begin{axis}[
                    xlabel={参数维},
                    ylabel={时间维},
                    enlargelimits=false,
                    width=5cm, height=5cm,
                    ytick=\empty,
                    xtick=\empty,
                    ticklabel style={font=\small},
                    label style={font=\small},
                    axis on top
                ]
                \addplot graphics [
                        xmin=-1, xmax=1, ymin=-1, ymax=1,
                    ] {./img/estimation/esprit_1.png};
            \end{axis}
        \end{tikzpicture}
        \caption{有噪声数据矩阵\( \mathbf{X} \)(仅绘制了实部)}
        \label{fig_noise_esprit}
    \end{figure}
\end{example}
\begin{solution}
    对数据矩阵进行SVD分解后,得到的奇异向量如\cref{fig_svd_result}所示。可以看到,右奇异值向量的前两个分量并不是复指数信号,而是两个复指数信号线性组合的结果。\cref{fig_svd_result_1}展示的左奇异向量同样也并不是纯净的单个目标的回波信号,而是两个目标回波信号的叠加。这进一步表明了,SVD分解得到的奇异值向量并不直接对应于目标的导向矢量,但两者张成的子空间是相同的。

    \begin{figure}[htb!]
        \centering
        \begin{subfigure}{.6\textwidth}
            \centering
            \begin{tikzpicture}
                \begin{axis}[
                        xlabel={时间维}, ylabel={实部},
                        ticklabel style={font=\small},
                        label style={font=\small},
                        grid, width=7cm, height=4cm,
                        xmin=0, xmax=1000,
                        legend cell align=left,
                        legend style={
                                anchor=north east,
                                font=\tiny,
                                draw=none,
                                fill=none
                            }
                    ]
                    \addplot[
                        c1,
                        thick,
                        smooth,
                    ] table[x=x, y=U1, col sep=comma] {./img/estimation/esprit_responses.csv};
                    \addplot[
                        c2,
                        thick,
                        smooth,
                    ] table[x=x, y=U2, col sep=comma] {./img/estimation/esprit_responses.csv};
                \end{axis}
            \end{tikzpicture}
            \caption{左奇异向量}
            \label{fig_svd_result_1}
        \end{subfigure}
        \begin{subfigure}{.37\textwidth}
            \centering
            \begin{tikzpicture}
                \begin{axis}[
                        xlabel={参数维},
                        ticklabel style={font=\small},
                        label style={font=\small},
                        grid,width=5cm, height=4cm,
                        xmin=0, xmax=39,
                        legend cell align=left,
                        legend style={
                                anchor=north east,
                                font=\tiny,
                                draw=none,
                                fill=none
                            }
                    ]
                    \addplot[
                        c1,
                        thick,
                    ] table[x=x, y=V1, col sep=comma] {./img/estimation/esprit_modes.csv};
                    \addplot[
                        c2,
                        thick,
                    ] table[x=x, y=V2, col sep=comma] {./img/estimation/esprit_modes.csv};
                \end{axis}
            \end{tikzpicture}
            \caption{右奇异向量}
            \label{fig_svd_result_2}
        \end{subfigure}
        \caption{SVD分解得到的奇异向量}
        \label{fig_svd_result}
    \end{figure}

    利用ESPRIT方法估计出矩阵\( \mathbf{\hat{\Psi}} \)后,计算其特征分解\( \mathbf{\hat{\Psi}} = \mathbf{T}^{-1} \mathbf{\hat{\Phi}} \mathbf{T}\),并通过特征值反解出目标参数为\( \hat{\omega}_1 = 0.0999785 \),\( \hat{\omega}_2 = -0.05001828 \),与真实值\( \omega_1 = 0.1 \),\( \omega_2 = -0.05 \)非常接近。将估计结果在归一化功率谱上标出,如\cref{fig_esprit_result}所示。可以看到,ESPRIT方法成功地分辨出了两个目标参数,并且其谱峰位置与CBF的峰值位置基本一致。

    \begin{figure}[htb!]
        \centering
        \begin{tikzpicture}
            \begin{axis}[
                    xlabel={参数维}, ylabel={归一化功率谱},
                    ticklabel style={font=\small},
                    xmin=-0.5, xmax=0.5, ymin=0, ymax=1.2,
                    % xtick=\empty,ytick=\empty,
                    width=8cm, height=4cm,
                    label style={font=\small},
                    grid,
                    legend cell align=left,
                    legend style={
                            anchor=north east,
                            font=\tiny,
                            draw=none,
                            fill=none
                        }
                ]
                \addplot[
                    c1,
                    thick,
                ] table[x=x, y=y, col sep=comma] {./img/estimation/cbf_spectrum_ref.csv};
                \addlegendentry{CBF}
                \addplot[
                    black,
                    thick,
                    dashed
                ] coordinates {(0.0999785,0) (0.0999785,1.2)};
                \addlegendentry{ESPRIT}
                \addplot[
                    black,
                    thick,
                    dashed
                ] coordinates {(-0.05001828,0) (-0.05001828,1.2)};
            \end{axis}
        \end{tikzpicture}
        \caption{ESPRIT估计结果}
        \label{fig_esprit_result}
    \end{figure}

    此外,利用变换矩阵\( \mathbf{T} \)可以将奇异向量转换为目标回波以及导向矢量,具体形式为
    \[
        \hat{\mathbf{A}} = \mathbf{V}_s \mathbf{T}^{-1}, \quad
        \hat{\mathbf{R}} = \mathbf{U}_s \mathbf{\Lambda}_s \mathbf{T}^{\mathrm{H}}.
    \]
    计算得到的结果如\cref{fig_svd_result_refined}所示。可以看到,经过变换后的右奇异向量已基本恢复为纯净的复指数信号,而左奇异向量也更接近于单个目标的回波信号。

    \begin{figure}[htb!]
        \centering
        \begin{subfigure}{.6\textwidth}
            \centering
            \begin{tikzpicture}
                \begin{axis}[
                        xlabel={时间维}, ylabel={实部},
                        ticklabel style={font=\small},
                        label style={font=\small},
                        grid, width=7cm, height=4cm,
                        xmin=0, xmax=1000,
                        legend cell align=left,
                        legend style={
                                anchor=north east,
                                font=\tiny,
                                draw=none,
                                fill=none
                            }
                    ]
                    \addplot[
                        c1,
                        thick,
                        smooth,
                    ] table[x=x, y=R1, col sep=comma] {./img/estimation/esprit_responses_refined.csv};
                    \addplot[
                        c2,
                        thick,
                        smooth,
                    ] table[x=x, y=R2, col sep=comma] {./img/estimation/esprit_responses_refined.csv};
                \end{axis}
            \end{tikzpicture}
            \caption{左奇异向量}
            \label{fig_svd_result_refined_1}
        \end{subfigure}
        \begin{subfigure}{.37\textwidth}
            \centering
            \begin{tikzpicture}
                \begin{axis}[
                        xlabel={参数维},
                        ticklabel style={font=\small},
                        label style={font=\small},
                        grid,width=5cm, height=4cm,
                        xmin=0, xmax=39,
                        legend cell align=left,
                        legend style={
                                anchor=north east,
                                font=\tiny,
                                draw=none,
                                fill=none
                            }
                    ]
                    \addplot[
                        c1,
                        thick,
                    ] table[x=x, y=A1, col sep=comma] {./img/estimation/esprit_modes_refined.csv};
                    \addplot[
                        c2,
                        thick,
                    ] table[x=x, y=A2, col sep=comma] {./img/estimation/esprit_modes_refined.csv};
                \end{axis}
            \end{tikzpicture}
            \caption{右奇异向量}
            \label{fig_svd_result_refined_2}
        \end{subfigure}
        \caption{修正后的奇异向量}
        \label{fig_svd_result_refined}
    \end{figure}
\end{solution}

\section{稀疏表示方法}

给定接收信号矩阵\( \mathbf{X} = \mathbf{R} \mathbf{A}^{\mathrm{T}}  + \mathbf{N} \in \mathbb{C}^{L \times M} \),其中导向矢量矩阵\( \mathbf{A} \in \mathbb{C}^{M \times K} \)中的列向量的形式已知,为
\[
    \bm{a}_{\omega_k}
    = \begin{bmatrix}
        1 & e^{j2\pi\omega_k} & \cdots & e^{j2\pi(M-1)\omega_k}
    \end{bmatrix}^{\mathrm{T}}, \quad k=1,2,\cdots,K.
\]
因此,不妨构建一个超完备的参数空间的离散字典
\[
    \mathbf{D} = \begin{bmatrix}
        \bm{a}_{\omega_1} & \bm{a}_{\omega_2} & \cdots & \bm{a}_{\omega_I}
    \end{bmatrix} \in \mathbb{C}^{M \times I},
\]
其中\( I \gg K \),且参数点\( \omega_i \)在参数空间内均匀分布。可以预见的是,只要\( I \)足够大,实际目标参数\( \omega_k \)必然包含在字典\( \mathbf{D} \)中或者与字典中的某个参数点非常接近。因此,利用字典\( \mathbf{D} \)可以将接收信号矩阵分解为
\[
    \mathbf{X} = \mathbf{S} \mathbf{D}^{\mathrm{T}} + \mathbf{N},
\]
而其中\( \mathbf{S} \in \mathbb{C}^{L \times I} \)为稀疏系数矩阵,其列向量\( \bm{s}_i \)表示参数点\( \omega_i \)对应的目标回波信号。显然,只有当\( \omega_i \)接近某个实际目标参数时,\( \bm{s}_i \)才可能非零,否则应为零。因此,系数矩阵\( \mathbf{S} \)的列向量是稀疏的。记向量\( \bm{p} \in \mathbb{C}^{I \times 1}\)为系数矩阵\( \mathbf{S} \)的列范数构成的向量,即
\[
    p_i = \|\bm{s}_i\|^2_2, \quad i=1,2,\cdots,I,
\]
则向量\( \bm{p} \)是稀疏的,且其非零分量对应的索引即为目标参数在字典中的位置。利用稀疏表示的思想,可以将参数估计问题转化为如下的稀疏恢复问题:
\[
    \begin{cases}
        \min\limits_{\mathbf{S}} \quad & \|\bm{p}\|_0,                                                                   \\
        \text{s.t.} \quad              & \|\mathbf{X} - \mathbf{S} \mathbf{D}^{\mathrm{T}}\|_{\mathrm{F}} \leq \epsilon, \\
                                       & \| \bm{s}_i \|_2 = p_i, \quad i=1,2,\cdots,I,                                   \\
    \end{cases}
\]
该稀疏恢复问题可以转换为一个二阶锥规划问题,并利用现有的凸优化工具进行求解。

首先,将零范数放宽为一范数,同时注意到\( \bm{p} \)中的分量均非负,因此有
\[
    \|\bm{p}\|_1 = \sum_{i=1}^I p_i = \bm{1}^{\mathrm{T}} \bm{p} = \sum_{i=1}^I \|\bm{s}_i\|_2.
\]
与此同时,引入辅助变量\( \varepsilon \)并构造如下的优化问题:
\[
    \begin{cases}
        \min \quad        & \epsilon + \lambda \varepsilon,                                                 \\
        \text{s.t.} \quad & \|\mathbf{X} - \mathbf{S} \mathbf{D}^{\mathrm{T}}\|_{\mathrm{F}} \leq \epsilon, \\
                          & \sum_{i=1}^I \|\bm{s}_i\|_2 \leq \varepsilon,                                   \\
    \end{cases}
\]
其中\( \lambda > 0 \)为权衡参数,当\( \lambda \)较大时,更加注重系数向量的稀疏性,而当\( \lambda \)较小时,更加注重数据拟合的精度。

\begin{example}
    利用上述稀疏表示方法对\cref{fig_noise_esprit}所示的包含噪声的数据进行进行参数估计。
\end{example}
\begin{solution}
    首先构建参数字典,取参数点\( \omega_i \)在区间\( [-0.5, 0.5] \)上均匀分布,共取\( I=200 \)个点。然后利用CVX工具箱求解上述的二阶锥规划问题,取权衡参数为\( \lambda = 3 \)。计算得到的矩阵\( \mathbf{S} \)以及对应的功率谱向量\( \bm{p} \)如\cref{fig_sparse_result}所示。

    \begin{figure}[htb!]
        \centering
        \begin{subfigure}{.7\textwidth}
            \centering
            \begin{tikzpicture}
                \begin{axis}[
                        xlabel={参数维},
                        ylabel={时间维},
                        enlargelimits=false,
                        width=8cm, height=4cm,
                        ytick=\empty,
                        xtick=\empty,
                        ticklabel style={font=\small},
                        label style={font=\small},
                        axis on top
                    ]
                    \addplot graphics [
                            xmin=-1, xmax=1, ymin=-1, ymax=1,
                        ] {./img/estimation/l1svd_2.png};
                \end{axis}
            \end{tikzpicture}
            \caption{矩阵\( \mathbf{S} \)(仅绘制了实部)}
            \label{fig_sparse_result_1}
        \end{subfigure}
        \begin{subfigure}{.7\textwidth}
            \centering
            \begin{tikzpicture}
                \begin{axis}[
                        xlabel={参数维}, ylabel={归一化功率谱(dB)},
                        ticklabel style={font=\small},
                        xmin=-0.5, xmax=0.5,
                        ymax=50,
                        % xtick=\empty,ytick=\empty,
                        width=8cm, height=4cm,
                        label style={font=\small},
                        grid,
                        legend cell align=left,
                        legend style={
                                anchor=north east,
                                font=\tiny,
                                draw=none,
                                fill=none
                            }
                    ]
                    \addplot[
                        c1,
                        thick,
                        smooth,
                    ] table[x=x, y=p1, col sep=comma] {./img/estimation/l1svd.csv};
                    \addlegendentry{MUSIC}
                    \addplot[
                        c2,
                        thick,
                        smooth,
                    ] table[x=x, y=p2, col sep=comma] {./img/estimation/l1svd.csv};
                    \addlegendentry{Sparse}
                \end{axis}
            \end{tikzpicture}
            \caption{归一化功率谱(dB)}
            \label{fig_sparse_result_2}
        \end{subfigure}
        \caption{稀疏表示方法的估计结果}
        \label{fig_sparse_result}
    \end{figure}

    从\cref{fig_sparse_result_1}可以看到,系数矩阵\( \mathbf{S} \)的列向量非常稀疏,只有两个分量显著非零,其余均接近于零。并且,非零的两个列向量就对应两个目标的回波信号。对应的功率谱如\cref{fig_sparse_result_2}所示,可以看到,稀疏表示方法成功地分辨出了两个目标参数,并且其谱峰位置与MUSIC方法的峰值位置基本一致,并且谱峰更加尖锐。
\end{solution}

需要注意的是,稀疏表示方法中权衡参数 \( \lambda \) 的选取对结果具有显著影响,通常需要通过多次实验进行调优。尽管稀疏表示方法在估计精度上往往优于传统方法,但其计算复杂度极高,运算耗时可能是 MUSIC 方法的数百甚至数千倍。为降低计算负担,可以对接收数据矩阵先行进行奇异值分解(SVD)降维,从而在一定程度上牺牲精度以换取复杂度的下降。设接收数据矩阵为\( \mathbf{X} \in \mathbb{C}^{L \times M} \),其SVD分解为
\[
    \mathbf{X} = \mathbf{U} \mathbf{\Lambda} \mathbf{V}^{\mathrm{H}}
    = \begin{bmatrix}
        \mathbf{U}_s & \mathbf{U}_n
    \end{bmatrix}
    \begin{bmatrix}
        \mathbf{\Lambda}_s & \mathbf{0}         \\
        \mathbf{0}         & \mathbf{\Lambda}_n
    \end{bmatrix}
    \begin{bmatrix}
        \mathbf{V}_s^{\mathrm{H}} \\
        \mathbf{V}_n^{\mathrm{H}}
    \end{bmatrix},
\]
其中,\( \mathbf{U}_s \in \mathbb{C}^{L \times K} \),\( \mathbf{\Lambda}_s \in \mathbb{R}^{K \times K} \),\( \mathbf{V}_s \in \mathbb{C}^{M \times K} \) 分别对应于前 \( K \) 个奇异值及其奇异向量。于是,可以构造降维后的数据矩阵:
\[
    \tilde{\mathbf{X}} = \mathbf{U}_s^{\mathrm{H}} \mathbf{X}
    = \mathbf{\Lambda}_s \mathbf{V}_s^{\mathrm{H}} \in \mathbb{C}^{K \times M}.
\]
一般而言,\( K \) 为目标个数,其数值远小于采样点数 \( L \),因此降维后的数据矩阵 \(\tilde{\mathbf{X}}\) 的行数显著减少,从而有效降低了稀疏恢复问题的计算复杂度。但与此同时,分解得到的稀疏系数矩阵 \(\mathbf{S}\) 将不再直接对应于原始目标回波信号。

\section{多域联合估计}
随着雷达探测技术的不断发展,现代雷达系统往往能够在多个维度上同时采集数据,例如方位、俯仰、多普勒等。这些多维信息天然适合组织为高阶张量形式。若仅截取其中某一维度的切片并采用传统参数估计方法进行处理,则不可避免地忽略其余维度所蕴含的结构特征,从而导致估计性能下降。为克服这一局限,近年来逐渐兴起了多域联合估计的方法,其核心思想是充分融合不同维度的信息,以实现更高精度和更强稳健性的目标参数估计。

\subsection{空时自适应处理}
考虑如下的接收信号张量模型:
\[
    \mathcal{X} = \mathcal{I}_{K} \times_1 \mathbf{R} \times_2 \mathbf{A} \times_3 \mathbf{B},
\]
其中\( \mathcal{I}_{K} \in \mathbb{C}^{K \times K \times K} \)为一个单位张量,\( \mathbf{R} \in \mathbb{C}^{L \times K} \)为目标回波信号矩阵,\( \mathbf{A} \in \mathbb{C}^{M \times K} \)为第一个参数的导向矩阵,\( \mathbf{B} \in \mathbb{C}^{N \times K} \)为第二个参数的导向矩阵。并且,两个参数对应的导向矢量有如下形式
\[
    \begin{split}
        \bm{a}_{\nu_k} = \begin{bmatrix}
                             1 & e^{j2\pi\nu_k} & \cdots & e^{j2\pi(M-1)\nu_k}
                         \end{bmatrix}^{\mathrm{T}}, \\
        \bm{b}_{\mu_k} = \begin{bmatrix}
                             1 & e^{j2\pi\mu_k} & \cdots & e^{j2\pi(N-1)\mu_k}
                         \end{bmatrix}^{\mathrm{T}}.
    \end{split}
\]

在经典波束形成(CBF)方法中,若同时对接收信号矩阵 \( \mathbf{X} \) 的行向量与列向量分别实施波束形成与脉冲压缩,则其输出可统一表示为
\[
    \mathbf{Z} = \mathbf{S}^{\mathrm{H}} \mathbf{X} \mathbf{W},
\]
其中\( \mathbf{S}  \)为匹配滤波矩阵,\( \mathbf{W} \)为``波束形成''矩阵。这一形式自然引出一个问题:是否存在 CBF 的推广,使其能够同时作用于接收信号张量 \( \mathcal{X} \) 的多个维度,实现所谓的“多维脉冲压缩”,并使目标在结果中表现为清晰的峰值。

答案是肯定的。具体而言,除了匹配滤波矩阵\( \mathbf{S} \)之外,还需要分别构建如下两个``波束形成''矩阵:
\[
    \mathbf{V} = \begin{bmatrix}
        \bm{v}_{\nu_1} & \bm{v}_{\nu_2} & \cdots & \bm{v}_{\nu_P}
    \end{bmatrix} \in \mathbb{C}^{M \times P}, \quad
    \mathbf{U} = \begin{bmatrix}
        \bm{u}_{\mu_1} & \bm{u}_{\mu_2} & \cdots & \bm{u}_{\mu_Q}
    \end{bmatrix} \in \mathbb{C}^{N \times Q},
\]
其中\( \nu_p \) 与\( \mu_q \)分别为第一个与第二个参数的扫描点,对应的列向量分别为
\[
    \begin{split}
        \bm{v}_{\nu_p} = \frac{1}{M} \begin{bmatrix}
                                         1 & e^{j2\pi\nu_p} & \cdots & e^{j2\pi(M-1)\nu_p}
                                     \end{bmatrix}^{\mathrm{H}}, \\
        \bm{u}_{\mu_q} = \frac{1}{N} \begin{bmatrix}
                                         1 & e^{j2\pi\mu_q} & \cdots & e^{j2\pi(N-1)\mu_q}
                                     \end{bmatrix}^{\mathrm{H}}.
    \end{split}
\]
于是,可以定义多维脉冲压缩的输出张量为
\[
    \mathcal{Z} = \mathcal{X} \times_1 \mathbf{S}^{\mathrm{H}} \times_2 \mathbf{V}^{\mathrm{T}} \times_3 \mathbf{U}^{\mathrm{T}} \in \mathbb{C}^{L \times P \times Q}.
\]

不难发现,就形式上而言,张量 \( \mathcal{Z} \) 实际上是二维脉压矩阵 \( \mathbf{Z} \) 的升维扩展。目标在该三维张量中依然表现为一个峰值,其三维坐标分别对应目标的距离、参数一与参数二。以多普勒线性阵列雷达系统为例,通过该方法即可在同一框架下实现对目标距离、方位角和速度的联合检测。下面,让我们对张量\( \mathcal{Z} \)进行进一步地分析,以证明其确实能够实现多维脉冲压缩的功能。

根据\cref{prop:associative},有
\[
\begin{split}
    \mathcal{Z} 
    & = \mathcal{X} \times_1 \mathbf{S}^{\mathrm{H}} \times_2 \mathbf{V}^{\mathrm{T}} \times_3 \mathbf{U}^{\mathrm{T}} 
      = \left( \mathcal{I}_K \times_1 \mathbf{R} \times_2 \mathbf{A} \times_3 \mathbf{B} \right) 
        \times_1 \mathbf{S}^{\mathrm{H}} \times_2 \mathbf{V}^{\mathrm{T}} \times_3 \mathbf{U}^{\mathrm{T}} \\
    & = \mathcal{I}_K \times_1 \left( \mathbf{S}^{\mathrm{H}} \mathbf{R} \right) 
                   \times_2 \left( \mathbf{V}^{\mathrm{T}} \mathbf{A} \right) 
                   \times_3 \left( \mathbf{U}^{\mathrm{T}} \mathbf{B} \right).
\end{split}
\]
注意到,矩阵 \( \tilde{\mathbf{R}} = \mathbf{S}^{\mathrm{H}} \mathbf{R} \in \mathbb{C}^{L \times K} \) 对应于对目标回波信号在距离维上的脉冲压缩。记其列向量为
\[
    \tilde{\bm{r}}_k = \mathbf{S}^{\mathrm{H}} \bm{r}_k, 
    \quad k=1,2,\dots,K,
\]
根据脉冲压缩原理,\( \tilde{\bm{r}}_k \) 近似为一个稀疏向量,仅在目标距离处存在显著峰值,其余位置接近零:
\[
    \tilde{\bm{r}}_k[i] \approx 
    \begin{cases}
        A_k, & i = d_k, \\
        0,   & i \neq d_k,
    \end{cases}
    \quad k=1,2,\dots,K,
\]
其中 \( A_k \) 表示目标的回波幅度与相位信息,\( d_k \) 为目标的距离索引。  

而矩阵 \( \tilde{\mathbf{A}} =  \mathbf{V}^{\mathrm{T}} \mathbf{A} \in \mathbb{C}^{P \times K} \) 对应于对第一个参数维的波束形成。记其列向量为
\[
    \tilde{\bm{a}}_k = \mathbf{V}^{\mathrm{T}} \bm{a}_{\nu_k}, 
    \quad k=1,2,\dots,K,
\]
由于不同频率的复指数向量之间近似正交,因此 \( \tilde{\bm{a}}_k \) 同样近似为一个稀疏向量,仅在目标参数一处出现显著峰值:
\[
    \tilde{\bm{a}}_k[p] \approx 
    \begin{cases}
        1, & p = \nu_k, \\
        0, & p \neq \nu_k,
    \end{cases}
    \quad k=1,2,\dots,K,
\]
其中 \( \nu_k \) 为目标参数一的取值。类似地,矩阵 \(  \tilde{\mathbf{B}} = \mathbf{U}^{\mathrm{H}} \mathbf{B} \in \mathbb{C}^{Q \times K} \) 对应于对第二个参数维的波束形成。记其列向量为
\[
    \tilde{\bm{b}}_k = \mathbf{U}^{\mathrm{T}} \bm{b}_{\mu_k}, 
    \quad k=1,2,\dots,K,
\]
则 \( \tilde{\bm{b}}_k \) 亦近似为一个稀疏向量,仅在目标参数二处存在显著峰值:
\[
    \tilde{\bm{b}}_k[q] \approx 
    \begin{cases}
        1, & q = \mu_k, \\
        0, & q \neq \mu_k,
    \end{cases}
    \quad k=1,2,\dots,K,
\]
其中 \( \mu_k \) 为目标参数二的取值。

利用如上近似以及\cref{prop:diag-kron},可以进一步地将张量 \( \mathcal{Z} \) 表示为
\[
\begin{split}
    \mathcal{Z} 
    & \approx \mathcal{I}_K \times_1 \tilde{\mathbf{R}} \times_2 \tilde{\mathbf{A}} \times_3 \tilde{\mathbf{B}} \\
    & = \sum_{k=1}^K \tilde{\bm{r}}_k \circ \tilde{\bm{a}}_k \circ \tilde{\bm{b}}_k,
\end{split}
\]
记\( \mathcal{Z}_k = \tilde{\bm{r}}_k \circ \tilde{\bm{a}}_k \circ \tilde{\bm{b}}_k \) ,则张量 \( \mathcal{Z} \) 可视为 \( K \) 个秩一张量的叠加。并且,每个秩一张量 \( \mathcal{Z}_k \) 中的元素有如下表示:
\[
    (\mathcal{Z}_k)_{ipq} =  \tilde{\bm{r}}_k[i] \cdot \tilde{\bm{a}}_k[p] \cdot \tilde{\bm{b}}_k[q] = \begin{cases}
        A_k, & i = d_k, p = \nu_k, q = \mu_k \\
        0,   & \text{其他}
    \end{cases},
\]
这表明每个秩一张量 \( \mathcal{Z}_k \) 仅在位置 \( (d_k, \nu_k, \mu_k) \) 处存在一个峰值,其余位置均为零。由此可见,张量 \( \mathcal{Z} \) 中的每个目标均对应于一个唯一的三维峰值,其坐标分别对应于目标的距离、参数一与参数二。类似地,计算张量 \( \mathcal{Z} \) 沿着第一个维度的功率,即
\[
    p_{pq} = \frac{1}{L} \sum_{i=1}^L |\mathcal{Z}_{ipq}|^2,
\]
则二维功率谱 \( \mathbf{P} \in \mathbb{R}^{P \times Q} \) 中同样会在每个目标的参数位置 \( (\nu_k, \mu_k) \) 处出现一个峰值。

然而,CBF 方法未考虑噪声与干扰的影响,其分辨率与估计精度均受到限制。前述多维脉冲压缩作为 CBF 的推广,同样存在这一问题。为进一步提升估计性能,可以将 Capon 算法推广至张量情形。由于一般处理的数据往往包含方位维和多普勒维的信息,因此该方法又被称为空时自适应处理(Space-Time Adaptive Processing, STAP)。

具体而言,STAP 方法对Capon算法进行拓展,构建了如下的优化模型
\[
    \begin{cases}
        \min \quad        & \frac{1}{L} \left\| \mathcal{X} \times_2 \bm{v}_{\nu}^{\mathrm{T}} \times_3 \bm{u}_{\mu}^{\mathrm{T}} \right\|^2                          \\
        \text{s.t.} \quad & \bm{a}^{\mathrm{T}}_{\nu} \bm{v}_{\nu} = 1 \\
                            & \bm{b}^{\mathrm{T}}_{\mu} \bm{u}_{\mu} = 1
    \end{cases}.
\]
对于向量\( \mathcal{X} \times_2 \bm{v}_{\nu}^{\mathrm{T}} \times_3 \bm{b}_{\mu}^{\mathrm{T}} \in \mathbb{C}^{L \times 1} \),其可以看作是分别利用两个参数维度的加权向量\( \bm{v}_{\nu} \)与\( \bm{u}_{\mu} \)对张量\( \mathcal{X} \)进行滤波后得到的滤波器输出。与Capon算法类似,该优化问题的目标函数同样为最小化滤波器输出的功率。与此同时,约束条件则确保对参数为 \( \nu \) 和 \( \mu \) 目标信号的响应不受损失。

利用\cref{prop:kronecker-kron},可以将滤波器的输出功率写成如下的形式:
\[
    \begin{split}
        \frac{1}{L} \left\| \mathcal{X} \times_2 \bm{v}_{\nu}^{\mathrm{T}} \times_3 \bm{u}_{\mu}^{\mathrm{T}} \right\|^2
        & = \frac{1}{L} \left\| (\bm{v}_{\nu} \otimes \bm{u}_{\mu})^{\mathrm{T}} \mathbf{X}_{23}  \right\|^2 \\
        & = \frac{1}{L}  ((\bm{v}_{\nu} \otimes \bm{u}_{\mu})^{\mathrm{T}} \mathbf{X}_{23}) \left((\bm{v}_{\nu} \otimes \bm{u}_{\mu})^{\mathrm{T}} \mathbf{X}_{23}\right)^{\mathrm{H}} \\
        & = (\bm{v}_{\nu} \otimes \bm{u}_{\mu})^{\mathrm{T}} \mathbf{\Sigma}_{23} (\overline{\bm{v}}_{\nu} \otimes \overline{\bm{u}}_{\mu}),
    \end{split}
\]
其中,\( \mathbf{X}_{23} \in \mathbb{C}^{MN \times L} \)为张量\( \mathcal{X} \)在第2与第3维的展开矩阵,\( \mathbf{\Sigma}_{23} = \frac{1}{L} \mathbf{X}_{23} \mathbf{X}_{23}^{\mathrm{H}} \in \mathbb{C}^{MN \times MN} \)为其协方差矩阵。

在 STAP 算法中,可以通过放宽约束条件,将原有的两个约束合并为一个,从而直接求解两个滤波器的克罗内克积 \( \bm{v}_{\nu} \otimes \bm{u}_{\mu} \)。
此时,优化模型可改写为
\begin{equation}
    \begin{cases}
        \min \quad        & \bm{w}_{\nu\mu}^{\mathrm{T}} \mathbf{\Sigma}_{23} \bm{w}_{\nu\mu} \\
        \text{s.t.} \quad & (\bm{a}_{\nu} \otimes \bm{b}_{\mu})^{\mathrm{T}} \bm{w}_{\nu\mu} = 1 ,
    \end{cases}
\end{equation}
其中,\( \bm{w}_{\nu\mu} \in \mathbb{C}^{MN \times 1} \) 表示联合滤波器。
该优化问题在形式上与传统 Capon 算法完全一致,因此可以直接得到解析解:
\begin{equation}
    \bm{w}_{\nu\mu}
    = \frac{\mathbf{\Sigma}_{23}^{-1} (\bm{a}_{\nu} \otimes \bm{b}_{\mu})}
    {(\bm{a}_{\nu} \otimes \bm{b}_{\mu})^{\mathrm{T}} \mathbf{\Sigma}_{23}^{-1} (\bm{a}_{\nu} \otimes \bm{b}_{\mu})}.
\end{equation}
类似于Capon方法,STAP方法得到的二维功率谱有如下表示:
\[
    p_{\nu\mu} =\frac{1}{(\bm{a}_{\nu} \otimes \bm{b}_{\mu})^{\mathrm{T}} \mathbf{\Sigma}_{23}^{-1} (\bm{a}_{\nu} \otimes \bm{b}_{\mu})}.
\]
然而,该方法并不能确保联合滤波器 \( \bm{w}_{\nu\mu} \) 能够分解为两个独立滤波器的克罗内克积形式,因此算法性能受到一定影响。为此,可以直接求解原始的优化模型,从而分别得到两个滤波器 \( \bm{v}_{\nu} \) 与 \( \bm{u}_{\mu} \)。

尽管无法直接得到该优化问题的解析解,但可以通过交替迭代的方法进行求解。同样,利用拉格朗日乘子法,可以构建如下的拉格朗日函数:
\[
    \mathcal{L}(\bm{v}_{\nu}, \bm{u}_{\mu}, \lambda_1, \lambda_2)
    = \frac{1}{2L} \left\| \mathcal{X} \times_2 \bm{v}_{\nu}^{\mathrm{T}} \times_3 \bm{u}_{\mu}^{\mathrm{T}} \right\|^2
      - \lambda_1 \left( \bm{a}^{\mathrm{T}}_{\nu} \bm{v}_{\nu} - 1 \right)
      - \lambda_2 \left( \bm{b}^{\mathrm{T}}_{\mu} \bm{u}_{\mu} - 1 \right).
\]
如将\( \mathbf{\Sigma}_{23} \)转换为一个四阶张量\( \mathcal{S} \in \mathbb{C}^{M \times N \times M \times N} \),则滤波器的输出功率可以进一步表示为
\[
    \frac{1}{L} \left\| \mathcal{X} \times_2 \bm{v}_{\nu}^{\mathrm{T}} \times_3 \bm{u}_{\mu}^{\mathrm{T}} \right\|^2 = \mathcal{S} \times_1 \bm{v}_{\nu}^{\mathrm{T}} \times_2 \bm{u}_{\mu}^{\mathrm{T}} \times_3 \bm{v}_{\nu}^{\mathrm{H}} \times_4 \bm{u}_{\mu}^{\mathrm{H}}.
\]
因此,拉格朗日函数可以写成
\[
    \mathcal{L}(\bm{v}_{\nu}, \bm{u}_{\mu}, \lambda_1, \lambda_2)
    = \frac{1}{2}\mathcal{S} \times_1 \bm{v}_{\nu}^{\mathrm{T}} \times_2 \bm{u}_{\mu}^{\mathrm{T}} \times_3 \bm{v}_{\nu}^{\mathrm{H}} \times_4 \bm{u}_{\mu}^{\mathrm{H}}
      - \lambda_1 \left( \bm{a}^{\mathrm{T}}_{\nu} \bm{v}_{\nu} - 1 \right)
      - \lambda_2 \left( \bm{b}^{\mathrm{T}}_{\mu} \bm{u}_{\mu} - 1 \right).
\]
方便起见,将\(\mathcal{S} \times_2 \bm{u}_{\mu}^{\mathrm{T}} \times_4 \bm{u}_{\mu}^{\mathrm{H}}\) 重塑为一个矩阵\( \mathbf{S}_{1} \in \mathbb{C}^{M \times M} \),则\(  \mathcal{S} \times_1 \bm{v}_{\nu}^{\mathrm{T}} \times_2 \bm{u}_{\mu}^{\mathrm{T}} \times_3 \bm{v}_{\nu}^{\mathrm{H}} \times_4 \bm{u}_{\mu}^{\mathrm{H}} \)可以表示为\(\bm{v}_{\nu}^{\mathrm{H}} \mathbf{S}_1 \bm{v}_{\nu}\)。类似地,将\(\mathcal{S} \times_1 \bm{v}_{\nu}^{\mathrm{T}} \times_3 \bm{v}_{\nu}^{\mathrm{H}}\)重塑为一个矩阵\( \mathbf{S}_{2} \in \mathbb{C}^{N \times N} \),则\(  \mathcal{S} \times_1 \bm{v}_{\nu}^{\mathrm{T}} \times_2 \bm{u}_{\mu}^{\mathrm{T}} \times_3 \bm{v}_{\nu}^{\mathrm{H}} \times_4 \bm{u}_{\mu}^{\mathrm{H}} \)可以表示为\(\bm{u}_{\mu}^{\mathrm{H}} \mathbf{S}_2 \bm{u}_{\mu}\)。
接下来,分别对 \( \bm{v}_{\nu} \) 与 \( \bm{u}_{\mu} \) 求偏导数,并令其为零,可得
\[
    \begin{cases}
        \frac{\partial \mathcal{L}}{\partial \bm{v}_{\nu}} 
        & = \mathbf{S}_1 \bm{v}_{\nu} - \lambda_1 \overline{\bm{a}}_{\nu} = 0, \\
        \frac{\partial \mathcal{L}}{\partial \bm{u}_{\mu}} 
        & = \mathbf{S}_2 \bm{u}_{\mu} - \lambda_2 \overline{\bm{b}}_{\mu} = 0.
    \end{cases}
\]
将上式联立约束条件,可得
\[
    \begin{cases}
        \bm{v}_{\nu} & = \frac{\mathbf{S}_1^{-1} \overline{\bm{a}}_{\nu}}{\bm{a}_{\nu}^{\mathrm{T}} \mathbf{S}_1^{-1} \overline{\bm{a}}_{\nu}}, \\
        \bm{u}_{\mu} & = \frac{\mathbf{S}_2^{-1} \overline{\bm{b}}_{\mu}}{\bm{b}_{\mu}^{\mathrm{T}} \mathbf{S}_2^{-1} \overline{\bm{b}}_{\mu}}.
    \end{cases}
\]
由于 \( \mathbf{S}_1 \) 与 \( \mathbf{S}_2 \) 均依赖于 \( \bm{u}_{\mu} \) 与 \( \bm{v}_{\nu} \),因此上述方程组无法直接求解。为此,可以采用交替迭代的方法,交替更新 \( \bm{v}_{\nu} \) 与 \( \bm{u}_{\mu} \),直至收敛为止。初始时,可以将 \( \bm{u}_{\mu} \) 与 \( \bm{v}_{\nu} \) 分别设为随机向量。

三种方法的二维功率谱比较如\cref{fig_stap_compare}所示。可以看到,相较于多维CBF方法,STAP方法的谱峰更加尖锐。而交替迭代STAP方法由于求解的滤波器满足更严格的约束条件,因此其谱峰更加集中,估计精度更高。

\begin{figure}[htb!]
    \centering
    \begin{subfigure}{.32\textwidth}
        \centering
            \begin{tikzpicture}
                \begin{axis}[
                        xlabel={参数一},
                        ylabel={参数二},
                        enlargelimits=false,
                        width=4.5cm, height=4.5cm,
                        ytick=\empty,
                        xtick=\empty,
                        ticklabel style={font=\small},
                        label style={font=\small},
                        axis on top
                    ]
                    \addplot graphics [
                            xmin=-1, xmax=1, ymin=-1, ymax=1,
                        ] {./img/estimation/stap1.png};
                \end{axis}
            \end{tikzpicture}
        \caption{多维CBF}
        \label{fig_stap_compare_1}
    \end{subfigure}
    \begin{subfigure}{.32\textwidth}
        \centering
                    \begin{tikzpicture}
                \begin{axis}[
                        xlabel={参数一},
                        ylabel={参数二},
                        enlargelimits=false,
                        width=4.5cm, height=4.5cm,
                        ytick=\empty,
                        xtick=\empty,
                        ticklabel style={font=\small},
                        label style={font=\small},
                        axis on top
                    ]
                    \addplot graphics [
                            xmin=-1, xmax=1, ymin=-1, ymax=1,
                        ] {./img/estimation/stap2.png};
                \end{axis}
            \end{tikzpicture}
        \caption{STAP}
        \label{fig_stap_compare_2}
    \end{subfigure}
    \begin{subfigure}{.32\textwidth}
        \centering
                    \begin{tikzpicture}
                \begin{axis}[
                        xlabel={参数一},
                        ylabel={参数二},
                        enlargelimits=false,
                        width=4.5cm, height=4.5cm,
                        ytick=\empty,
                        xtick=\empty,
                        ticklabel style={font=\small},
                        label style={font=\small},
                        axis on top
                    ]
                    \addplot graphics [
                            xmin=-1, xmax=1, ymin=-1, ymax=1,
                        ] {./img/estimation/stap3.png};
                \end{axis}
            \end{tikzpicture}
        \caption{交替迭代STAP}
        \label{fig_stap_compare_3}
    \end{subfigure}
    \caption{三种方法的二维功率谱比较}
    \label{fig_stap_compare}
\end{figure}

\subsection{张量分解方法}
正如前文所述,考虑噪声的情况下,多域联合接收到的数据张量有如下模型:
\[
    \mathcal{X} = \mathcal{I}_{K} \times_1 \mathbf{R} \times_2 \mathbf{A} \times_3 \mathbf{B} + \mathcal{N} = \sum_{k=1}^K \bm{r}_k \circ \bm{a}_{\nu_k} \circ \bm{b}_{\mu_k} + \mathcal{N},
\]
其中信号buff可以看作是 \( K \) 个秩一张量的叠加。因此,作为SVD分解的推广,张量CP分解方法同样可以直接对接收数据张量进行处理,从而实现对目标参数的联合估计。对应的优化模型为
\[
        \min \quad \left\| \mathcal{X} - \mathcal{I}_K \times_1 \mathbf{R} \times_2 \mathbf{A} \times_3 \mathbf{B} \right\|_{\mathrm{F}}^2
\]
方便起见,将待求解的矩阵同样记作\( \mathbf{R} \in \mathbb{C}^{L \times K} \),\( \mathbf{A} \in \mathbb{C}^{M \times K} \),\( \mathbf{B} \in \mathbb{C}^{N \times K} \)。该优化问题可以通过交替最小二乘法(Alternating Least Squares, ALS)进行求解。具体而言,固定其中两个矩阵,仅对另一个矩阵进行优化。以优化矩阵 \( \mathbf{R} \) 为例,固定矩阵 \( \mathbf{A} \) 与 \( \mathbf{B} \) 后,根据\cref{prop:apx_kron_kprod},有
\[
    (\mathcal{I}_K \times_1 \mathbf{R} \times_2 \mathbf{A} \times_3 \mathbf{B})_1 = \mathbf{R} (\mathcal{I}_K \times_2 \mathbf{A} \times_3 \mathbf{B})_1
\]
其中\( (\mathcal{I}_K \times_2 \mathbf{A} \times_3 \mathbf{B})_1 \in \mathbb{C}^{K \times M N} \)表示张量\( \mathcal{I}_K \times_2 \mathbf{A} \times_3 \mathbf{B} \)的模-1展开矩阵。进一步地,利用 \cref{prop:kronecker-kron},有
\[
    (\mathcal{I}_K \times_2 \mathbf{A} \times_3 \mathbf{B})_{23} =  (\mathbf{B} \otimes \mathbf{A}) (\mathcal{I}_K)_{23} \in \mathbb{C}^{M N \times K}.
\]
记\( \mathbf{B} = \begin{bmatrix} \bm{b}_1 & \bm{b}_2 & \cdots & \bm{b}_K \end{bmatrix} \in \mathbb{C}^{N \times K} \),以及\( \mathbf{A} = \begin{bmatrix} \bm{a}_1 & \bm{a}_2 & \cdots & \bm{a}_K \end{bmatrix} \in \mathbb{C}^{M \times K} \),则有
\[
    \mathbf{B} \otimes \mathbf{A} = \begin{bmatrix}
        \mathbf{B} \otimes \bm{a}_1 & \mathbf{B} \otimes \bm{a}_2 & \cdots & \mathbf{B} \otimes \bm{a}_K
    \end{bmatrix} \in \mathbb{C}^{M N \times K^2}.
\]
与此同时,\( (\mathcal{I}_K)_{23} \)的形式为
\[
    (\mathcal{I}_K)_{23} = \begin{bmatrix}
        \mathbf{E}_1  \\ \mathbf{E}_2  \\ \vdots \\ \mathbf{E}_K
    \end{bmatrix} \in \mathbb{R}^{K^2 \times K},
\]
其中,\( \mathbf{E}_k \in \mathbb{R}^{K \times K} \)为一个稀疏矩阵,仅在第\( k \)列的第\( k \)行处取值为一,其余位置均为零。由此可见,矩阵 \( (\mathcal{I}_K)_{23} \) 实际上是一个选择矩阵,其作用是从矩阵 \( \mathbf{B} \otimes \mathbf{A} \) 的 \( K^2 \) 列中选择出其中的 \( K \) 列。因此,有
\[
    (\mathbf{B} \otimes \mathbf{A}) (\mathcal{I}_K)_{23} \in \mathbb{C}^{M N \times K} = \sum_{k=1}^K (\mathbf{B} \otimes \bm{a}_k) \mathbf{E}_k,
\]
而\( (\mathbf{B} \otimes \bm{a}_k) \mathbf{E}_k \)仅选择出矩阵 \( \mathbf{B} \otimes \bm{a}_k \) 的第 \( k \) 列,因此有
\[
    (\mathbf{B} \otimes \mathbf{A}) (\mathcal{I}_K)_{23} = \begin{bmatrix}
        0 & \cdots& \bm{b}_k \otimes \bm{a}_k & 0 & \cdots & 0
    \end{bmatrix}.
\]
所以,矩阵 \( (\mathcal{I}_K \times_2 \mathbf{A} \times_3 \mathbf{B})_{1} \) 可以表示为
\[
    (\mathcal{I}_K \times_2 \mathbf{A} \times_3 \mathbf{B})_{1} = \begin{bmatrix}
        \bm{b}_1 \otimes \bm{a}_1 & \bm{b}_2 \otimes \bm{a}_2 & \cdots & \bm{b}_K \otimes \bm{a}_K
    \end{bmatrix}^{\mathrm{T}} \in \mathbb{C}^{K \times M N}.
\]
通常也被记作为
\[
    \mathbf{B} \ast  \mathbf{A} = \begin{bmatrix}
        \bm{b}_1 \otimes \bm{a}_1 & \bm{b}_2 \otimes \bm{a}_2 & \cdots & \bm{b}_K \otimes \bm{a}_K
    \end{bmatrix} \in \mathbb{C}^{M N \times K},
\]
称为 Khatri-Rao 积(Khatri-Rao Product)。

此外,不难验证\( (\mathcal{I}_K \times_2 \mathbf{A} \times_3 \mathbf{B})_1 \in \mathbb{C}^{K \times M N} \)与\( \mathbf{B} \ast  \mathbf{A} = (\mathbf{B} \otimes \mathbf{A}) (\mathcal{I}_K)_{23} \in \mathbb{C}^{M N \times K} \)之间是转置关系。因此,优化矩阵 \( \mathbf{R} \) 的子问题可写成   
\[
    \min \quad \left\| \mathbf{X}_{1} - \mathbf{R} (\mathbf{B} \ast  \mathbf{A})^{\mathrm{T}} \right\|_{\mathrm{F}}^2,
\]
其中,\( \mathbf{X}_{1} \in \mathbb{C}^{L \times M N} \)为张量\( \mathcal{X} \)的模-1展开矩阵。该问题是一个标准的最小二乘问题,对应的闭式解为
\[
    \mathbf{R} = \mathbf{X}_{1} (\mathbf{B} \ast  \mathbf{A}) \left( (\mathbf{B} \ast  \mathbf{A})^{\mathrm{T}} (\mathbf{B} \ast  \mathbf{A}) \right)^{-1}.
\]
类似地,可以分别对矩阵 \( \mathbf{A} \) 与 \( \mathbf{B} \) 进行优化,最终得到如下的更新公式:
\[
    \begin{cases}
        \mathbf{R} = \mathbf{X}_{1} (\mathbf{B} \ast  \mathbf{A}) \left( (\mathbf{B} \ast  \mathbf{A})^{\mathrm{T}} (\mathbf{B} \ast  \mathbf{A}) \right)^{-1}, \\
        \mathbf{A} = \mathbf{X}_{2} (\mathbf{B} \ast \mathbf{R}) \left( (\mathbf{B} \ast \mathbf{R})^{\mathrm{T}} (\mathbf{B} \ast  \mathbf{R}) \right)^{-1}, \\
        \mathbf{B} = \mathbf{X}_{3} (\mathbf{A} \ast  \mathbf{R}) \left( (\mathbf{A} \ast  \mathbf{R})^{\mathrm{T}} (\mathbf{A} \ast  \mathbf{R}) \right)^{-1},
    \end{cases}
\]
其中,\( \mathbf{X}_{2} \in \mathbb{C}^{M \times L N} \)与\( \mathbf{X}_{3} \in \mathbb{C}^{N \times L M} \)分别为张量\( \mathcal{X} \)的模-2与模-3展开矩阵。交替迭代上述三个公式,直至收敛为止,即可得到最终的估计结果。如\cref{fig_cp_de}所示,对应的向量分别是\( K=2 \)时,张量分解得到的三个矩阵的列向量的实部。
\begin{figure}[htb!]
    \centering
    \begin{subfigure}{.6\textwidth}
        \centering
            \begin{tikzpicture}
                \begin{axis}[
                        xlabel={时间维}, ylabel={实部},
                        ticklabel style={font=\small},
                        label style={font=\small},
                        grid, width=7cm, height=4cm,
                        xmin=-2, xmax=2,
                        legend cell align=left,
                        legend style={
                                anchor=north east,
                                font=\tiny,
                                draw=none,
                                fill=none
                            }
                    ]
                    \addplot[
                        c1,
                        thick,
                        smooth,
                    ] table[x=x, y=r1, col sep=comma] {./img/estimation/cp_r.csv};
                    \addplot[
                        c2,
                        thick,
                        smooth,
                    ] table[x=x, y=r2, col sep=comma] {./img/estimation/cp_r.csv};
                \end{axis}
            \end{tikzpicture}
        \caption{矩阵\( \mathbf{R} \)的列向量实部}
        \label{fig_cp_de_1}
    \end{subfigure}

    \begin{subfigure}{.45\textwidth}
        \centering
        \begin{tikzpicture}
                \begin{axis}[
                        xlabel={参数一},
                        ticklabel style={font=\small},
                        label style={font=\small},
                        grid,width=5cm, height=4cm,
                        xmin=0, xmax=49,
                        legend cell align=left,
                        legend style={
                                anchor=north east,
                                font=\tiny,
                                draw=none,
                                fill=none
                            }
                    ]
                    \addplot[
                        c1,
                        thick,
                    ] table[x=x, y=a1, col sep=comma] {./img/estimation/cp_a.csv};
                    \addplot[
                        c2,
                        thick,
                    ] table[x=x, y=a2, col sep=comma] {./img/estimation/cp_a.csv};
                \end{axis}
            \end{tikzpicture}
        \caption{矩阵\( \mathbf{A} \)的列向量实部}
        \label{fig_cp_de_2}
    \end{subfigure}
    \begin{subfigure}{.45\textwidth}
        \centering
                \begin{tikzpicture}
                \begin{axis}[
                        xlabel={参数一},
                        ticklabel style={font=\small},
                        label style={font=\small},
                        grid,width=5cm, height=4cm,
                        xmin=0, xmax=49,
                        legend cell align=left,
                        legend style={
                                anchor=north east,
                                font=\tiny,
                                draw=none,
                                fill=none
                            }
                    ]
                    \addplot[
                        c1,
                        thick,
                    ] table[x=x, y=b1, col sep=comma] {./img/estimation/cp_b.csv};
                    \addplot[
                        c2,
                        thick,
                    ] table[x=x, y=b2, col sep=comma] {./img/estimation/cp_b.csv};
                \end{axis}
            \end{tikzpicture}
        \caption{矩阵\( \mathbf{B} \)的列向量实部}
        \label{fig_cp_de_3}
    \end{subfigure}
    \caption{张量分解结果}
    \label{fig_cp_de}
\end{figure}

可以发现,相较于SVD,其高阶推广 CP 分解获得的结果直接就对应每一个目标。事实上,可以证明,在特定条件下,对于高阶张量,其CP分解是唯一的。

此外,对于获得的矩阵\( \mathbf{A} \)和\( \mathbf{B} \),计算\( \mathbf{B} \ast \mathbf{A} \)的零空间,记作\( \mathbf{U} \),则类似于MUSIC方法,对应的二维谱函数可表示为
\[
    p_{\nu\mu} = \frac{1}{\| (\bm{b}_{\mu} \otimes \bm{a}_{\nu})^{\mathrm{T}} \mathbf{U} \|^2}.
\]
\cref{fig_cp_music}展示了CP分解结合MUSIC方法得到的二维谱函数。

\begin{figure}[htb!]
    \centering
        \begin{tikzpicture}
            \begin{axis}[
                    xlabel={参数一},
                    ylabel={参数二},
                    enlargelimits=false,
                    width=4.5cm, height=4.5cm,
                    ytick=\empty,
                    xtick=\empty,
                    ticklabel style={font=\small},
                    label style={font=\small},
                    axis on top
                ]
                \addplot graphics [
                        xmin=-1, xmax=1, ymin=-1, ymax=1,
                    ] {./img/estimation/cp_music.png};
            \end{axis}
        \end{tikzpicture}
    \caption{CP分解结合MUSIC方法得到的二维谱函数}
    \label{fig_cp_music}
\end{figure}
