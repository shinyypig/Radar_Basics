\chapter{目标参数估计}

在完成信号检测之后,雷达系统还需要进一步提取目标的关键参数,例如方位角、距离、多普勒频率以及极化特征等。这些参数的准确估计不仅关系到目标的定位与跟踪精度,也是后续态势感知与目标识别的重要基础。从算法发展的角度来看,目标参数估计大体经历了几个阶段:最初的经典波束形成方法,随后发展到最小方差处理方法,再到子空间方法和稀疏表示方法。近年来,随着多维观测手段的兴起,多域联合估计方法逐渐受到关注。这类方法能够同时利用空间、多普勒甚至极化等多维信息,从而实现更强的参数解耦和更高的估计精度。

在\cref{sec_radar_signal_model}中,我们介绍了雷达的接收信号模型。对于一个阵列多普勒雷达系统,其接收信号可以表示为一个张量:
\[
    \begin{split}
        \mathcal{X} = \mathcal{I}_K \times_1 \mathbf{R} \times_2 \mathbf{A} \times_3 \mathbf{B} \times_4 \mathbf{P},
    \end{split}
\]
其中,\( K \)为目标数,而矩阵\( \mathbf{R} \)、\( \mathbf{A} \)、\( \mathbf{B} \)和\( \mathbf{P} \)分别包含了目标的距离、方位角、俯仰角、多普勒信息。利用脉冲压缩技术,可以获取目标的距离信息,而本章的重点则是如何进一步从数据中提取目标的方位角和多普勒频率等参数。

\section{波束形成方法}
考虑均匀线性阵列或单阵元在多脉冲下的观测情形,此时,接收信号退化为一个矩阵:
\[
    \mathbf{X} = \mathbf{R} \mathbf{A}^{\mathrm{T}} \in \mathbb{C}^{L \times M},
\]
其中\( \mathbf{R} = \begin{bsmallmatrix} \bm{r}_1 & \bm{r}_2 & \cdots & \bm{r}_K \end{bsmallmatrix} \in \mathbb{C}^{L \times K} \)为目标信号矩阵,\( \mathbf{A} = \begin{bsmallmatrix} \bm{a}_{\omega_1} & \bm{a}_{\omega_2} & \cdots & \bm{a}_{\omega_K} \end{bsmallmatrix} \in \mathbb{C}^{M \times K} \)为阵列流形矩阵(导向矢量矩阵),其列向量为目标的空间或多普勒流形向量,其第\( k \)列通常有如下形式:
\[
    \bm{a}_{\omega_k} = \begin{bmatrix}
        1 & e^{j 2 \pi \omega_k} & \cdots & e^{j 2 \pi \omega_k (M-1)}
    \end{bmatrix}^{\mathrm{T}}.
\]
其中,参数 $\omega_k$ 的物理意义取决于雷达体制:若为阵列雷达,则 $\omega_k = \frac{d}{\lambda}\sin \theta_k$,与目标的方位角 $\theta_k$ 相关;若为单阵元多脉冲雷达,则 $\omega_k = \frac{2 T_r v_k}{\lambda}$,与目标的径向速度 $v_k$ 相关。

波束形成(Beamforming)是最经典的参数估计方法,其基本思想是通过对接收信号进行加权求和,形成一个指向特定方向的空间滤波器或指定速度的多普勒滤波器,从而增强该方向或速度上的信号分量。对于给定滤波器\( \bm{w}_{\omega} \),波束形成的输出可以表示为:
\[
    \bm{y}_{\omega} = \mathbf{X} \bm{w}_{\omega} \in \mathbb{C}^{L \times 1},
\]
可以看作是某个方向或速度上接收到的回波。

\subsection{经典波束形成}

经典波束形成(Conventional Beamforming, CBF)方法利用了不同频率复指数信号在维度充分大时趋于正交的性质,即当 $M$ 足够大时,有
\[
    \frac{1}{M}\bm{a}_{\omega_i}^{\mathrm{H}} \bm{a}_{\omega_j} \approx
    \begin{cases}
        1, & \omega_i = \omega_j,     \\
        0, & \omega_i \neq \omega_j .
    \end{cases}
\]
因此,将滤波器权向量选为与目标流形向量相匹配的形式:
\[
    \bm{w}_{\omega} = \frac{1}{M} \overline{\bm{a}}
    = \frac{1}{M} \begin{bmatrix}
        1 & e^{j 2 \pi \omega} & \cdots & e^{j 2 \pi \omega (M-1)}
    \end{bmatrix}^{\mathrm{H}},
\]
即可有效提取出对应参数值 $\omega$ 的信号分量。该方法又称为 \textbf{Bartlett 波束形成},是最经典的一种波束形成策略。

进一步地,通过扫描不同的参数值\( \omega \),将原始信号投影到不同的方向或速度上,则可以分别获得不同方向或速度上的信号分量。具体而言,对于阵列信号,可以构建如下矩阵:
\[
    \mathbf{W} = \frac{1}{M}\begin{bmatrix}
        1                                                   & 1                                                    & \cdots & 1                                                   \\
        e^{- j 2 \pi \frac{d \sin \theta_1}{\lambda}}       & e^{- j 2 \pi \frac{d \sin \theta_2}{\lambda}}        & \cdots & e^{- j 2 \pi \frac{d \sin \theta_N}{\lambda}}       \\
        \vdots                                              & \vdots                                               & \cdots & \vdots                                              \\
        e^{- j 2 \pi \frac{(M-1) d \sin \theta_1}{\lambda}} & e^{- j 2 \pi \frac{ (M-1) d \sin \theta_2}{\lambda}} & \cdots & e^{- j 2 \pi \frac{(M-1) d \sin \theta_N}{\lambda}}
    \end{bmatrix}\in \mathbb{C}^{M \times N},
\]
其中,\( \theta_1, \theta_2, \cdots, \theta_N \)为扫描的角度值。对于多脉冲信号,可以构建类似的矩阵:
\[
    \mathbf{W} = \frac{1}{M}\begin{bmatrix}
        1                                             & 1                                             & \cdots & 1                                             \\
        e^{- j 2 \pi \frac{2 T_r v_1}{\lambda}}       & e^{- j 2 \pi \frac{2 T_r v_2}{\lambda}}       & \cdots & e^{- j 2 \pi \frac{2 T_r v_N}{\lambda}}       \\
        \vdots                                        & \vdots                                        & \cdots & \vdots                                        \\
        e^{- j 2 \pi \frac{2 (M-1) T_r v_1}{\lambda}} & e^{- j 2 \pi \frac{2 (M-1) T_r v_2}{\lambda}} & \cdots & e^{- j 2 \pi \frac{2 (M-1) T_r v_N}{\lambda}}
    \end{bmatrix} \in \mathbb{C}^{M \times N},
\]
其中,\( v_1, v_2, \cdots, v_N \)为扫描的速度值。对应的波束形成可以表示为如下的矩阵乘法:
\[
    \mathbf{Y} = \mathbf{X} \mathbf{W} \in \mathbb{C}^{L \times N}.
\]
有趣的是,对于多脉冲信号,当所选扫描速度 $v_n$ 满足
\[
    \frac{2 T_r v_n}{\lambda} = \frac{n-1}{M}, \quad n = 1, 2, \cdots, N
\]
时,波束形成矩阵 $\mathbf{W}$ 正好对应于一个 $M \times M$ 的离散傅里叶变换(Discrete Fourier Transform,DFT)矩阵的前 $N$ 行。特别地,当 $M = N$ 时,$\mathbf{W}$ 即为标准的 DFT 矩阵。在这种情况下,波束形成过程等价于对接收信号在参数维度(慢时间维)上进行离散傅里叶变换,从而实现对多普勒频率的估计。

此外,根据\cref{apx.conv-corr-mat}中的结论,对 $\mathbf{X}$ 的列向量 $\bm{x}_i$ 进行脉冲压缩同样可以写成矩阵乘法形式:
\[
    \bm{y}_i = \mathbf{S}^{\mathrm{H}}\bm{x}_i,
\]
其中,$\mathbf{S}$ 为由参考向量构造的矩阵,有如下形式:
\[
    \mathbf{S} =
    \begin{bmatrix}
        s_1    & 0       & \cdots & 0      \\
        s_2    & s_1     & \cdots & 0      \\
        \vdots & \vdots  & \ddots & \vdots \\
        s_W    & s_{W-1} & \ddots & s_1    \\
        0      & s_W     & \ddots & s_2    \\
        \vdots & \vdots  & \ddots & \vdots \\
        0      & 0       & \cdots & s_W
    \end{bmatrix}.
\]
由此可见,当同时对接收信号矩阵 $\mathbf{X}$ 的行向量与列向量分别实施波束形成与脉冲压缩时,其结果可统一表示为
\[
    \mathbf{Z} = \mathbf{S}^{\mathrm{H}} \mathbf{X}\mathbf{W}.
\]

在矩阵 $\mathbf{Z}$ 中,目标表现为显著的峰值,其位置对应目标的距离及其他参数(方位或径向速度)。换言之,波束形成可视为在参数维度上的匹配滤波,即对目标信号执行``参数域脉冲压缩''。结合距离维与参数维的双重脉冲压缩,便可实现目标的定位与参数估计。

\begin{example}
    对\cref{fig_data_x}中的接收数据矩阵\( \mathbf{X} \)进行脉冲压缩和波束形成处理。

    \begin{figure}[htb!]
        \centering
        % \includegraphics[width=.4\textwidth]{./img}
        \begin{tikzpicture}
            \begin{axis}[
                    xlabel={参数维},
                    ylabel={时间维},
                    enlargelimits=false,
                    width=4.5cm, height=4.5cm,
                    ytick=\empty,
                    xtick=\empty,
                    ticklabel style={font=\small},
                    label style={font=\small},
                    axis on top
                ]
                \addplot graphics [
                        xmin=-1, xmax=1, ymin=-1, ymax=1,
                    ] {./img/estimation/est_1.png};
            \end{axis}
        \end{tikzpicture}
        \caption{接收数据矩阵\( \mathbf{X} \)(仅绘制了实部)}
        \label{fig_data_x}
    \end{figure}
\end{example}

\begin{solution}
    处理结果如\cref{fig_compressed}所示。\cref{fig_compressed_1} 给出了对 \( \mathbf{X} \) 的行向量进行脉冲压缩的结果:原本具有一定展宽的目标回波在压缩后转变为尖锐的脉冲,在图像中表现为水平直线;同时,由于两个目标对应的直线具有不同的频率起伏,说明其参数值(方位或速度)存在差异。

    \cref{fig_compressed_2} 展示了对 \( \mathbf{X} \) 的行向量进行波束形成的结果:两个目标在参数维上也被压缩为尖锐的脉冲,在图像中对应两条垂直直线,并且这两条直线明显呈现出 Chirp 信号的特征。

    \cref{fig_compressed_3} 则给出了在行向量和列向量上分别进行脉冲压缩与波束形成的联合结果:此时,两个目标在距离维和参数维均被压缩为尖锐脉冲,在图像中表现为两个亮点,其位置正好对应目标的距离和参数值。

    \begin{figure}[htb!]
        \centering
        \begin{subfigure}{.3\textwidth}
            \centering
            \begin{tikzpicture}
                \begin{axis}[
                        xlabel={参数维},
                        ylabel={时间维},
                        enlargelimits=false,
                        width=4.5cm, height=4.5cm,
                        ytick=\empty,
                        xtick=\empty,
                        ticklabel style={font=\small},
                        label style={font=\small},
                        axis on top
                    ]
                    \addplot graphics [
                            xmin=-1, xmax=1, ymin=-1, ymax=1,
                        ] {./img/estimation/est_2.png};
                \end{axis}
            \end{tikzpicture}
            \caption{\( \mathbf{S}\mathbf{X} \)实部}
            \label{fig_compressed_1}
        \end{subfigure}
        \begin{subfigure}{.3\textwidth}
            \centering
            \begin{tikzpicture}
                \begin{axis}[
                        xlabel={参数维},
                        ylabel={时间维},
                        enlargelimits=false,
                        width=4.5cm, height=4.5cm,
                        ytick=\empty,
                        xtick=\empty,
                        ticklabel style={font=\small},
                        label style={font=\small},
                        axis on top
                    ]
                    \addplot graphics [
                            xmin=-1, xmax=1, ymin=-1, ymax=1,
                        ] {./img/estimation/est_3.png};
                \end{axis}
            \end{tikzpicture}
            \caption{\( \mathbf{X}\mathbf{W} \)实部}
            \label{fig_compressed_2}
        \end{subfigure}
        \begin{subfigure}{.3\textwidth}
            \centering
            \begin{tikzpicture}
                \begin{axis}[
                        xlabel={参数维},
                        ylabel={时间维},
                        enlargelimits=false,
                        width=4.5cm, height=4.5cm,
                        ytick=\empty,
                        xtick=\empty,
                        ticklabel style={font=\small},
                        label style={font=\small},
                        axis on top
                    ]
                    \addplot graphics [
                            xmin=-1, xmax=1, ymin=-1, ymax=1,
                        ] {./img/estimation/est_4.png};
                \end{axis}
            \end{tikzpicture}
            \caption{\(  \mathbf{S}\mathbf{X}\mathbf{W} \)幅度}
            \label{fig_compressed_3}
        \end{subfigure}
        \caption{距离维和参数维二重脉冲压缩示意图}
        \label{fig_compressed}
    \end{figure}
\end{solution}

在某些情况下,我们可能并不关心目标的距离信息,而仅希望获得其参数信息(如方位或速度)。此时,可以直接利用波束形成后的输出计算参数维度上的功率谱。具体地,对于参数值 $\omega$,对应输出的功率可表示为
\[
    p_{\omega} = \frac{1}{L}\|\bm{y}_{\omega}\|_2^2
    = \frac{1}{L}\|\mathbf{X}\bm{w}_{\omega}\|_2^2
    = \frac{1}{L}\bm{w}_{\omega}^{\mathrm{H}}\mathbf{X}^{\mathrm{H}}\mathbf{X}\bm{w}_{\omega}
    = \bm{w}_{\omega}^{\mathrm{H}} \mathbf{\Sigma}_{\mathbf{X}} \bm{w}_{\omega},
\]
其中,$\mathbf{\Sigma}_{\mathbf{X}} = \frac{1}{L}\mathbf{X}^{\mathrm{H}}\mathbf{X}$ 为接收信号矩阵的协方差矩阵(一般假设接收信号均值为零)。通过对不同参数值 $\omega$ 进行扫描,即可得到参数维度的功率谱估计,从而实现目标参数的直接估计。

对\cref{fig_data_x}中的数据进行波束形成处理,并计算参数维度的功率谱,结果如\cref{fig_spectrum_cbf}所示。可以看到,功率谱在目标参数值处出现了明显的峰值,并且峰值在参数维度上的位置与\cref{fig_compressed_3}中的目标水平位置一致。通过对功率谱进行峰值检测,即可实现目标参数的估计。

\begin{figure}[htb!]
    \centering
    \begin{tikzpicture}
        \begin{axis}[
                xlabel=参数维, ylabel=归一化功率谱,
                ticklabel style={font=\small},
                label style={font=\small},
                % xtick=\empty,ytick=\empty,
                xmin = -0.5, xmax=0.5, ymin=0, ymax=1.2,
                grid, width=8cm, height=4cm,
                legend cell align=left,
                legend style={
                        anchor=north east,
                        font=\tiny,
                        draw=none,
                        fill=none
                    }
            ]
            \addplot[
                c1,
                thick,
            ] table[x=w, y=p, col sep=comma] {./img/estimation/cbf.csv};
        \end{axis}
    \end{tikzpicture}
    \caption{波束形成功率谱估计}
    \label{fig_spectrum_cbf}
\end{figure}

对于阵列信号而言,波束形成的一大优势是可以直接通过移相器硬件实现。通过控制各阵元的相位,对原始模拟信号加权求和,即可在空间上实现对特定方向的滤波。这样只需一路模数转换器对波束形成后的信号进行采样与处理,便能显著节省硬件资源。参数扫描则通过调节移相器的相位来完成,对应的雷达系统称为无源相控阵雷达(Passive Phased Array Radar)。这种方式的缺点在于无法同时观测多个方向的信号分量,因此在目标数量较多或干扰复杂的环境下,性能容易受限。相比之下,若对各阵元的信号分别采样并数字化,再通过数字处理实现波束合成,则能够同时获得多个方向的信号分量,从而提升系统的灵活性与抗干扰能力,但也会增加硬件开销。这类系统即为有源相控阵雷达(Active Phased Array Radar)。

尽管波束形成方法简单直观,但其分辨率和抗干扰能力有限,尤其在目标间距较近或存在强干扰时,性能会显著下降。对于一个特定的参数值\( \omega \),波束形成的输出可以进一步表示为:
\[
    \bm{y}_{\omega} = \mathbf{X} \bm{w}_{\omega} = \sum_{k=1}^{K} \bm{r}_k \bm{a}_{\omega_k}^{\mathrm{T}} \bm{w}_{\omega} = \sum_{k=1}^{K} \bm{r}_k \frac{\bm{a}_{\omega_k}^{\mathrm{T}} \overline{\bm{a}}_{\omega}}{M}.
\]
从中可以发现,波束形成的输出实际上是各个目标回波信号的加权和,对应的权重为\( \frac{\bm{a}_{\omega_k}^{\mathrm{T}} \overline{\bm{a}}_{\omega}}{M} \)。理想情况下,当目标参数值与滤波器参数值相匹配时,该权重接近1,否则趋于0。但在实际应用中,\( M \)的数值有限,此时即便\( \omega_k \neq \omega \)时,对应的权重也可能较大,导致提取出来的信号中混入了其他目标的成分,从而影响估计精度。

注意到
\[
    \begin{split}
        \frac{\bm{a}_{\omega_k}^{\mathrm{T}} \overline{\bm{a}}_{\omega}}{M} & = \frac{1}{M} \sum_{m=0}^{M-1} e^{j 2 \pi m (\omega_k - \omega)} = \frac{1}{M} \frac{1 - e^{j 2 \pi M (\omega_k - \omega)}}{1 - e^{j 2 \pi (\omega_k - \omega)}} \\
                                                                            & = \frac{1}{M} \frac{\sin(\pi M (\omega_k - \omega))}{\sin(\pi (\omega_k - \omega))} e^{j \pi (M-1)(\omega_k - \omega)}.
    \end{split}
\]
可以发现该权重的取值仅和\( M \)以及参数差值\( \omega_k - \omega \)有关。如\cref{fig_weight}所示,随着\( M \)的增大,权重函数的主瓣变窄,旁瓣降低,从而提升了参数分辨能力。

\begin{figure}[htb!]
    \centering
    \begin{tikzpicture}
        \begin{axis}[
                xlabel=$ \omega_k - \omega $, ylabel=权重幅度,
                ticklabel style={font=\small},
                label style={font=\small},
                grid, smooth,
                xmin = -0.5, xmax=0.5, ymin=0, ymax=1,
                width=10cm, height=5cm,
                legend cell align=left,
                clip=false,
                legend style={
                        anchor=north east,
                        font=\tiny,
                        draw=none,
                        fill=none
                    }
            ]
            \addplot[
                c1,
                thick,
            ] table[x=w, y=a1, col sep=comma] {./img/estimation/weight.csv};
            \addlegendentry{\( M=11 \)};
            \addplot[
                c2,
                thick,
            ] table[x=w, y=a2, col sep=comma] {./img/estimation/weight.csv};
            \addlegendentry{\( M=21 \)};
            \addplot[
                c3,
                thick,
            ] table[x=w, y=a3, col sep=comma] {./img/estimation/weight.csv};
            \addlegendentry{\( M=51 \)};
        \end{axis}
    \end{tikzpicture}
    \caption{不同\( M \)值下波束形成权重函数}
    \label{fig_weight}
\end{figure}

\subsection{和差比幅测角}
无源相控阵节约了硬件开销,但其无法同时观测多个方向的信号分量,有源相控阵虽然提升了灵活性,但硬件成本大大增加。为此,雷达系统中常采用一种折衷的方案,即和差比幅测角(Sum–Difference Amplitude Comparison)方法。 该方法通过将阵列划分为多个子阵,并对各子阵的输出进行加权组合,从而在一定程度上兼顾了硬件复杂度与参数估计性能。

不妨考虑一个均匀线性阵列,其阵元总数为 $2M$,并将其划分为两个子阵,每个子阵包含 $M$ 个阵元。对于给定方位角 $\theta$,和差比幅测角方法构造如下两组滤波器:
\[
    \bm{w}_{\text{sum}} = \frac{1}{2M}
    \begin{bmatrix}
        \bm{w}_1 \\
        \bm{w}_2
    \end{bmatrix}, \quad  \bm{w}_{\text{diff}} = \frac{1}{2M}
    \begin{bmatrix}
        \bm{w}_1 \\
        -\bm{w}_2
    \end{bmatrix},
\]
其中,\( \bm{w}_1 \) 和 \( \bm{w}_2 \) 分别为前\( M \) 个阵元和后\( M \) 个阵元的权向量,有如下形式:
\[
    \bm{w}_1 = \begin{bmatrix}
        1                                           \\
        e^{- j 2 \pi \frac{d \sin \theta}{\lambda}} \\
        \vdots                                      \\
        e^{- j 2 \pi \frac{(M-1) d \sin \theta}{\lambda}}
    \end{bmatrix}, \quad
    \bm{w}_2 = \begin{bmatrix}
        e^{- j 2 \pi \frac{M d \sin \theta}{\lambda}}     \\
        e^{- j 2 \pi \frac{(M+1) d \sin \theta}{\lambda}} \\
        \vdots                                            \\
        e^{- j 2 \pi \frac{(2M-1) d \sin \theta}{\lambda}}
    \end{bmatrix}.
\]
记 $\bm{w}_{\text{sum}}$ 为两个子阵的和波束,$\bm{w}_{\text{diff}}$ 为两个子阵的差波束。对于位于方位角 $\theta_k$ 的目标,其波束形成输出分别为
\[
    \bm{y}_{\text{sum}} =  \bm{r}_k \bm{a}_{\omega_k}^{\mathrm{T}} \bm{w}_{\text{sum}},
    \quad
    \bm{y}_{\text{diff}} = \bm{r}_k \bm{a}_{\omega_k}^{\mathrm{T}} \bm{w}_{\text{diff}}.
\]
可以看出,两者在时域波形上保持完全一致,仅在复数比例因子上有所不同,分别为$\bm{a}_{\omega_k}^{\mathrm{T}} \bm{w}_{\text{sum}}$ 和 $\bm{a}_{\omega_k}^{\mathrm{T}} \bm{w}_{\text{diff}}$。该比例因子同时包含幅度与相位信息,其比值正是和差比幅测角所利用的参数。

为了分析这两个幅度的关系,同样将\( \bm{a}_{\omega_k} \) 拆分为两部分:
\[
    \bm{a}_{\omega_k} =
    \begin{bmatrix}
        \bm{a}_{\omega_k,1} \\
        \bm{a}_{\omega_k,2}
    \end{bmatrix},
\]
其中,
\[
    \bm{a}_{\omega_k,1} = \begin{bmatrix}
        1                                           \\
        e^{j 2 \pi \frac{d \sin \theta_k}{\lambda}} \\
        \vdots                                      \\
        e^{j 2 \pi \frac{(M-1) d \sin \theta_k}{\lambda}}
    \end{bmatrix}, \quad
    \bm{a}_{\omega_k,2} = \begin{bmatrix}
        e^{j 2 \pi \frac{M d \sin \theta_k}{\lambda}}     \\
        e^{j 2 \pi \frac{(M+1) d \sin \theta_k}{\lambda}} \\
        \vdots                                            \\
        e^{j 2 \pi \frac{(2M-1) d \sin \theta_k}{\lambda}}
    \end{bmatrix}.
\]
那么,对于和波束,我们有
\[
    \begin{split}
        \bm{a}_{\omega_k}^{\mathrm{T}} \bm{w}_{\text{sum}} & = \frac{1}{2M}\begin{bmatrix}
                                                                               \bm{a}_{\omega_k,1}^{\mathrm{T}} & \bm{a}_{\omega_k,2}^{\mathrm{T}}
                                                                           \end{bmatrix}
        \begin{bmatrix}
            \bm{w}_1 \\
            \bm{w}_2
        \end{bmatrix}                                                                                                                                                                                                                                                                                  \\
                                                           & = \frac{1}{2M} \bm{a}_{\omega_k,1}^{\mathrm{T}} \bm{w}_1 + \frac{1}{2M} \bm{a}_{\omega_k,2}^{\mathrm{T}} \bm{w}_2                                                                                                                          \\
                                                           & = \frac{1}{2M} \bm{a}_{\omega_k,1}^{\mathrm{T}} \bm{w}_1 + \frac{1}{2M} \left( e^{j 2 \pi \frac{M d \sin \theta_k}{\lambda}} \bm{a}_{\omega_k,1}^{\mathrm{T}} \right) \left( e^{-j 2 \pi \frac{M d \sin \theta}{\lambda}} \bm{w}_1 \right) \\
                                                           & = \frac{1 + e^{j 2\pi \frac{M d (\sin \theta_k - \sin \theta)}{\lambda}}}{2M} \bm{a}_{\omega_k,1}^{\mathrm{T}} \bm{w}_1.
    \end{split}
\]
类似地,对于差波束,有
\[
    \begin{split}
        \bm{a}_{\omega_k}^{\mathrm{T}} \bm{w}_{\text{diff}} & = \frac{1}{2M}\begin{bmatrix}
                                                                                \bm{a}_{\omega_k,1}^{\mathrm{T}} & \bm{a}_{\omega_k,2}^{\mathrm{T}}
                                                                            \end{bmatrix}
        \begin{bmatrix}
            \bm{w}_1 \\
            -\bm{w}_2
        \end{bmatrix}                                                                                                                                                                                                                                                                                     \\
                                                            & = \frac{1}{2M} \bm{a}_{\omega_k,1}^{\mathrm{T}} \bm{w}_1 - \frac{1}{2M} \bm{a}_{\omega_k,2}^{\mathrm{T}} \bm{w}_2                                                                                                                            \\
                                                            & = \frac{1}{2M} \bm{a}_{\omega_k,1}^{\mathrm{T}} \bm{w}_1 - \frac{1}{2M} \left( e^{j 2 \pi \frac{M d \sin \theta_k}{\lambda}} \bm{a}_{\omega_k,1}^{\mathrm{T}} \right) \left( e^{-j 2 \pi \frac{M d \sin \theta_k}{\lambda}} \bm{w}_1 \right) \\
                                                            & = = \frac{1 - e^{j 2\pi \frac{M d (\sin \theta_k - \sin \theta)}{\lambda}}}{2M} \bm{a}_{\omega_k,1}^{\mathrm{T}} \bm{w}_1.
    \end{split}
\]

不妨令\( y_{\text{sum}} \)和\( y_{\text{diff}} \)分别为对\( \bm{y}_{\text{sum}} \)和\( \bm{y}_{\text{diff}} \)进行脉冲压缩后的峰值,则有
\[
    \frac{y_{\text{diff}}}{y_{\text{sum}}} = \frac{1 - e^{j 2\pi \frac{M d (\sin \theta_k - \sin \theta)}{\lambda}}}{1 + e^{j 2\pi \frac{M d (\sin \theta_k - \sin \theta)}{\lambda}}}.
\]
记\( \phi = \pi \frac{M d (\sin \theta_k - \sin \theta)}{\lambda} \),则上式可化简为
\[
    \frac{y_{\text{diff}}}{y_{\text{sum}}} = \frac{1 - e^{j 2 \phi}}{1 + e^{j 2 \phi}} = \frac{e^{-j \phi} - e^{j \phi}}{e^{-j \phi} + e^{j \phi}} = \frac{-2j \sin \phi}{2 \cos \phi} =-j \tan \phi.
\]
因此,两个峰值的虚部之比为
\[
    r = \operatorname{Imag}\left( \frac{y_{\text{diff}}}{y_{\text{sum}}} \right) = - \tan \phi = - \tan \left( \pi \frac{M d (\sin \theta_k - \sin \theta)}{\lambda} \right).
\]

注意到,在\( \theta_k \)与\( \theta \)接近的情况下,有
\[
    \sin \theta_k - \sin \theta = 2 \cos \left( \frac{\theta_k + \theta}{2} \right) \sin \left( \frac{\theta_k - \theta}{2} \right) \approx 2 \cos \theta  \frac{\theta_k - \theta}{2} = \cos \theta (\theta_k - \theta).
\]
将其代入上式,得到
\[
    r \approx - \tan \left( \pi \frac{M d \cos \theta}{\lambda} (\theta_k - \theta) \right),
\]
进一步推导,有
\[
    \theta_k \approx \theta + \frac{\lambda}{\pi M d \cos \theta} \arctan (-r).
\]

\cref{fig_ac}展示了\( \theta \)等于0时,目标角度\( \theta_k \)和和差比幅值\( r \)之间的关系曲线。可以看到,随着目标角度偏离扫描角度,和差比幅值呈现出非线性变化的趋势。通过测量和差比幅值\( r \),并结合上述关系式,即可实现对目标方位角\( \theta_k \)的估计。

\begin{figure}[htb!]
    \centering
    \begin{tikzpicture}
        \begin{axis}[
                xlabel=\( \theta_k(\text{度}) \), ylabel=\( r \),
                ticklabel style={font=\small},
                label style={font=\small},
                grid, xmin=-6, xmax=6,
                width=6cm, height=6cm,
                legend cell align=left,
                legend style={
                        anchor=north east,
                        font=\tiny,
                        draw=none,
                        fill=none
                    }
            ]
            \addplot[
                c1,
                thick,
            ] table[x=x, y=y, col sep=comma] {./img/estimation/ac.csv};
        \end{axis}
    \end{tikzpicture}
    \caption{目标角度与和差比幅值关系曲线}
    \label{fig_ac}
\end{figure}

相较于无源相控阵雷达,和差比幅法需要额外增加一路模数转换器。但这一硬件代价带来的好处是,在主瓣范围内,目标角度能够得到更为精确的测量。此外,和差比幅法的计算复杂度也较低,主要涉及简单的加减运算和脉冲压缩处理,适合实时应用。

\subsection{最小方差无失真响应}
经典波束形成方法并未考虑噪声与干扰的统计特性,因此在复杂环境下性能往往受到限制。为此,最小方差无失真响应(Minimum Variance Distortionless Response, MVDR)方法在设计滤波器时引入了噪声与干扰的协方差矩阵,通过优化权向量以最小化输出功率,同时保证对目标信号无失真响应。该方法的发明者为 Jack Capon,因此有时也称为 Capon 算法。

设 \( \bm{w}_{\omega} \) 为待设计的滤波器权向量,则对应的输出为
\[
    \bm{y}_{\omega} = \mathbf{X} \bm{w}_{\omega}.
\]
直观地,我们希望输出 \(\bm{y}_{\omega}\) 尽可能只包含参数值为 \(\omega\) 的目标信号,同时抑制其他信号与噪声干扰。注意到目标信号对应的导向向量为 \(\bm{a}_{\omega}\),因此目标分量在输出中的占比与 \(|\bm{a}_{\omega}^{\mathrm{T}} \bm{w}_{\omega}|\) 成正比。由此可构造如下优化问题:
\[
    \max_{\bm{w}_{\omega}} \frac{|\bm{a}_{\omega}^{\mathrm{T}} \bm{w}_{\omega}|^2}{ \frac{1}{L} \|\bm{y}_{\omega}\|_2^2},
\]
即在保证输出功率尽可能小的前提下,使目标信号成分最大化。该问题可转化为如下形式的约束优化问题:
\[
    \begin{cases}
        \min_{\bm{w}_{\omega}} \quad & \frac{1}{L} \|\bm{y}_{\omega}\|_2^2               \\
        \text{s.t.} \quad            & \bm{a}_{\omega}^{\mathrm{T}} \bm{w}_{\omega} = 1,
    \end{cases}
\]
其中约束条件保证了对目标信号的无失真响应。

记接收信号的协方差矩阵为
\[
    \mathbf{\Sigma}_{\mathbf{X}} = \frac{1}{L} \mathbf{X}^{\mathrm{H}} \mathbf{X},
\]
则目标函数可重写为
\[
    \frac{1}{L} \|\bm{y}_{\omega}\|_2^2
    = \frac{1}{L} \bm{w}_{\omega}^{\mathrm{H}} \mathbf{X}^{\mathrm{H}} \mathbf{X} \bm{w}_{\omega}
    = \bm{w}_{\omega}^{\mathrm{H}} \mathbf{\Sigma}_{\mathbf{X}} \bm{w}_{\omega}.
\]
利用拉格朗日乘子法可得该优化问题的闭式解为
\[
    \bm{w}_{\omega}
    = \frac{\mathbf{\Sigma}_{\mathbf{X}}^{-1} \overline{\bm{a}}_{\omega}}
    {\overline{\bm{a}}_{\omega}^{\mathrm{H}} \mathbf{\Sigma}_{\mathbf{X}}^{-1} \overline{\bm{a}}_{\omega}}.
\]

遍历所有参数值 \(\omega\),即可构造滤波器组矩阵
\[
    \mathbf{W} = \begin{bmatrix} \bm{w}_{\omega_1} & \bm{w}_{\omega_2} & \cdots \end{bmatrix},
\]
并由此得到波束形成的输出矩阵
\[
    \mathbf{Y} = \mathbf{X} \mathbf{W}.
\]
与经典波束形成类似,可通过计算各列向量的功率来获得参数维度的功率谱:
\[
    p_{\omega}
    = \frac{1}{L} \|\bm{y}_{\omega}\|_2^2
    = \bm{w}_{\omega}^{\mathrm{H}} \mathbf{\Sigma}_{\mathbf{X}} \bm{w}_{\omega}
    = \frac{1}{\overline{\bm{a}}_{\omega}^{\mathrm{H}} \mathbf{\Sigma}_{\mathbf{X}}^{-1} \overline{\bm{a}}_{\omega}}.
\]

\begin{example}
    对\cref{fig_noise_capon}中的包含噪声的数据分别进行 CBF 和 Capon 方法处理,计算对应的二维脉压结果和参数维度的功率谱估计。
    \begin{figure}[htb!]
        \centering
        \begin{tikzpicture}
            \begin{axis}[
                    xlabel={参数维},
                    ylabel={时间维},
                    enlargelimits=false,
                    width=5cm, height=5cm,
                    ytick=\empty,
                    xtick=\empty,
                    ticklabel style={font=\small},
                    label style={font=\small},
                    axis on top
                ]
                \addplot graphics [
                        xmin=-1, xmax=1, ymin=-1, ymax=1,
                    ] {./img/estimation/capon_1.png};
            \end{axis}
        \end{tikzpicture}
        \caption{有噪声数据矩阵\( \mathbf{X} \)(仅绘制了实部)}
        \label{fig_noise_capon}
    \end{figure}
\end{example}
\begin{solution}
    二维脉压结果如\cref{fig_compressed}所示。与经典波束形成相比,Capon 方法在参数维度上展现出更高的分辨率,两个目标在参数维上被压缩为更尖锐的脉冲,并且旁瓣水平显著降低,从而提升了目标的可分辨性和检测性能。
    \begin{figure}[htb!]
        \centering
        \begin{subfigure}{.4\textwidth}
            \centering
            \begin{tikzpicture}
                \begin{axis}[
                        xlabel={参数维},
                        ylabel={时间维},
                        enlargelimits=false,
                        width=5cm, height=5cm,
                        ytick=\empty,
                        xtick=\empty,
                        ticklabel style={font=\small},
                        label style={font=\small},
                        axis on top
                    ]
                    \addplot graphics [
                            xmin=-1, xmax=1, ymin=-1, ymax=1,
                        ] {./img/estimation/capon_2.png};
                \end{axis}
            \end{tikzpicture}
            \caption{CBF}
            \label{fig_compressed_capon_1}
        \end{subfigure}
        \begin{subfigure}{.4\textwidth}
            \centering
            \begin{tikzpicture}
                \begin{axis}[
                        xlabel={参数维},
                        ylabel={时间维},
                        enlargelimits=false,
                        width=5cm, height=5cm,
                        ytick=\empty,
                        xtick=\empty,
                        ticklabel style={font=\small},
                        label style={font=\small},
                        axis on top
                    ]
                    \addplot graphics [
                            xmin=-1, xmax=1, ymin=-1, ymax=1,
                        ] {./img/estimation/capon_3.png};
                \end{axis}
            \end{tikzpicture}
            \caption{Capon}
            \label{fig_compressed_capon_2}
        \end{subfigure}
        \caption{CBF 和 Capon 方法处理后的数据矩阵的幅度图}
        \label{fig_compressed_capon}
    \end{figure}

    参数维度的功率谱估计如\cref{fig_compressed_capon2}所示。可以看到,Capon 方法在参数维度上展现出更高的分辨率,两个目标在参数维上被压缩为更尖锐的脉冲,并且旁瓣水平显著降低,有效提升了目标的可分辨性和检测性能。
    \begin{figure}[htb!]
        \centering
        \begin{tikzpicture}
            \begin{axis}[
                    xlabel={参数维}, ylabel={归一化功率谱},
                    ticklabel style={font=\small},
                    xmin=-0.5, xmax=0.5, ymin=0, ymax=1.2,
                    % xtick=\empty,ytick=\empty,
                    width=8cm, height=4cm,
                    label style={font=\small},
                    grid,
                    legend cell align=left,
                    legend style={
                            anchor=north east,
                            font=\tiny,
                            draw=none,
                            fill=none
                        }
                ]
                \addplot[
                    c1,
                    thick,
                ] table[x=w, y=p1, col sep=comma] {./img/estimation/capon.csv};
                \addlegendentry{CBF}
                \addplot[
                    c2,
                    thick,
                ] table[x=w, y=p2, col sep=comma] {./img/estimation/capon.csv};
                \addlegendentry{Capon}
            \end{axis}
        \end{tikzpicture}
        \caption{CBF 和 Capon 方法处理后的数据矩阵的幅度图}
        \label{fig_compressed_capon2}
    \end{figure}
\end{solution}


在实际应用中,由于接收信号的样本数量有限,协方差矩阵的估计往往不够准确,甚至可能不可逆。为缓解这一问题,可以借鉴岭回归(Ridge Regression)的思想,在目标函数中加入权重向量的范数惩罚项,从而得到如下优化模型:
\[
    \begin{aligned}
        \min_{\bm{w}_{\omega}} \quad &
        \bm{w}_{\omega}^{\mathrm{H}} \mathbf{\Sigma}_{\mathbf{X}} \bm{w}_{\omega}
        + \delta \|\bm{w}_{\omega}\|_2^2                                                 \\
        \text{s.t.} \quad            & \bm{a}_{\omega}^{\mathrm{T}} \bm{w}_{\omega} = 1,
    \end{aligned}
\]
其中 \(\delta > 0\) 为正则化参数。利用拉格朗日乘子法,可得其闭式解为
\[
    \bm{w}_{\omega}
    = \frac{(\mathbf{\Sigma}_{\mathbf{X}} + \delta \mathbf{I})^{-1} \,\overline{\bm{a}}_{\omega}}
    {\overline{\bm{a}}_{\omega}^{\mathrm{H}} (\mathbf{\Sigma}_{\mathbf{X}} + \delta \mathbf{I})^{-1} \,\overline{\bm{a}}_{\omega}}.
\]

由此可见,该方法等价于将原协方差矩阵替换为 \(\mathbf{\Sigma}_{\mathbf{X}} + \delta \mathbf{I}\),在保证无失真约束的同时显著改善了矩阵的条件数,从而增强了算法的鲁棒性。这一技巧通常被称为对角加载(Diagonal Loading)。

\section{子空间方法}
从滤波的角度来看,经典波束形成与 Capon 算法本质上都是通过设计权向量,对接收信号矩阵进行加权叠加,以实现对特定参数下的目标回波的增强,以及对噪声和干扰的抑制。不同之处在于,Capon 算法利用接收信号的协方差矩阵,在保证无失真响应的同时最小化输出功率,因此在噪声和干扰抑制方面较经典波束形成更为有效。然而,Capon 算法依然受限于“滤波”这一框架:对于白噪声而言,无论滤波器如何设计,其输出噪声功率的下限始终等于噪声本身。因此,即便在高信噪比条件下,Capon 谱的分辨率仍受限于通道数\( M \),难以满足高精度参数估计的需求。

为突破 Capon 算法的分辨率瓶颈,研究者提出了基于子空间的高分辨率方法。其核心思想不再依赖滤波器权向量的设计,而是通过对接收信号协方差矩阵进行特征分解,将信号与噪声划分到互相正交的子空间中,并利用这种正交性实现参数估计。借助这一几何结构,子空间方法能够获得远超传统滤波框架的分辨能力。典型代表包括 MUSIC 和 ESPRIT 方法,它们利用信号子空间与噪声子空间的正交特性,实现了高分辨率的参数估计。

\subsection{MUSIC方法}
考虑含噪接收信号模型
\[
    \mathbf{X} = \mathbf{R}\mathbf{A}^{\mathrm{T}} + \mathbf{N} \in \mathbb{C}^{L \times M},
\]
其中 \(\mathbf{N}\) 为加性噪声。若噪声为零均值、方差为 \(\sigma^2\) 的白噪声,则其协方差矩阵近似为对角矩阵:
\[
    \mathbf{\Sigma}_{\mathbf{N}}
    = \frac{1}{L}\mathbf{N}^{\mathrm{H}}\mathbf{N}
    \approx \sigma^2 \mathbf{I} \in \mathbb{C}^{M \times M}.
\]

对于信号部分,一般假设各目标对应的距离互不相同,即 \(\mathbf{R}\) 的列向量互相正交。此时信号协方差矩阵可写为
\[
    \mathbf{\Sigma}_{\mathbf{S}}
    = \frac{1}{L}\,\overline{\mathbf{A}}\,\mathbf{R}^{\mathrm{H}}\mathbf{R}\,\mathbf{A}^{\mathrm{T}}
    = \overline{\mathbf{A}}
    \left(\frac{1}{L}\mathbf{R}^{\mathrm{H}}\mathbf{R}\right)
    \mathbf{A}^{\mathrm{T}}
    \approx \overline{\mathbf{A}}\,\mathbf{\Sigma}_{\mathbf{R}}\,\mathbf{A}^{\mathrm{T}},
\]
其中
\[
    \mathbf{\Sigma}_{\mathbf{R}}
    = \frac{1}{L}\mathbf{R}^{\mathrm{H}}\mathbf{R}
    = \operatorname{diag}(\sigma_1^2,\sigma_2^2,\cdots,\sigma_K^2),
\]
其对角元素即为各目标的功率。又因为信号与噪声独立,即满足\(\mathbf{R}\mathbf{A}^{\mathrm{T}}\mathbf{N}^{\mathrm{H}} \approx \mathbf{0}\),则整个接收信号的协方差矩阵为
\[
    \begin{split}
        \mathbf{\Sigma}_{\mathbf{X}}
         & = \frac{1}{L}\big(\mathbf{R}\mathbf{A}^{\mathrm{T}}+\mathbf{N}\big)^{\mathrm{H}}
        \big(\mathbf{R}\mathbf{A}^{\mathrm{T}}+\mathbf{N}\big)                                   \\
         & \approx \frac{1}{L}\mathbf{A}\mathbf{R}^{\mathrm{H}}\mathbf{R}\mathbf{A}^{\mathrm{T}}
        + \frac{1}{L}\mathbf{N}^{\mathrm{H}}\mathbf{N}                                           \\
         & = \mathbf{\Sigma}_{\mathbf{S}} + \mathbf{\Sigma}_{\mathbf{N}}                         \\
         & \approx \overline{\mathbf{A}}\,\mathbf{\Sigma}_{\mathbf{R}}\,\mathbf{A}^{\mathrm{T}}
        + \sigma^2 \mathbf{I}.
    \end{split}
\]

考虑到矩阵 \(\mathbf{A}\) 的列向量由不同频率的复指数信号组成,而当阵元数 \(M\) 足够大时,不同频率的复指数信号近似正交。因此,对于协方差矩阵 \(\mathbf{\Sigma}_{\mathbf{X}}\),不难验证 \(\mathbf{A}\) 的列向量的共轭 \(\overline{\bm{a}}_{\omega_k}\) 是其特征向量,对应的特征值为 \(\sigma_k^2 M + \sigma^2\)。具体推导如下:
\[
    \begin{split}
        \mathbf{\Sigma}_{\mathbf{X}} \overline{\bm{a}}_{\omega_k}
         & \approx \left( \overline{\mathbf{A}} \mathbf{\Sigma}_{\mathbf{R}} \mathbf{A}^{\mathrm{T}} + \sigma^2 \mathbf{I} \right) \overline{\bm{a}}_{\omega_k}
        = \overline{\mathbf{A}} \mathbf{\Sigma}_{\mathbf{R}} \left( \mathbf{A}^{\mathrm{T}} \overline{\bm{a}}_{\omega_k} \right) + \sigma^2 \overline{\bm{a}}_{\omega_k}                            \\
         & = \overline{\mathbf{A}} \mathbf{\Sigma}_{\mathbf{R}} \begin{bmatrix}
                                                                    0      \\
                                                                    \vdots \\
                                                                    M      \\
                                                                    \vdots \\
                                                                    0
                                                                \end{bmatrix} + \sigma^2 \overline{\bm{a}}_{\omega_k} = \overline{\mathbf{A}} \begin{bmatrix}
                                                                                                                                                  0            \\
                                                                                                                                                  \vdots       \\
                                                                                                                                                  \sigma_k^2 M \\
                                                                                                                                                  \vdots       \\
                                                                                                                                                  0
                                                                                                                                              \end{bmatrix} + \sigma^2 \overline{\bm{a}}_{\omega_k} \\
         & = \sigma_k^2 M \overline{\bm{a}}_{\omega_k} + \sigma^2 \overline{\bm{a}}_{\omega_k}
        = \left( \sigma_k^2 M + \sigma^2 \right) \overline{\bm{a}}_{\omega_k}.
    \end{split}
\]

另一方面,一般有 \(M > K\),因此 \(\mathbf{\Sigma}_{\mathbf{X}}\) 至少还存在 \(M-K\) 个不在 \(\mathbf{A}\) 列空间内的特征向量,记为 \(\bm{e}_{K+1},\bm{e}_{K+2},\dots,\bm{e}_M\)(方便起见,令\( \bm{e}_k \)的模长与\( \bm{a}_{\omega_k}\) 模长相同,都为\( M \))。由于协方差矩阵为共轭对称矩阵,其特征向量必然正交,因此有:
\[
    \mathbf{\Sigma}_{\mathbf{X}} \bm{e}_i
    \approx \big(\overline{\mathbf{A}}\,\mathbf{\Sigma}_{\mathbf{R}}\,\mathbf{A}^{\mathrm{T}} + \sigma^2 \mathbf{I}\big)\bm{e}_i
    = \sigma^2 \bm{e}_i,\qquad i=K+1,\dots,M.
\]

综上,\(\mathbf{\Sigma}_{\mathbf{X}}\) 的特征结构可分为两类:由 \(\overline{\bm{a}}_{\omega_1},\dots,\overline{\bm{a}}_{\omega_K}\) 张成的信号子空间,对应特征值为\(\sigma_1^2 M+\sigma^2,\dots,\sigma_K^2 M+\sigma^2\);由 \(\bm{e}_{K+1},\dots,\bm{e}_M\) 张成的噪声子空间,对应特征值均为 \(\sigma^2\);且信号子空间与噪声子空间彼此正交。

又因为信号对应的特征值满足\(\sigma_k^2 M + \sigma^2 > \sigma^2\),故可通过对 \(\mathbf{\Sigma}_{\mathbf{X}}\) 进行特征分解,提取最大的 \(K\) 个特征值及其特征向量来构造信号子空间,其余部分即为噪声子空间。具体地,设
\[
    \mathbf{\Sigma}_{\mathbf{X}}
    = \mathbf{U}\mathbf{\Lambda}\mathbf{U}^{\mathrm{H}}
    = \begin{bmatrix}
        \mathbf{U}_{s} & \mathbf{U}_{n}
    \end{bmatrix}
    \begin{bmatrix}
        \mathbf{\Lambda}_{s} & \mathbf{0}           \\
        \mathbf{0}           & \mathbf{\Lambda}_{n}
    \end{bmatrix}
    \begin{bmatrix}
        \mathbf{U}_{s}^{\mathrm{H}} \\
        \mathbf{U}_{n}^{\mathrm{H}}
    \end{bmatrix},
\]
其中,\(\mathbf{U}_{s}\) 包含前 \(K\) 个最大特征值对应的特征向量,构成信号子空间;\(\mathbf{U}_{n}\) 包含剩余 \(M-K\) 个特征值对应的特征向量,构成噪声子空间。

% 考虑到\( \mathbf{A} \)的列向量为不同频率的复指数信号,而不同频率的复指数信号随着\( M \)的增大趋于正交。因此,对于协方差矩阵\( \mathbf{\Sigma}_{\mathbf{X}} \),不难验证,矩阵\( \mathbf{A} \)的列向量的共轭\( \overline{\bm{a}}_{\omega_k} \)为其特征向量,且对应的特征值为\( \sigma_k^2 M + \sigma^2 \):
% \[
%     \begin{split}
%         \mathbf{\Sigma}_{\mathbf{X}} \overline{\bm{a}}_{\omega_k}
%          & \approx \left( \overline{\mathbf{A}} \mathbf{\Sigma}_{\mathbf{R}} \mathbf{A}^{\mathrm{T}} + \sigma^2 \mathbf{I} \right) \overline{\bm{a}}_{\omega_k}
%         = \overline{\mathbf{A}} \mathbf{\Sigma}_{\mathbf{R}} \left( \mathbf{A}^{\mathrm{T}} \overline{\bm{a}}_{\omega_k} \right) + \sigma^2 \overline{\bm{a}}_{\omega_k}                            \\
%          & = \overline{\mathbf{A}} \mathbf{\Sigma}_{\mathbf{R}} \begin{bmatrix}
%                                                                     0      \\
%                                                                     \vdots \\
%                                                                     M      \\
%                                                                     \vdots \\
%                                                                     0
%                                                                 \end{bmatrix} + \sigma^2 \overline{\bm{a}}_{\omega_k} = \overline{\mathbf{A}} \begin{bmatrix}
%                                                                                                                                                   0            \\
%                                                                                                                                                   \vdots       \\
%                                                                                                                                                   \sigma_k^2 M \\
%                                                                                                                                                   \vdots       \\
%                                                                                                                                                   0
%                                                                                                                                               \end{bmatrix} + \sigma^2 \overline{\bm{a}}_{\omega_k} \\
%          & = \sigma_k^2 M \overline{\bm{a}}_{\omega_k} + \sigma^2 \overline{\bm{a}}_{\omega_k}
%         = \left( \sigma_k^2 M + \sigma^2 \right) \overline{\bm{a}}_{\omega_k}.
%     \end{split}
% \]
% 一般来说,\( M > K \),因此矩阵\( \mathbf{\Sigma}_{\mathbf{X}} \)至少有\( M-K \)个特征向量不在\( \mathbf{A} \)的列空间内。记这些特征向量为\( \bm{e}_{K+1}, \bm{e}_{K+2}, \ldots, \bm{e}_M \),且都与\( \mathbf{A} \)的列向量的共轭正交,可以验证,这些特征向量对应的特征值均为\( \sigma^2 \):
% \[
%     \begin{split}
%         \mathbf{\Sigma}_{\mathbf{X}} \bm{e}_i
%          & \approx \left( \overline{\mathbf{A}} \mathbf{\Sigma}_{\mathbf{R}} \mathbf{A}^{\mathrm{T}} + \sigma^2 \mathbf{I} \right) \bm{e}_i
%         = \sigma^2 \bm{e}_i.
%     \end{split}
% \]

% 综上,矩阵\( \mathbf{\Sigma}_{\mathbf{X}} \)的特征向量和特征值可以划分为两组:\( \overline{\bm{a}}_{\omega_1}, \overline{\bm{a}}_{\omega_2}, \cdots, \overline{\bm{a}}_{\omega_K} \)对应的特征值为\( \sigma_1^2 M + \sigma^2, \sigma_2^2 M + \sigma^2, \ldots, \sigma_K^2 M + \sigma^2 \);而\( \bm{e}_{K+1}, \bm{e}_{K+2}, \ldots, \bm{e}_M \)对应的特征值均为\( \sigma^2 \)。这两组特征向量张成的子空间分别被称作信号子空间和噪声子空间。而共轭对称矩阵的特征向量必然正交,因此信号子空间与噪声子空间之间同样是互相正交的。

% 此外,注意到信号对应的特征值\( \sigma^2_k M + \sigma^2 > \sigma^2 \),因此可以通过对协方差矩阵进行特征分解,并提取最大的\( K \)个特征值及其对应的特征向量,用来作为信号子空间,剩余的部分作为噪声子空间。具体地,设协方差矩阵的特征分解为
% \[
%     \mathbf{\Sigma}_{\mathbf{X}} = \mathbf{U} \mathbf{\Lambda} \mathbf{U}^{\mathrm{H}} = \begin{bmatrix}
%         \mathbf{U}_{s} & \mathbf{U}_{n}
%     \end{bmatrix} \begin{bmatrix}
%         \mathbf{\Lambda}_{s} & \mathbf{0}           \\
%         \mathbf{0}           & \mathbf{\Lambda}_{n}
%     \end{bmatrix} \begin{bmatrix}
%         \mathbf{U}_{s}^{\mathrm{H}} \\
%         \mathbf{U}_{n}^{\mathrm{H}}
%     \end{bmatrix},
% \]
% 其中,\( \mathbf{U}_{s} \)包含了前\( K \)个最大的特征值对应的特征向量,张成信号子空间;而\( \mathbf{U}_{n} \)包含了剩余的\( M-K \)个特征值对应的特征向量,张成噪声子空间。

Mutiple Signal Classification (MUSIC) 方法正是利用了信号子空间与噪声子空间的正交性,实现对目标参数的高分辨率估计。给定一个导向矢量\( \bm{a}_{\omega} \),如果其对应的参数值\( \omega \)恰好是某个目标的参数值\( \omega_k \),那么该导向矢量必然位于信号子空间内,因此与噪声子空间正交。反之,如果\( \omega \)不是任何目标的参数值,那么导向矢量\( \bm{a}_{\omega} \)将位于噪声子空间内。

进一步地,可以通过将该导向矢量投影到噪声子空间来衡量其与噪声子空间的正交程度,即\( \|\mathbf{U}_{n}^{\mathrm{H}} \overline{\bm{a}}_{\omega}\|_2^2\)。如果该值接近于零,则说明\( \bm{a}_{\omega} \)与噪声子空间正交,进而推断出\( \omega \)可能是某个目标的参数值,反之亦然。基于这一思路,MUSIC 方法定义了如下的谱函数:
\[
    p_{\omega} = \frac{1}{\|\mathbf{U}_{n}^{\mathrm{H}} \overline{\bm{a}}_{\omega}\|_2^2} = \frac{1}{\overline{\bm{a}}_{\omega}^{\mathrm{H}} \mathbf{U}_{n} \mathbf{U}_{n}^{\mathrm{H}} \overline{\bm{a}}_{\omega}}.
\]
遍历所有参数值\( \omega \),即可得到参数维度的功率谱估计。谱函数在目标参数值处会出现尖锐的峰值,从而实现对目标参数的高分辨率估计。

如\cref{fig_music_eg}所示,在强噪声环境下,CBF、Capon 与 MUSIC 三种方法的功率谱估计结果差异显著。可以观察到,MUSIC 方法的谱峰更加尖锐,旁瓣水平更低。此外,由于引入了信号子空间与噪声子空间,MUSIC 方法能够有效抑制噪声,因此其功率谱中的底噪明显低于 CBF 与 Capon,显著提升了目标的可检测性。
\begin{figure}[htb!]
    \centering
    \begin{tikzpicture}
        \begin{axis}[
                xlabel={参数维}, ylabel={归一化功率谱},
                ticklabel style={font=\small},
                xmin=-0.5, xmax=0.5, ymin=0, ymax=1.2,
                % xtick=\empty,ytick=\empty,
                width=8cm, height=4cm,
                label style={font=\small},
                grid,
                legend cell align=left,
                legend style={
                        anchor=north east,
                        font=\tiny,
                        draw=none,
                        fill=none
                    }
            ]
            \addplot[
                c1,
                thick,
            ] table[x=f, y=p1, col sep=comma] {./img/estimation/music_spectrum.csv};
            \addlegendentry{CBF}
            \addplot[
                c2,
                thick,
            ] table[x=f, y=p2, col sep=comma] {./img/estimation/music_spectrum.csv};
            \addlegendentry{Capon}
            \addplot[
                c3,
                thick,
            ] table[x=f, y=p3, col sep=comma] {./img/estimation/music_spectrum.csv};
            \addlegendentry{MUSIC}
        \end{axis}
    \end{tikzpicture}
    \caption{不同波束形成算法的归一化功率谱比较}
    \label{fig_music_eg}
\end{figure}

与 CBF 和 Capon 方法不同,MUSIC 方法的推导过程中并未涉及滤波器权向量的设计,因此难以直接得到如 \cref{fig_compressed_capon} 所示的二维脉压结果。然而,MUSIC 的核心在于利用接收信号矩阵的低秩结构。基于这一点,可以通过对接收信号矩阵进行截断奇异值分解(Truncated SVD),即仅保留前 \(K\) 个奇异值及其对应的奇异向量,对数据进行重构,从而实现信号成分的提取与噪声的抑制。

设接收信号矩阵的奇异值分解为
\[
    \mathbf{X} = \mathbf{U}\mathbf{\Lambda}\mathbf{V}^{\mathrm{H}},
\]
则其协方差矩阵可表示为
\[
    \mathbf{\Sigma}_{\mathbf{X}}
    = \tfrac{1}{L}\mathbf{X}^{\mathrm{H}}\mathbf{X}
    = \tfrac{1}{L}\mathbf{V}\mathbf{\Lambda}^{\mathrm{H}}\mathbf{\Lambda}\mathbf{V}^{\mathrm{H}}.
\]
其中,\(\mathbf{\Lambda}\in\mathbb{R}^{L\times M}\) 为对角矩阵:
\[
    \mathbf{\Lambda} = \begin{bmatrix}
        \lambda_1 &        &           \\
                  & \ddots &           \\
                  &        & \lambda_M \\
        0         & \cdots & 0         \\
        \vdots    & \ddots & \vdots    \\
        0         & \cdots & 0
    \end{bmatrix} \in \mathbb{R}^{L \times M},
\]
从而有
\[
    \mathbf{\Lambda}^{\mathrm{H}} \mathbf{\Lambda} = \begin{bmatrix}
        \lambda_1^2 &             &        &             \\
                    & \lambda_2^2 &        &             \\
                    &             & \ddots &             \\
                    &             &        & \lambda_M^2
    \end{bmatrix} \in \mathbb{R}^{M \times M}.
\]

由此可见,右奇异向量矩阵 \(\mathbf{V}\) 正是协方差矩阵的特征向量,而奇异值的平方 \(\lambda_i^2\) 与协方差矩阵的特征值成正比。因此,前 \(K\) 个最大的奇异值对应于信号子空间,其余则对应于噪声子空间。据此,通过保留前 \(K\) 个奇异值及其对应的奇异向量,可近似提取信号子空间并抑制噪声。降噪后的数据矩阵为
\[
    \hat{\mathbf{X}} = \mathbf{U}_K \mathbf{\Lambda}_K \mathbf{V}_K^{\mathrm{H}},
\]
其中,\(\mathbf{U}_K\) 与 \(\mathbf{V}_K\) 分别由 \(\mathbf{U}\)、\(\mathbf{V}\) 的前 \(K\) 列构成,\(\mathbf{\Lambda}_K\) 为前 \(K\) 个奇异值构成的对角矩阵:
\[
    \mathbf{\Lambda}_K =
    \begin{bmatrix}
        \lambda_1 &           &        &           \\
                  & \lambda_2 &        &           \\
                  &           & \ddots &           \\
                  &           &        & \lambda_K
    \end{bmatrix} \in \mathbb{R}^{K\times K}.
\]

% 与 CBF 和 Capon 方法不同,MUSIC 方法的推导过程中并未涉及滤波器权向量的设计,因此难以直接得到如 \cref{fig_compressed_capon} 所示的二维脉压结果。不过需要指出的是,MUSIC 的核心在于利用接收信号矩阵的低秩结构。基于这一点,可以通过对接收信号矩阵进行截断奇异值分解(Truncated SVD),即仅保留前 \(K\) 个奇异值及其对应的奇异向量,对数据进行重构,从而实现信号成分的提取与噪声的有效抑制。

% 设接收信号矩阵的奇异值分解为
% \[
%     \mathbf{X} = \mathbf{U} \mathbf{\Lambda} \mathbf{V}^{\mathrm{H}}.
% \]
% 此时\( \mathbf{X} \)的协方差矩阵可以表示为
% \[
%     \mathbf{\Sigma}_{\mathbf{X}} = \frac{1}{L} \mathbf{X}^{\mathrm{H}} \mathbf{X} = \frac{1}{L} \mathbf{V} \mathbf{\Lambda}^{\mathrm{H}} \mathbf{U}^{\mathrm{H}} \mathbf{U} \mathbf{\Lambda} \mathbf{V}^{\mathrm{H}} = \frac{1}{L} \mathbf{V} \mathbf{\Lambda}^{\mathrm{H}} \mathbf{\Lambda} \mathbf{V}^{\mathrm{H}}.
% \]
% 注意到,矩阵\( \mathbf{\Lambda} \in \mathbb{C}^{L \times M} \)是一个对角矩阵,有如下形式:
% \[
%     \mathbf{\Lambda} = \begin{bmatrix}
%         \lambda_1 &           &        &           \\
%                   & \lambda_2 &        &           \\
%                   &           & \ddots &           \\
%                   &           &        & \lambda_M \\
%         0         & 0         & \cdots & 0         \\
%         \vdots    & \vdots    & \ddots & \vdots    \\
%         0         & 0         & \cdots & 0
%     \end{bmatrix} \in \mathbb{R}^{L \times M},
% \]
% 因此,有
% \[
%     \mathbf{\Lambda}^{\mathrm{H}} \mathbf{\Lambda} = \begin{bmatrix}
%         \lambda_1^2 &             &        &             \\
%                     & \lambda_2^2 &        &             \\
%                     &             & \ddots &             \\
%                     &             &        & \lambda_M^2
%     \end{bmatrix} \in \mathbb{R}^{M \times M}.
% \]

% 不难发现,SVD分解的右奇异向量\( \mathbf{V} \)即为协方差矩阵的特征向量,而奇异值的平方\( \lambda_i^2 \)与协方差矩阵的特征值成正比。因此,前\( K \)个最大的奇异值同样对应于信号子空间,而剩余的奇异值对应于噪声子空间。

% 因此,可以通过保留前\( K \)个最大的奇异值及其对应的奇异向量,来近似地提取信号子空间,从而实现对信号成分的提取与噪声的抑制。具体地,降噪后的数据矩阵有如下形式:
% \[
%     \hat{\mathbf{X}} = \mathbf{U}_K \mathbf{\Lambda}_K \mathbf{V}_K^{\mathrm{H}},
% \]
% 其中\( \mathbf{U}_K \)为\( \mathbf{U} \)的前\( K \)列,\( \mathbf{V}_K \)为\( \mathbf{V} \)的前\( K \)列,\( \mathbf{\Lambda}_K \)为前\( K \)个奇异值构成的对角矩阵:
% \[
%     \mathbf{\Lambda}_K = \begin{bmatrix}
%         \lambda_1 &           &        &           \\
%                   & \lambda_2 &        &           \\
%                   &           & \ddots &           \\
%                   &           &        & \lambda_K
%     \end{bmatrix} \in \mathbb{R}^{K \times K}
% \]

\begin{example}
    对\cref{fig_noise_music}中的包含噪声的数据进行SVD分解,并保留前\( K=2 \)个奇异值及其对应的奇异向量,计算降噪后的数据矩阵\( \hat{\mathbf{X}} \)。
    \begin{figure}[htb!]
        \centering
        \begin{tikzpicture}
            \begin{axis}[
                    xlabel={参数维},
                    ylabel={时间维},
                    enlargelimits=false,
                    width=5cm, height=5cm,
                    ytick=\empty,
                    xtick=\empty,
                    ticklabel style={font=\small},
                    label style={font=\small},
                    axis on top
                ]
                \addplot graphics [
                        xmin=-1, xmax=1, ymin=-1, ymax=1,
                    ] {./img/estimation/music_1.png};
            \end{axis}
        \end{tikzpicture}
        \caption{有噪声数据矩阵\( \mathbf{X} \)(仅绘制了实部)}
        \label{fig_noise_music}
    \end{figure}
\end{example}
\begin{solution}
    降噪后的数据矩阵如\cref{fig_denoised}所示。可以看到,经过截断奇异值分解后,数据矩阵中的噪声成分被有效抑制,目标信号得以更清晰地呈现。
    \begin{figure}[htb!]
        \centering
        \begin{tikzpicture}
            \begin{axis}[
                    xlabel={参数维},
                    ylabel={时间维},
                    enlargelimits=false,
                    width=5cm, height=5cm,
                    ytick=\empty,
                    xtick=\empty,
                    ticklabel style={font=\small},
                    label style={font=\small},
                    axis on top
                ]
                \addplot graphics [
                        xmin=-1, xmax=1, ymin=-1, ymax=1,
                    ] {./img/estimation/music_2.png};
            \end{axis}
        \end{tikzpicture}
        \caption{降噪后的数据矩阵\( \hat{\mathbf{X}} \)(仅绘制了实部)}
        \label{fig_denoised}
    \end{figure}

    可以对降噪后的数据矩阵先后采用匹配滤波和 Capon 算法进行二维脉压处理,从而获得更清晰的目标回波图像。需要注意的是,此时接收信号矩阵的协方差矩阵已不再满秩,因此在实现 Capon 算法时应引入对角加载技术。各算法处理结果的幅度图如\cref{fig_compressed_music}所示。由图可见,经过降噪处理后,目标信号得到更清晰的展现,旁瓣水平显著降低,从而有效提升了目标的分辨能力与检测性能。

    \begin{figure}[htb!]
        \centering
        \begin{subfigure}{.32\textwidth}
            \centering
            \begin{tikzpicture}
                \begin{axis}[
                        xlabel={参数维},
                        ylabel={时间维},
                        enlargelimits=false,
                        width=5cm, height=5cm,
                        ytick=\empty,
                        xtick=\empty,
                        ticklabel style={font=\small},
                        label style={font=\small},
                        axis on top
                    ]
                    \addplot graphics [
                            xmin=-1, xmax=1, ymin=-1, ymax=1,
                        ] {./img/estimation/music_3.png};
                \end{axis}
            \end{tikzpicture}
            \caption{CBF}
            \label{fig_2d_compress_denoise_1}
        \end{subfigure}
        \begin{subfigure}{.32\textwidth}
            \centering
            \begin{tikzpicture}
                \begin{axis}[
                        xlabel={参数维},
                        ylabel={时间维},
                        enlargelimits=false,
                        width=5cm, height=5cm,
                        ytick=\empty,
                        xtick=\empty,
                        ticklabel style={font=\small},
                        label style={font=\small},
                        axis on top
                    ]
                    \addplot graphics [
                            xmin=-1, xmax=1, ymin=-1, ymax=1,
                        ] {./img/estimation/music_4.png};
                \end{axis}
            \end{tikzpicture}
            \caption{Capon}
            \label{fig_2d_compress_denoise_2}
        \end{subfigure}
        \begin{subfigure}{.32\textwidth}
            \centering
            \begin{tikzpicture}
                \begin{axis}[
                        xlabel={参数维},
                        ylabel={时间维},
                        enlargelimits=false,
                        width=5cm, height=5cm,
                        ytick=\empty,
                        xtick=\empty,
                        ticklabel style={font=\small},
                        label style={font=\small},
                        axis on top
                    ]
                    \addplot graphics [
                            xmin=-1, xmax=1, ymin=-1, ymax=1,
                        ] {./img/estimation/music_5.png};
                \end{axis}
            \end{tikzpicture}
            \caption{SVD降噪}
            \label{fig_2d_compress_denoise_3}
        \end{subfigure}
        \caption{不同方法处理后的数据矩阵的幅度图}
        \label{fig_compressed_music}
    \end{figure}
\end{solution}

接下来,我们定性地讨论一下 MUSIC 方法为何优于 Capon 方法。注意到,Capon 方法的功率谱输出为
\[
    p_{\omega}^{\text{Capon}} = \frac{1}{\overline{\bm{a}}_{\omega}^{\mathrm{H}} \mathbf{\Sigma}_{\mathbf{X}}^{-1} \overline{\bm{a}}_{\omega}}.
\]
又因为
\[
    \mathbf{\Sigma}_{\mathbf{X}}
    = \mathbf{U} \mathbf{\Lambda} \mathbf{U}^{\mathrm{H}},
\]
其中
{
\tiny
\[
    \mathbf{U} \approx \frac{1}{\sqrt{M}}\begin{bmatrix}
        \overline{\bm{a}}_1 & \cdots & \overline{\bm{a}}_K & \bm{e}_{K+1} & \cdots & \bm{e}_M
    \end{bmatrix}, \quad
    \mathbf{\Lambda} \approx \begin{bmatrix}
        \sigma_1^2 M + \sigma^2 &        &                         &          &        &          \\
                                & \ddots &                         &          &        &          \\
                                &        & \sigma_K^2 M + \sigma^2 &          &        &          \\
                                &        &                         & \sigma^2 &        &          \\
                                &        &                         &          & \ddots &          \\
                                &        &                         &          &        & \sigma^2
    \end{bmatrix}.
\]
}
基于SVD分解,可以得到\( \mathbf{\Sigma}_{\mathbf{X}} \) 的逆为
\[
    \mathbf{\Sigma}_{\mathbf{X}}^{-1} \approx \mathbf{U} \mathbf{\Lambda}^{-1} \mathbf{U}^{\mathrm{H}}.
\]

因此,对于目标参数 \(\overline{\bm{a}}_{\omega_k}\),Capon 方法的谱值为
\[
    p_{\omega_k}^{\text{Capon}}
    \approx \frac{1}{\tfrac{M}{\sigma_k^2 M + \sigma^2}}
    = \frac{\sigma_k^2 M + \sigma^2}{M},
\]
而对于非目标参数 \(\omega \neq \omega_k\),有
\[
    p_{\omega}^{\text{Capon}}
    \approx \frac{1}{\tfrac{M}{\sigma^2}}
    = \frac{\sigma^2}{M}.
\]
由此可见,Capon 谱峰与底噪的比值为
\[
    \frac{p_{\omega_k}^{\text{Capon}}}{p_{\omega}^{\text{Capon}}}
    \approx \frac{\sigma_k^2 M}{\sigma^2} + 1.
\]

对于 MUSIC 方法,其谱函数定义为
\[
    p_{\omega}^{\text{MUSIC}}
    = \frac{1}{\overline{\bm{a}}_{\omega}^{\mathrm{H}}
    \mathbf{U}_n \mathbf{U}_n^{\mathrm{H}}
    \overline{\bm{a}}_{\omega}}.
\]
注意到
\[
    \mathbf{U}_n \mathbf{U}_n^{\mathrm{H}}
    = \mathbf{U}\,\tilde{\mathbf{\Lambda}}\,\mathbf{U}^{\mathrm{H}}, \qquad
    \tilde{\mathbf{\Lambda}}
    = \operatorname{diag}(\underbrace{0,\dots,0}_{K},
    \underbrace{1,\dots,1}_{M-K}),
\]
因此,对于目标参数 \(\overline{\bm{a}}_{\omega_k}\),分母为零,
\[
    p_{\omega_k}^{\text{MUSIC}} \approx +\infty,
\]
而对于非目标参数值 \(\omega \neq \omega_k\),有
\[
    p_{\omega}^{\text{MUSIC}} \approx \frac{1}{M}.
\]
故 MUSIC 谱峰与底噪之比趋于无穷大,其谱峰显著尖锐。

综上,Capon 与 MUSIC 的谱函数可统一表示为
\[
    p_{\omega} = \frac{1}{\overline{\bm{a}}_{\omega}^{\mathrm{H}}
        \mathbf{W}\,\overline{\bm{a}}_{\omega}},
\]
其中 Capon 方法对应 \(\mathbf{W} = \mathbf{\Sigma}_{\mathbf{X}}^{-1}\),即对信号与噪声赋予与功率成反比的加权;而 MUSIC 方法对应 \(\mathbf{W} = \mathbf{U}_n \mathbf{U}_n^{\mathrm{H}}\),是噪声子空间的投影矩阵,其权重仅取 \(0/1\),本质上是一种非线性划分。正是这种权重结构的差异,使得 MUSIC 能够实现比 Capon 更高的分辨率。

\subsection{ESPRIT方法}

\section{稀疏表示方法}

\section{多域联合估计}

\subsection{空时自适应处理}

\subsection{张量分解方法}
