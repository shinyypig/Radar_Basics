\chapter{无源定位}

雷达是一种主动探测设备,通过发射电磁波,并接收目标反射的回波来确定目标的位置和速度。但在真实战场中,这是一种很危险的行为,极容易暴露自身位置,从而遭到敌方的打击。因此,无源定位技术应运而生。无源定位是指通过被动接收目标发射的电磁波信号,来获取目标的相关信息。常见的无源定位技术有测向定位、时差定位和频差定位等。无源定位技术的优点在于隐蔽性强,难以被敌方发现,因此在现代战争中得到了广泛应用。

\section{测向定位}
\subsection{二维平面}
利用机械扫描,或者电子扫描等方式,雷达基站可以获得所接收的电子的方位信息。本节以二维平面为例,显然,一个雷达基站确定了一条过目标以及雷达基站自身的直线。如图\cref{fig_locate_example}所示,在二维平面上,至少二个雷达基站可以确定目标位置。
\begin{figure}[htb!]
    \centering
    \includegraphics[width=0.5\textwidth]{img/locate/2d_example.tikz}
    \caption{测向定位示意图}
    \label{fig_locate_example}
\end{figure}

但在实际应用中,由于会受到噪声和干扰的影响,雷达基站获得的目标角度信息往往存在误差,为了提高定位精度,通常需要多个基站进行联合定位。但此时一个新的问题出现了,由于误差的存在,多个基站所确定的直线并不一定相交于同一点(如\cref{fig_locate_error}所示),此时目标的位置该如何确定呢?
\begin{figure}[htb!]
    \centering
    \includegraphics[width=0.5\textwidth]{img/locate/2d_err.tikz}
    \caption{无源定位可能存在观测误差}
    \label{fig_locate_error}
\end{figure}

设有\( N \)个基站,且第\( n \)个基站确定的过目标的直线方程为\( a_{n1} x_1 + a_{n2} x_2 = b_n \),则\( N \)个基站确定了如下的方程组:
\[
    \begin{cases}
        a_{11} x_1 + a_{12} x_2 = b_1 \\
        a_{21} x_1 + a_{22} x_2 = b_2 \\
        \vdots                        \\
        a_{N1} x_1 + a_{N2} x_2 = b_N
    \end{cases}.
\]
由于存在误差,以上方程组并不一定有解。但我们希望最终求解得到的目标位置\( (x_1, x_2) \)尽可能满足所有直线方程。此时,新的问题出现了,我们该如何衡量解满足方程的程度呢?在测向定位中,这个指标很容易就可以得到了,即目标到某个基站所确定的直线的距离。根据点到直线的距离公式,目标到第\( n \)个基站所确定的直线的距离为:
\[
    d_n = \frac{|a_{n1} x + a_{n2} y - b_n|}{\sqrt{a_{n1}^2 + a_{n2}^2}}.
\]
因此,可以得到如下的目标函数:
\[
    \min_{x_1, x_2} \sum_{n=1}^{N} d_n^2 = \sum_{n=1}^{N} \frac{(a_{n1} x + a_{n2} y - b_n)^2}{a_{n1}^2 + a_{n2}^2}.
\]
注意到,比起最小化距离和,我们更倾向于最小化距离的平方和,这样可以避免绝对值带来的分段函数。

此外,目标到第\( n \)个基站所确定的直线的距离表达式中存在一个带根号的分母,这不便于计算;而对于一条直线方程,等式两边同除一个非零的数并不会改变这条直线。因此,我们可以通过归一化的方式,将分母变为1,从而简化计算。具体地,我们可以将第\( n \)个基站所确定的直线方程两边同除以\( \sqrt{a_n^2 + b_n^2} \),从而得到新的方程组:
\[
    \begin{cases}
        \frac{a_{11}}{\sqrt{a_{11}^2 + a_{12}^2}} x + \frac{a_{12}}{\sqrt{a_{11}^2 + a_{12}^2}} y = \frac{b_1}{\sqrt{a_{11}^2 + a_{12}^2}} \\
        \frac{a_{21}}{\sqrt{a_{21}^2 + a_{22}^2}} x + \frac{a_{22}}{\sqrt{a_{21}^2 + a_{22}^2}} y = \frac{b_2}{\sqrt{a_{21}^2 + a_{22}^2}} \\
        \vdots                                                                                                                             \\
        \frac{a_{N1}}{\sqrt{a_{N1}^2 + a_{N2}^2}} x + \frac{a_{N2}}{\sqrt{a_{N1}^2 + a_{N2}^2}} y = \frac{b_N}{\sqrt{a_{N1}^2 + a_{N2}^2}}
    \end{cases}.
\]
记
\[
    \begin{cases}
        \hat{a}_{n1} = \frac{a_{n1}}{\sqrt{a_{n1}^2 + a_{n2}^2}} \\
        \hat{a}_{n2} = \frac{a_{n2}}{\sqrt{a_{n1}^2 + a_{n2}^2}} \\
        \hat{b}_n = \frac{b_n}{\sqrt{a_{n1}^2 + a_{n2}^2}}       \\
    \end{cases},
\]
则目标到第\( n \)个基站所确定的直线的距离可以简化为:
\[
    d_n = |\hat{a}_{n1} x + \hat{a}_{n2} y - \hat{b}_n|.
\]

为了简化计算,我们可以将方程组写成矩阵的形式:
\[
    \begin{bmatrix}
        \hat{a}_{11} & \hat{a}_{12} \\
        \hat{a}_{21} & \hat{a}_{22} \\
        \vdots       & \vdots       \\
        \hat{a}_{N1} & \hat{a}_{N2}
    \end{bmatrix}
    \begin{bmatrix}
        x_1 \\
        x_2
    \end{bmatrix}
    =
    \begin{bmatrix}
        \hat{b}_1 \\
        \hat{b}_2 \\
        \vdots    \\
        \hat{b}_N
    \end{bmatrix}
\]
也即\( \mathbf{A} \bm{x} = \bm{b} \)。此时目标到所有基站确定的直线的距离构成的向量\( \bm{d} = \begin{bsmallmatrix} d_1 & d_2 & \cdots & d_N \end{bsmallmatrix}^{\mathrm{T}} \)有如下的表达式:
\[
    d_n = \left| (\mathbf{A} \bm{x} - \bm{b})_n \right|.
\]
而目标函数则可以简化为:
\[
    \min_{\bm{x}} \left\| \mathbf{A} \bm{x} - \bm{b} \right\|^2.
\]
对于该目标函数,我们有如下定理:
\begin{theorem}
    \( f(\bm{x})  = \left\| \mathbf{A} \bm{x} - \bm{b} \right\|^2 \)为一个凸函数。
\end{theorem}
\begin{proof}
    根据定义,函数\( f(\bm{x}) \)为凸函数当且仅当对于任意的\( \bm{x}_1, \bm{x}_2 \in \mathbb{R}^n \),都有
    \[
        \frac{f(\bm{x}_1) + f(\bm{x}_2)}{2} \geq f\left(\frac{\bm{x}_1 + \bm{x}_2}{2}\right).
    \]

    将目标函数\( f(x) \)展开,有
    \[
        f(\bm{x}) = \left\| \mathbf{A} \bm{x} - \bm{b} \right\|^2 = (\mathbf{A} \bm{x} - \bm{b})^{\mathrm{T}} (\mathbf{A} \bm{x} - \bm{b}) = \bm{x}^{\mathrm{T}} \mathbf{A}^{\mathrm{T}} \mathbf{A} \bm{x} - 2 \bm{b}^{\mathrm{T}} \mathbf{A} \bm{x} + \bm{b}^{\mathrm{T}} \bm{b}.
    \]
    因此,\( f\left(\frac{\bm{x}_1 + \bm{x}_2}{2}\right) \) 可以展开为:
    \[
        f\left(\frac{\bm{x}_1 + \bm{x}_2}{2}\right) = \frac{1}{4} \bm{x}_1^{\mathrm{T}} \mathbf{A}^{\mathrm{T}} \mathbf{A} \bm{x}_1 + \frac{1}{4} \bm{x}_2^{\mathrm{T}} \mathbf{A}^{\mathrm{T}} \mathbf{A} \bm{x}_2 + \frac{1}{2} \bm{x}_1^{\mathrm{T}} \mathbf{A}^{\mathrm{T}} \mathbf{A} \bm{x}_2 - \bm{b}^{\mathrm{T}} \mathbf{A} (\bm{x}_1 + \bm{x}_2) + \bm{b}^{\mathrm{T}} \bm{b}.
    \]
    与此同时,\( \frac{f(\bm{x}_1) + f(\bm{x}_2)}{2} \) 可以展开为:
    \[
        \frac{f(\bm{x}_1) + f(\bm{x}_2)}{2} = \frac{1}{2} \bm{x}_1^{\mathrm{T}} \mathbf{A}^{\mathrm{T}} \mathbf{A} \bm{x}_1 + \frac{1}{2} \bm{x}_2^{\mathrm{T}} \mathbf{A}^{\mathrm{T}} \mathbf{A} \bm{x}_2 - \bm{b}^{\mathrm{T}} \mathbf{A} (\bm{x}_1 + \bm{x}_2) + \bm{b}^{\mathrm{T}} \bm{b}.
    \]
    所以,只需证明
    \[
        \begin{split}
            \frac{f(\bm{x}_1) + f(\bm{x}_2)}{2} - f\left(\frac{\bm{x}_1 + \bm{x}_2}{2}\right) & = \frac{1}{4} \bm{x}_1^{\mathrm{T}} \mathbf{A}^{\mathrm{T}} \mathbf{A} \bm{x}_1 + \frac{1}{4} \bm{x}_2^{\mathrm{T}} \mathbf{A}^{\mathrm{T}} \mathbf{A} \bm{x}_2 - \frac{1}{2} \bm{x}_1^{\mathrm{T}} \mathbf{A}^{\mathrm{T}} \mathbf{A} \bm{x}_2 \\
                                                                                              & = \frac{1}{4} (\bm{x}_1 - \bm{x}_2)^{\mathrm{T}} \mathbf{A}^{\mathrm{T}} \mathbf{A} (\bm{x}_1 - \bm{x}_2) \geq 0.
        \end{split}
    \]
    记\( \bm{v} = \mathbf{A} (\bm{x}_1 - \bm{x}_2) \),显然有
    \[
        \bm{v}^{\mathrm{T}} \bm{v} = (\bm{x}_1 - \bm{x}_2)^{\mathrm{T}} \mathbf{A}^{\mathrm{T}} \mathbf{A} (\bm{x}_1 - \bm{x}_2) \geq 0.
    \]
    因此,\( f(\bm{x}) \)为一个凸函数。
\end{proof}

对于一个凸函数,我们可以通过求解其导数为0的点来找到最优解。具体而言,对于目标函数\( f(\bm{x}) \),我们有
\[
    \frac{\partial f(\bm{x})}{\partial \bm{x}} = 2 \mathbf{A}^{\mathrm{T}} (\mathbf{A} \bm{x} - \bm{b}) = 0.
\]
因此,最优解为
\[
    \bm{x} = \left( \mathbf{A}^{\mathrm{T}} \mathbf{A} \right)^{-1} \mathbf{A}^{\mathrm{T}} \bm{b}.
\]
这就是著名的最小二乘解。

\begin{example}
    设二维平面上有三个基站,确定了如下的方程组:
    \[
        \begin{cases}
            2 x + y = 4  \\
            2x - 3y = -4 \\
            1.9x - 1.1 y = 0
        \end{cases}.
    \]
    请求解目标位置\( (x, y) \)。
\end{example}
\begin{solution}
    首先将直线方程归一化,得到
    \[
        \begin{cases}
            \frac{2}{\sqrt{5}} x + \frac{1}{\sqrt{5}} y = \frac{4}{\sqrt{5}}     \\
            \frac{2}{\sqrt{13}} x - \frac{3}{\sqrt{13}} y = -\frac{4}{\sqrt{13}} \\
            \frac{1.9}{\sqrt{4.82}} x - \frac{1.1}{\sqrt{4.82}} y = 0
        \end{cases}.
    \]
    构建对应的矩阵和向量:
    \[
        \mathbf{A} =
        \begin{bmatrix}
            \frac{2}{\sqrt{5}}      & \frac{1}{\sqrt{5}}       \\
            \frac{2}{\sqrt{13}}     & -\frac{3}{\sqrt{13}}     \\
            \frac{1.9}{\sqrt{4.82}} & -\frac{1.1}{\sqrt{4.82}}
        \end{bmatrix}, \quad
        \bm{b} =
        \begin{bmatrix}
            \frac{4}{\sqrt{5}}   \\
            -\frac{4}{\sqrt{13}} \\
            0
        \end{bmatrix}.
    \]
    代入最小二乘解公式,得到
    \[
        \bm{x} = \left( \mathbf{A}^{\mathrm{T}} \mathbf{A} \right)^{-1} \mathbf{A}^{\mathrm{T}} \bm{b} =
        \begin{bmatrix}
            1.0540 \\
            1.9635
        \end{bmatrix}.
    \]
    从\cref{fig_locate_example_func}可以看到,最小二乘得到的解正位于目标函数的最小值处。这意味该点到所有基站确定的直线的距离的平方和和最小,也就是目标最有可能出现的位置。
    \begin{figure}[htb!]
        \centering
        \begin{tikzpicture}[scale=0.8]
            \begin{axis}[
                    % xlabel=$ x $, ylabel=$ y $,
                    ticklabel style={font=\tiny},
                    label style={font=\tiny},
                    axis equal image,
                    xmin=0.75, xmax=1.35,
                    ymin=1.7, ymax=2.3,
                    view={0}{90},
                    legend cell align=left,
                    legend style={
                            anchor=north east,
                            font=\tiny,
                            draw=none,
                            fill=none
                        },
                ]
                \coordinate (X1) at (axis cs:2, 0);
                \coordinate (X2) at (axis cs:-2, 0);
                \coordinate (X3) at (axis cs:0, 0);
                \coordinate (T) at (axis cs:1, 2);
                \coordinate (T1) at (axis cs:1.1, 1.9);
                \coordinate (T2) at (axis cs:1.0540, 1.9635);

                \addplot3[domain=0.75:1.35, domain y=1.7:2.3, surf, opacity=0.5, samples=30] {58169/31330 * x^2 - 15513*x*y/15665 - 128*x/65 + 35821 * y^2 / 31330 - 224 * y / 65 + 288/65};

                \draw[thick, dashed] ($(X1)!0.5!(T)$) -- ($(X1)!1.5!(T)$);
                \draw[thick, dashed] ($(X2)!0.5!(T)$) -- ($(X2)!1.5!(T)$);
                \draw[thick, dashed] ($(X3)!0.5!(T1)$) -- ($(X3)!1.5!(T1)$);

                \draw[c2, fill=c2!60] (T2) circle (4pt);
            \end{axis}
        \end{tikzpicture}
        \caption{测向定位目标函数}
        \label{fig_locate_example_func}
    \end{figure}
\end{solution}

\subsection{三维空间}

在实际应用中,我们更多地需要确定目标在三维空间中的位置,此时又该如何求解呢?注意到,在三维空间中,雷达基站并不是直接确定一条过目标的直线,而是分别确定目标所在的方位角和俯仰角,也就是确定了两个过目标的平面。因此,类似于二维平面中的情况,我们可以通过至少两个基站确定的四个平面来求解目标的位置。

设共确定了\( N \)个平面,对应的方程组为:
\[
    \begin{cases}
        a_{11} x + a_{12} y + a_{13} z = b_1 \\
        a_{21} x + a_{22} y + a_{23} z = b_2 \\
        \vdots                               \\
        a_{N1} x + a_{N2} y + a_{N3} z = b_N
    \end{cases}.
\]
同样,我们可以将方程组归一化,得到
\[
    \begin{cases}
        \frac{a_{11}}{\sqrt{a_{11}^2 + a_{12}^2 + a_{13}^2}} x + \frac{a_{12}}{\sqrt{a_{11}^2 + a_{12}^2 + a_{13}^2}} y + \frac{a_{13}}{\sqrt{a_{11}^2 + a_{12}^2 + a_{13}^2}} z = \frac{b_1}{\sqrt{a_{11}^2 + a_{12}^2 + a_{13}^2}} \\
        \frac{a_{21}}{\sqrt{a_{21}^2 + a_{22}^2 + a_{23}^2}} x + \frac{a_{22}}{\sqrt{a_{21}^2 + a_{22}^2 + a_{23}^2}} y + \frac{a_{23}}{\sqrt{a_{21}^2 + a_{22}^2 + a_{23}^2}} z = \frac{b_2}{\sqrt{a_{21}^2 + a_{22}^2 + a_{23}^2}} \\
        \vdots                                                                                                                                                                                                                       \\
        \frac{a_{N1}}{\sqrt{a_{N1}^2 + a_{N2}^2 + a_{N3}^2}} x + \frac{a_{N2}}{\sqrt{a_{N1}^2 + a_{N2}^2 + a_{N3}^2}} y + \frac{a_{N3}}{\sqrt{a_{N1}^2 + a_{N2}^2 + a_{N3}^2}} z = \frac{b_N}{\sqrt{a_{N1}^2 + a_{N2}^2 + a_{N3}^2}}
    \end{cases}.
\]
记
\[
    \begin{cases}
        \hat{a}_{n1} = \frac{a_{n1}}{\sqrt{a_{n1}^2 + a_{n2}^2 + a_{n3}^2}} \\
        \hat{a}_{n2} = \frac{a_{n2}}{\sqrt{a_{n1}^2 + a_{n2}^2 + a_{n3}^2}} \\
        \hat{a}_{n3} = \frac{a_{n3}}{\sqrt{a_{n1}^2 + a_{n2}^2 + a_{n3}^2}} \\
        \hat{b}_n = \frac{b_n}{\sqrt{a_{n1}^2 + a_{n2}^2 + a_{n3}^2}}       \\
    \end{cases},
\]
则目标到第\( n \)个基站所确定的平面的距离可以简化为:
\[
    d_n = |\hat{a}_{n1} x + \hat{a}_{n2} y + \hat{a}_{n3} z - \hat{b}_n|.
\]
后面的求解过程与二维平面中的情况相同,这里不再赘述。

当然,我们也可以直接使用三维空间中的直线方程来求解目标位置。设有\( N \)个基站,且第\( n \)个基站确定的过目标的直线方程参数方程为
\[
    \begin{cases}
        x_1 = l_{n1} + k_{n1} t \\
        x_2 = l_{n2} + k_{n2} t \\
        x_3 = l_{n3} + k_{n3} t
    \end{cases}.
\]
记
\[
    \bm{x} =
    \begin{bmatrix}
        x_1 \\
        x_2 \\
        x_3
    \end{bmatrix}, \quad
    \bm{l}_n =
    \begin{bmatrix}
        l_{n1} \\
        l_{n2} \\
        l_{n3}
    \end{bmatrix}, \quad
    \bm{k}_n =
    \begin{bmatrix}
        k_{n1} \\
        k_{n2} \\
        k_{n3}
    \end{bmatrix},
\]
则第\( n \)个基站确定的直线方程可以简化为
\[
    \bm{x} = \bm{l}_n + \bm{k}_n t.
\]
注意到,\( \bm{k}_n \)代表了直线的方向,因此不妨令其模长为1,即\(\bm{k}_n^{\mathrm{T}}\bm{k}_n = 1 \)。

设目标的位置为\( \bm{w} \),则目标到第\( n \)个基站所确定的直线的距离平方为
\[
    d_n^2 = \min_{t} \left\| \bm{w} - \bm{l}_n - \bm{k}_n t \right\|^2.
\]
因此,我们可以得到如下的目标函数:
\[
    \min_{\bm{w}} \sum_{n=1}^{N} d_n^2.
\]

首先,让我们来求解目标到第\( n \)个基站所确定的直线的距离平方。首先将对应的目标函数展开:
\[
    \begin{split}
        d_n^2 & = \min_{t} \left\| \bm{w} - \bm{l}_n - \bm{k}_n t \right\|^2                                                                                            \\
              & = \min_{t} (\bm{w} - \bm{l}_n - \bm{k}_n t)^{\mathrm{T}} (\bm{w} - \bm{l}_n - \bm{k}_n t)                                                               \\
              & = \min_{t} t^2 (\bm{k}_n^{\mathrm{T}} \bm{k}_n)  - 2 t (\bm{w} - \bm{l}_n)^{\mathrm{T}}\bm{k}_n + (\bm{w} - \bm{l}_n)^{\mathrm{T}} (\bm{w} - \bm{l}_n).
    \end{split}
\]
该函数为一个关于\( t \)的二次函数,显然其最小值出现在
\[
    t = \frac{(\bm{w} - \bm{l}_n)^{\mathrm{T}} \bm{k}_n}{\bm{k}_n^{\mathrm{T}} \bm{k}_n} = (\bm{w} - \bm{l}_n)^{\mathrm{T}} \bm{k}_n.
\]
将其代入到距离的平方公式中,得到
\[
    \begin{split}
        d_n^2 & = \left\|\bm{w} - \bm{l}_n - \bm{k}_n(\bm{w} - \bm{l}_n)^{\mathrm{T}} \bm{k}_n\right\|^2                                                                                               \\
              & = \left\| \bm{w} - \bm{l}_n - \bm{k}_n \bm{k}_n^{\mathrm{T}} (\bm{w} - \bm{l}_n) \right\|^2                                                                                            \\
              & = \left\| \left(\mathbf{I} - \bm{k}_n \bm{k}_n^{\mathrm{T}}\right) ( \bm{w} - \bm{l}_n) \right\|^2                                                                                     \\
              & = (\bm{w} - \bm{l}_n)^{\mathrm{T}} \left(\mathbf{I} - \bm{k}_n \bm{k}_n^{\mathrm{T}}\right)^{\mathrm{T}} \left(\mathbf{I} - \bm{k}_n \bm{k}_n^{\mathrm{T}}\right) ( \bm{w} - \bm{l}_n) \\
              & = (\bm{w} - \bm{l}_n)^{\mathrm{T}} \left(\mathbf{I} - \bm{k}_n \bm{k}_n^{\mathrm{T}}\right) ( \bm{w} - \bm{l}_n)                                                                       \\
    \end{split}
\]
记\( \mathbf{P}_n = \mathbf{I} - \bm{k}_n \bm{k}_n^{\mathrm{T}} \)因此,目标函数可以简化为
\[
    \begin{split}
        \min_{\bm{w}} \sum_{n=1}^{N} d_n^2 & = \min_{\bm{w}} \sum_{n=1}^{N} (\bm{w} - \bm{l}_n)^{\mathrm{T}} \mathbf{P}_n ( \bm{w} - \bm{l}_n)                                                                                                                           \\
                                           & = \min_{\bm{w}} \bm{w}^{\mathrm{T}} \left(\sum_{n=1}^{N} \mathbf{P}_n\right) \bm{w} - 2 \sum_{n=1}^{N} \left(\bm{l}_n^{\mathrm{T}} \mathbf{P}_n\right) \bm{w} + \sum_{n=1}^{N} \bm{l}_n^{\mathrm{T}} \mathbf{P}_n \bm{l}_n.
    \end{split}
\]
记
\[
    \mathbf{P} = \sum_{n=1}^{N} \mathbf{P}_n, \quad
    \bm{l} = \left(\sum_{n=1}^{N} \bm{l}_n^{\mathrm{T}} \mathbf{P}_n\right)^{\mathrm{T}},
\]
则目标函数等价于
\[
    \min_{\bm{w}} \bm{w}^{\mathrm{T}} \mathbf{P} \bm{w} - 2 \bm{l}^{\mathrm{T}} \bm{w}.
\]
其关于\( \bm{w} \)的导数为
\[
    \frac{\partial f(\bm{w})}{\partial \bm{w}} = 2 \mathbf{P} \bm{w} - 2 \bm{l}.
\]
因此,最优解为
\[
    \bm{w} = \mathbf{P}^{-1} \bm{l}.
\]

\section{时差定位}

利用目标发射的电磁波抵达不同位置的基站的时间不同,也可以确定目标的位置。设有\( 2 \)个基站,其中基站1和基站2分别于\( t_1 \)和\( t_2 \)时刻接收到目标发射的电磁波信号,那么目标到两个基站之间的距离差为
\[
    d = c (t_2 - t_1),
\]
其中\( c \)为光速。根据几何知识,我们知道与两个固定的点(称为焦点)的距离差是常数的点的轨迹为双曲线,如\cref{fig_hyerbola}所示。

\begin{figure}[htb!]
    \centering
    \includegraphics[width=.35\textwidth]{./img/locate/tdoa_1.tikz}
    \caption{双曲线示例}
    \label{fig_hyerbola}
\end{figure}

显然,只有两个基站时,只能确定目标所在的双曲线。不过,可以通过增加一个基站,通过求解两个双曲线的交点\footnote{实际上是三个双曲线,并且这三个双曲线必然交于一点}来确定目标的位置,如\cref{fig_tdoa}所示。

\begin{figure}[htb!]
    \centering
    \includegraphics[width=.35\textwidth]{./img/locate/tdoa_2.tikz}
    \caption{时差定位示例}
    \label{fig_tdoa}
\end{figure}

实际应用中,时差的测量同样也会受到噪声和干扰的影响,因此我们可以通过增加基站的数量来提高定位精度。此时,多个基站确定的双曲线并不一定交于一点。设有\( N \)个基站,且第\( i \)个基站接收到目标发射的电磁波信号的时刻为\( t_i \),则目标到第\( i \)个基站和第\( j \)个基站的距离差为
\[
    d_{ij} = c (t_j - t_i).
\]
设目标位置为\( \bm{x} = \begin{bsmallmatrix} x_1 & x_2 \end{bsmallmatrix}^{\mathrm{T}} \),第\( i \)个基站的位置为\( \bm{l}_i = \begin{bsmallmatrix} l_{1i} & l_{2i} \end{bsmallmatrix}^{\mathrm{T}} \),则目标到第\( i \)个基站的距离为
\[
    d_i = \left\| \bm{x} - \bm{l}_i \right\| = \sqrt{(\bm{x} - \bm{l}_i)^{\mathrm{T}} (\bm{x} - \bm{l}_i)} = \sqrt{(x_1 - l_{1i})^2 + (x_2 - l_{2i})^2}.
\]
因此,我们可以构建如下的目标函数:
\[
    \min_{\bm{x}} \sum_{i=1}^{N} \sum_{j=1}^{N} (d_i - d_j - d_{ij})^2.
\]
很容易发现,该目标函数并不是一个凸函数,对于这样的一个目标函数,我们可以使用梯度下降法来求解最优解。

设目标函数为\( f(\bm{x}) \),其关于\( \bm{x} \)的导数为
\[
    \begin{split}
        \frac{\partial f(\bm{x})}{\partial \bm{x}} & =2 \sum_{i=1}^{N} \sum_{j=1}^{N} (d_i - d_j - d_{ij}) \left( \frac{\partial d_i}{\partial \bm{x}} - \frac{\partial d_j}{\partial \bm{x}} - \frac{\partial d_{ij}}{\partial \bm{x}} \right) \\
                                                   & = \sum_{i=1}^{N} \sum_{j=1}^{N} (d_i - d_j - d_{ij}) \left( \frac{\bm{x} - \bm{l}_i}{d_i} - \frac{\bm{x} - \bm{l}_j}{d_j} \right)                                                          \\
                                                   & = \sum_{i=1}^{N} \sum_{j=1}^{N} (d_i - d_j - d_{ij}) \left( \bm{x}_i - \bm{x}_j \right),
    \end{split}
\]
其中\( \bm{x}_i =  \frac{\bm{x} - \bm{l}_i}{d_i}\)。因此,我们可以得到如下的迭代公式:
\[
    \bm{x}_{k+1} = \bm{x}_k - \eta \frac{\partial f(\bm{x}_k)}{\partial \bm{x}}.
\]
其中\( \eta \)为学习率。

在实践中,学习率的选择非常重要,过大的学习率会导致目标函数震荡,甚至发散;而过小的学习率则会导致收敛速度过慢。为了避免学习率选择不当导致的震荡和发散,我们可以使用牛顿法来求解最优解。牛顿法的基本思想是通过二阶导数来加速收敛速度。对于目标函数为\( f(\bm{x}) \),其关于\( \bm{x} \)的二阶导数,也即海森(Hessian)矩阵为:
\[
    \begin{split}
        \frac{\partial^2 f(\bm{x})}{\partial \bm{x}^2} & = \frac{\left( \frac{\partial f(\bm{x})}{\partial\bm{x}} \right)^{\mathrm{T}}}{\partial \bm{x}}                                                                                                                                                                         \\
                                                       & = \sum_{i=1}^{N} \sum_{j=1}^{N} \left(\bm{x}_i - \bm{x}_j\right)^{\mathrm{T}} \left(\bm{x}_i - \bm{x}_j\right) + (d_i - d_j - d_{ij}) \left( \frac{\mathbf{I} - \bm{x}_i \bm{x}_i^{\mathrm{T}}}{d_i}  - \frac{\mathbf{I} - \bm{x}_j \bm{x}_j^{\mathrm{T}}}{d_j}\right).
    \end{split}
\]
记\( \bm{d}_{\bm{x}} =  \frac{\partial f(\bm{x})}{\partial \bm{x}}\),\( \mathbf{H}_{\bm{x}} =  \frac{\partial^2 f(\bm{x})}{\partial \bm{x}^2}\),则牛顿法的迭代公式为
\[
    \bm{x}_{k+1} = \bm{x}_k - \mathbf{H}_{\bm{x}_k}^{-1} \bm{d}_{\bm{x}_k}.
\]

从\cref{fig_convergence}可以看到,不同的学习率下,梯度下降法的收敛速度差异很大,并且越大的学习率并不一定收敛得更快。由于牛顿法的收敛速度为二次收敛,而梯度下降法的收敛速度为线性收敛,因此牛顿法的收敛速度远快于梯度下降法。
\begin{figure}[htb!]
    \centering
    \includegraphics[width=.5\textwidth]{./img/locate/convergence.tikz}
    \caption{梯度下降法和牛顿法收敛速度对比}
    \label{fig_convergence}
\end{figure}

此外,需要注意的是,时差定位的目标函数可能会存在多个局部最优解。为了避免陷入局部最优解,有很多方法可以使用,例如随机初始化、模拟退火等。这里我们不再赘述。

\section{频差定位}
利用多普勒效应,我们可以通过运动中的接收基站来确定静止目标的位置。设有目标位置为\( \bm{x} \),并且在\( i \)个时刻,接收基站的位置为\( \bm{l}_i \),速度为\( \bm{v}_i \)。如目标发射的信号频率为\( f_0 \),则根据多普勒效应,接收基站在第\( i \)个时刻接收到的信号频率为
\[
    f_i = f_0 \left(1 + \frac{\bm{v}_i^{\mathrm{T}} (\bm{x} - \bm{l}_i)}{c\left\| \bm{x} - \bm{l}_i \right\|}\right).
\]
但是由于目标发射的信号频率\( f_0 \)是未知的,因此我们无法直接使用上述公式来确定目标的位置。为此,我们需要再次进行观测,设在第\( j \)个时刻,接收基站的位置为\( \bm{l}_j \),速度为\( \bm{v}_j \),则接收基站在第\( j \)个时刻接收到的信号频率为
\[
    f_j = f_0 \left(1 + \frac{\bm{v}_j^{\mathrm{T}} (\bm{x} - \bm{l}_j)}{c\left\| \bm{x} - \bm{l}_j \right\|}\right).
\]
将两式相比,我们可以得到
\[
    \frac{f_i}{f_j} = \frac{1 + \frac{\bm{v}_i^{\mathrm{T}} (\bm{x} - \bm{l}_i)}{c\left\| \bm{x} - \bm{l}_i \right\|}}{1 + \frac{\bm{v}_j^{\mathrm{T}} (\bm{x} - \bm{l}_j)}{c\left\| \bm{x} - \bm{l}_j \right\|}}.
\]
对其进行变形,得到
\[
    f_j \left(1 + \frac{\bm{v}_i^{\mathrm{T}} (\bm{x} - \bm{l}_i)}{c\left\| \bm{x} - \bm{l}_i \right\|}\right) - f_i \left(1 + \frac{\bm{v}_j^{\mathrm{T}} (\bm{x} - \bm{l}_j)}{c \left\| \bm{x} - \bm{l}_j \right\|}\right) = 0.
\]
记
\[
    f_{ij}(\bm{x}) = f_j \left(1 + \frac{\bm{v}_i^{\mathrm{T}} (\bm{x} - \bm{l}_i)}{c\left\| \bm{x} - \bm{l}_i \right\|}\right) - f_i \left(1 + \frac{\bm{v}_j^{\mathrm{T}} (\bm{x} - \bm{l}_j)}{c \left\| \bm{x} - \bm{l}_j \right\|}\right),
\]
则函数\( |f_{ij}(\bm{x})| \)具有\cref{fig_fdoa}所示的等高线。

\begin{figure}[htb!]
    \centering
    \begin{tikzpicture}
        \begin{axis}[
                view={0}{90},  % 俯视图
                colormap/viridis,  % 色彩方案
                contour/labels=false,  % 显示等高线标签
                ticklabel style={font=\tiny},
                % label style={font=\tiny},
                axis equal image,
                xlabel=$ x_1 $, ylabel=$ x_2 $,
            ]
            % 绘制等高线
            \addplot3[
                contour gnuplot,  % 使用gnuplot计算等高线
                contour/number=15,  % 等高线数量
                mesh/rows=200,  % 数据网格的行数
                mesh/cols=200,  % 数据网格的列数
                table/col sep=comma  % CSV分隔符为逗号
            ] table {./img/locate/fdoa.csv};
            \draw[c1, fill=c1!60] (axis cs:-1, 0) circle (2pt);
            \draw[c1, fill=c1!60] (axis cs:1, 0) circle (2pt);
            \draw[c2, fill=c2!60] (axis cs:0, 2) circle (2pt);
        \end{axis}
    \end{tikzpicture}
    \caption{频差定位误差等高线(红色点为目标,蓝色点为观测点,基站自左向右匀速移动)}
    \label{fig_fdoa}
\end{figure}

同样地,我们可以通过增加基站的数量来提高定位精度,设共进行了\( N \)次观测,对应的频差定位目标函数为
\[
    \min_{\bm{x}} \sum_{i=1}^{N} \sum_{j=1}^{N} f_{ij}^2(\bm{x}).
\]
该目标函数的求解与时差定位类似,这里不再赘述。
