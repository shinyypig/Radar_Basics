\chapter{雷达发射与接收信号模型}

\section{发射信号模型}\label{sec_radar_signal_model}
在第一节中我们提到,典型的雷达系统通常采用同一副天线进行信号的发射与接收。雷达通过发射信号,并测量接收回波与发射信号之间的时延来确定目标的距离。严格而言,这种工作方式主要适用于脉冲体制雷达,其发射信号为离散的脉冲序列。与之不同,连续波体制雷达持续向外辐射电磁波,并利用多普勒效应测量目标的距离与速度。鉴于实际应用中脉冲体制雷达更为常见,且连续波雷达的信号处理可在一定条件下转化为脉冲体制的处理形式,因此本书将主要关注脉冲体制雷达的信号模型。

对于单天线雷达系统,其发射信号由三部分组成
\[
    s(t) = \text{rect}\left(\frac{t}{T}\right) p(t) e^{j 2 \pi f_c t},
\]
其中 \( p(t) \) 为低频包络信号,\( e^{j 2 \pi f_c t} \)为高频载波信号,而\( \text{rect}\left(\frac{t}{T}\right) \)为矩形窗函数,有如下表达式
\[
    \text{rect}\left(\frac{t}{T}\right) =
    \begin{cases}
        1, & |t| \leq \frac{T}{2} \\
        0, & |t| > \frac{T}{2}
    \end{cases}.
\]

早期雷达结构简单,对应的低频包络表达式为\( p(t) = 1 \),即发射信号是一个截断的复指数信号。判断是否有回波信号也是通过简单的阈值判断,通常设定一个固定的阈值,当接收到的信号强度超过该阈值时,认为检测到了目标回波。因此,当两个目标对应的回波信号之间的时延差小于脉冲宽度时,系统可能会将其误判为同一目标的回波。所以,早期雷达的距离分辨率取决于发射脉冲的宽度,两者之间的关系可以用以下公式表示:
\[
    \Delta R = \frac{cT}{2},
\]
其中,\( \Delta R \)为距离分辨率,\( c \)为光速。可以发现,脉冲宽度越小,距离分辨率越高。但是,脉冲宽度变小的同时,发射信号的能量也会相应减少,这可能导致信号的信噪比降低,从而影响目标的检测性能。换句话说,距离分辨率与目标检测性能之间存在一定的矛盾。

\begin{figure}[htb!]
    \centering
    \begin{subfigure}{.45\textwidth}
        \centering
        \includegraphics[width=.9\textwidth]{./img/signal/demo_radar_signal1.tikz}
        \caption{}
        \label{fig_radar_signal_1_1}
    \end{subfigure}
    \begin{subfigure}{.45\textwidth}
        \centering
        \includegraphics[width=.9\textwidth]{./img/signal/demo_radar_signal2.tikz}
        \caption{}
        \label{fig_radar_signal_1_2}
    \end{subfigure}
    \caption{单天线雷达发射与接收信号示意图(仅绘制了实部) (a) 发射信号 (b) 接收信号}
    \label{fig_radar_signal_1}
\end{figure}

为兼顾距离分辨率与检测性能,现代雷达系统通常在低频包络 $p(t)$ 中引入时间调制,使其具有一定带宽。这样不仅可以在保持发射信号能量的同时提高距离分辨率,还能提升抗干扰能力与目标检测性能。常见的时间调制方式包括线性调频(Linear Frequency Modulation, LFM)、非线性调频(Nonlinear Frequency Modulation, NLFM)以及相位编码(Phase Coding)等,其中线性调频因其实现简单、性能稳定而被广泛采用。

线性调频信号的低频包络可表示为
\[
    p(t) = e^{j \pi \kappa t^2},
\]
其中 $\kappa$ 为调频率,决定了信号频率随时间的变化速率。由于该信号的瞬时相位为
\[
    \phi(t) = \pi \kappa t^2,
\]
其瞬时频率可由相位求导得到:
\[
    f_{\text{inst}}(t) = \frac{1}{2\pi} \frac{d \phi(t)}{dt} = \kappa t.
\]
如\cref{fig_chirp}所示,线性调频信号的瞬时频率与时间成正比,呈线性变化。

\begin{figure}[htb!]
    \centering
    \begin{subfigure}{.45\textwidth}
        \centering
        \includegraphics[width=.9\textwidth]{./img/signal/chirp.tikz}
        \caption{}
        \label{fig_chirp_1}
    \end{subfigure}
    \begin{subfigure}{.45\textwidth}
        \centering
        \includegraphics[width=.9\textwidth]{./img/signal/chirp_stft.tikz}
        \caption{}
        \label{fig_chirp_2}
    \end{subfigure}
    \caption{线性调频信号示意图 (a) 信号实部 (b) 短时傅里叶变换结果}
    \label{fig_chirp}
\end{figure}

为了进一步提升系统的抗干扰能力与目标检测性能,可采用非线性调频技术,其瞬时相位函数可以有多种形式,例如:
\begin{enumerate}
    \item 三次相位:$\phi(t) = \frac{\kappa}{2} t^3$,瞬时频率随时间二次变化(如\cref{fig_cubic}所示);
    \item 高次多项式相位:$\phi(t) = \sum_{n=0}^{N} a_n t^n$,可灵活控制频率变化规律;
    \item 正弦相位:$\phi(t) = A \sin(2 \pi f_m t + \phi_0)$,瞬时频率呈周期性波动;
    \item 指数相位:$\phi(t) = A \big(e^{\alpha t} - 1\big)$,瞬时频率按指数规律变化。
\end{enumerate}
这些非线性调频形式能够在特定应用中优化脉冲压缩旁瓣特性,提升目标检测与抗干扰性能。但其实现复杂度相对较高,硬件要求也更为严格。

\begin{figure}[htb!]
    \centering
    \begin{subfigure}{.45\textwidth}
        \centering
        \includegraphics[width=.9\textwidth]{./img/signal/cubic.tikz}
        \caption{}
        \label{fig_cubic_1}
    \end{subfigure}
    \begin{subfigure}{.45\textwidth}
        \centering
        \includegraphics[width=.9\textwidth]{./img/signal/cubic_stft.tikz}
        \caption{}
        \label{fig_cubic_2}
    \end{subfigure}
    \caption{三次相位调频信号示意图 (a) 信号实部 (b) 短时傅里叶变换结果}
    \label{fig_cubic}
\end{figure}


相位编码是一种通过在发射信号中引入离散相位变化来实现调制的技术,其核心思想是在信号持续时间内按照预定的编码序列对载波相位进行跳变。常见的相位编码方式包括:
\begin{enumerate}
    \item 二相编码(Binary Phase Shift Keying, BPSK):相位取 $\{0, \pi\}$ 两个值,对应发射信号符号为 $\{+1, -1\}$,典型代表为 Barker 码(如\cref{fig_barker}所示);
    \item 多相编码(M-ary Phase Shift Keying, MPSK):相位在 $M$ 个等间隔值之间跳变,例如四相编码(Quadrature Phase Shift Keying,QPSK)取 $\{0, \pi/2, \pi, 3\pi/2\}$;
    \item Frank 编码:将信号分为若干子脉冲,每个子脉冲的相位按二维矩阵规律编码,可实现较低旁瓣;
    \item P1--P4 编码:适用于脉冲压缩的特殊多相编码序列,通过相位设计获得理想的自相关特性。
\end{enumerate}
相位编码实现相对简单,主要通过数字信号处理技术对信号进行离散化和相位调制。其优点在于能够灵活控制信号的频谱特性,提高抗干扰能力,但对硬件实现要求较高。

\begin{figure}[htb!]
    \centering
    \includegraphics[width=.4\textwidth]{./img/signal/barker13.tikz}
    \caption{Barker-13 相位编码示意图}
    \label{fig_barker}
\end{figure}

若系统的接收天线与发射天线共用或距离较近,则在理想情况下,接收信号可表示为
\[
    r(t) = \sum_{k=1}^{K} A_k s\left(t - \frac{2 R_k}{c}\right),
\]
其中,$K$ 为目标数量,$A_k$ 为一个复数,包含了目标反射信号的幅度与相位信息,$R_k$ 表示第 $k$ 个目标与雷达间的距离,$c$ 为光速。通过检测接收信号中是否包含与发射信号相匹配的特征,并估计相应的时延,即可实现对目标的检测与定位。

\section{阵列接收信号模型}

\subsection{一维均匀线性阵列}
单天线雷达只能获取目标的距离信息,若需进一步测量目标方位,通常需要配备具有指向性的天线,并借助机械旋转实现目标定位。然而,此类雷达系统往往依赖体积庞大且易损的伺服机构。为克服这些缺点,相控阵雷达(Phased Array Radar)应运而生。相控阵雷达利用多天线阵列同时接收信号,并通过分析各接收通道间的相位差,实现对目标方位的精确测量。

以一维均匀线性阵列(Uniform Linear Array, ULA)为例,设有 $M$ 个天线单元(阵元),且阵元间距为 $d$。在目标足够远的情况下,目标反射的电磁波可以近似看作是平面波。

\begin{figure}[htb!]
    \centering
    \includegraphics[width=.6\textwidth]{./img/signal/array_model.tikz}
    \caption{一维线性阵列示意图}
    \label{fig_array}
\end{figure}

如\cref{fig_array}所示,设第 $k$ 个目标与阵列天线的连线与阵列法向之间的夹角为 $\theta_k$。则第 $0$ 个阵元接收到的回波信号,相较于第 $1$ 个阵元接收到的信号,其传播路径少了 $d \sin\theta_k$ 的距离。依次类推,第 $m$ 个阵元接收到的信号,相较于第 $0$ 个阵元接收到的信号,其传播路径少了 $m d \sin\theta_k$ 的距离。不妨以第0个阵元为参考,令其接收到的第\( k \)个目标的回波信号为
\[
    r_k(t) = A_k s\left(t - \frac{2 R_k}{c}\right),
\]
则第 $m$ 个阵元接收到的信号为
\[
    r_k^{(m)}(t) = A_k s\left(t - \frac{2 R_k - m d \sin\theta_k}{c}\right).
\]

将发射信号的基本模型
\[
    s(t) = \operatorname{rect}\left(\frac{t}{T}\right) p(t) e^{j 2 \pi f_c t},
\]
代入线性阵列接收信号模型,可得
\[
    \small
    r_k^{(m)}(t) = A_k \operatorname{rect}\left(\frac{t - \frac{2 R_k - m d \sin\theta_k}{c}}{T}\right)
    p\left(t - \frac{2 R_k - m d \sin\theta_k}{c}\right)
    e^{j 2 \pi f_c \left(t - \frac{2 R_k - m d \sin\theta_k}{c}\right)},
\]
注意到,对于矩形窗函数和低频包络信号,额外的时延 $\frac{m d \sin\theta_k}{c}$ 对其影响可以忽略不计。于是,上式可近似为
\[
    \begin{split}
        r_k^{(m)}(t) \approx & A_k \operatorname{rect}\left(\frac{t - \frac{2 R_k}{c}}{T}\right)
        p\left(t - \frac{2 R_k}{c}\right)
        e^{j 2 \pi f_c \left(t - \frac{2 R_k - m d \sin\theta_k}{c}\right)}                      \\
        =                    & A_k \operatorname{rect}\left(\frac{t - \frac{2 R_k}{c}}{T}\right)
        p\left(t - \frac{2 R_k}{c}\right)
        e^{j 2 \pi f_c \left(t - \frac{2 R_k}{c}\right)}
        e^{j 2 \pi f_c \frac{m d \sin\theta_k}{c}}                                               \\
        =                    & A_k s\left(t - \frac{2 R_k}{c}\right)
        e^{j 2 \pi f_c \frac{m d \sin\theta_k}{c}}                                               \\
        =                    & r_k(t) e^{j 2 \pi f_c \frac{m d \sin\theta_k}{c}}.
    \end{split}
\]
这表明,不同阵元接收到的回波信号,与第 0 个阵元接收到的信号相比,可近似视为相差一个与\( m \)有关的相位因子。

在实际雷达系统中,连续的接收信号经过模数转换采样,可以得到一系列的离散采样点。设每个阵元的采样点数为 \(L\),则第 \(0\) 个阵元接收到的第 \(k\) 个目标的离散回波信号可表示为列向量
\[
    \bm{r}_k =
    \begin{bmatrix}
        r_k(t_0) & r_k(t_1) & \cdots & r_k(t_{L-1})
    \end{bmatrix}^{\mathrm{T}}.
\]
对于第 \(m\) 个阵元,由于目标方位 \(\theta_k\) 引入了阵元间相位差,其回波信号向量可写为
\[
    \bm{r}^{(m)}_k
    = \bm{r}_k \, e^{j 2 \pi f_c \frac{m d \sin\theta_k}{c}},
\]
其中 \(f_c\) 为载频,\(d\) 为阵元间距,\(c\) 为电磁波传播速度。

将所有 \(M\) 个阵元接收到的第 \(k\) 个目标的回波信号按列排列,可以发现第\( k \)个目标对应的接收信号矩阵是一个秩一矩阵,
\[
    \begin{aligned}
        \mathbf{X}_k
         & = \begin{bmatrix}
                 \bm{r}_k^{(0)} & \bm{r}_k^{(1)} & \cdots & \bm{r}_k^{(M-1)}
             \end{bmatrix}                                                           \\
         & = \begin{bmatrix}
                 \bm{r}_k & \bm{r}_k e^{j 2 \pi f_c \frac{d \sin\theta_k}{c}} & \cdots & \bm{r}_k e^{j 2 \pi f_c \frac{(M-1) d \sin\theta_k}{c}}
             \end{bmatrix} \\
         & = \bm{r}_k
        \begin{bmatrix}
            1 & e^{j 2 \pi f_c \frac{d \sin\theta_k}{c}} & \cdots & e^{j 2 \pi f_c \frac{(M-1) d \sin\theta_k}{c}}
        \end{bmatrix}^{\mathrm{T}}                               \\
         & = \bm{r}_k \bm{a}_{\theta_k}^{\mathrm{T}},
    \end{aligned}
\]
其中
\[
    \bm{a}_{\theta_k} =
    \begin{bmatrix}
        1 & e^{j 2 \pi f_c \frac{d \sin\theta_k}{c}} & \cdots & e^{j 2 \pi f_c \frac{(M-1) d \sin\theta_k}{c}}
    \end{bmatrix}^{\mathrm{T}}
\]
称为\textbf{导向矢量}(Steering Vector),其相位包含了目标的方位信息。如图\ref{fig_array_angle}所示,当目标位于阵列的正前方时,所有阵元接收到的信号相位相同;当目标偏离阵列的正前方时,不同阵元接收到的信号之间会存在相位差。

\begin{figure}[htb!]
    \centering
    \begin{subfigure}{.23\textwidth}
        \centering
        \includegraphics[width=.9\textwidth]{./img/signal/array_angle_0.tikz}
        \caption{}
        \label{fig_array_angle_1}
    \end{subfigure}
    \begin{subfigure}{.23\textwidth}
        \centering
        \includegraphics[width=.9\textwidth]{./img/signal/array_angle_p.tikz}
        \caption{}
        \label{fig_array_angle_2}
    \end{subfigure}
    \begin{subfigure}{.23\textwidth}
        \centering
        \includegraphics[width=.9\textwidth]{./img/signal/array_angle_n.tikz}
        \caption{}
        \label{fig_array_angle_3}
    \end{subfigure}
    \caption{一维线性阵列接收信号示例(仅绘制了实部) (a) 目标角度为0 (b) 目标角度为正 (c) 目标角度为负}
    \label{fig_array_angle}
\end{figure}

注意到,导向矢量本质上是一段复指数序列,其对应的``空间频率''为
\[
    f_k = \frac{f_c d \sin\theta_k}{c}.
\]
由于 \( |\theta_k| \leq \tfrac{\pi}{2} \),可得最大的频率为
\[
    |f_k| \leq \frac{f_c d}{c} = \frac{d}{\lambda_c},
\]
其中 \(\lambda = \tfrac{c}{f_c}\) 为载波波长。另一方面,\(\bm{a}_{\theta_k}\) 可以理解为以单位采样率(每阵元采样一次)对该复指数信号进行采样得到的向量。根据奈奎斯特采样定理,为避免频率混叠必须满足
\[
    |f_k| < \frac{1}{2}.
\]
因此,为防止出现空间混叠,阵元间距应满足
\[
    d \leq \frac{\lambda}{2}.
\]


进一步地,在不考虑噪声与干扰的情况下,整个阵列的接收信号矩阵可表示为
\[
    \mathbf{X}
    = \sum_{k=1}^{K} \mathbf{X}_k
    = \sum_{k=1}^{K} \bm{r}_k \, \bm{a}_{\theta_k}^{\mathrm{T}},
\]
这表明理想情况下,对于一维均匀线阵,接收信号矩阵是由多个秩一矩阵叠加而成的。记
\[
    \mathbf{R} = \begin{bmatrix}
        \bm{r}_1 & \bm{r}_2 & \cdots & \bm{r}_K
    \end{bmatrix}, \quad \mathbf{A} = \begin{bmatrix}
        \bm{a}_{\theta_1} & \bm{a}_{\theta_2} & \cdots & \bm{a}_{\theta_K}
    \end{bmatrix},
\]
可以验证
\[
    \mathbf{X} = \mathbf{R} \mathbf{A}^{\mathrm{T}},
\]
其中\( \mathbf{R} \)被称为目标信号矩阵包含了目标的距离信息,\( \mathbf{A} \)被称为导向矢量矩阵(Steering Matrix)或阵列流形矩阵(Array Manifold Matrix)包含了目标的方位信息。

\subsection{二维均匀线性阵列}
为了能够在三维空间中对目标进行定位,通常需要使用二维均匀线性阵列。与一维阵列类似,二维阵列的接收信号模型也可以通过导向矢量来表示。

设有一个 \(M \times N\) 的二维均匀线性阵列,两个方向上的阵元间距分别为 \(d_x\) 和 \(d_y\)。方便起见,令该二维阵列位于\( x-y \) 平面上。因此,阵列上的第 \( (m,n) \) 个阵元的坐标可以表示为
\[
    \bm{l}^{(m, n)} = \begin{bmatrix}
        m d_x & n d_y & 0
    \end{bmatrix}^{\mathrm{T}}.
\]
设第\( k \)个目标相对于阵列坐标系的俯仰角为 \(\theta_k\),方位角为 \(\phi_k\)。如\cref{fig_2d_array_eg}所示,根据球面坐标系到直角坐标系的转换关系,指向目标的单位向量可以表示为
\[
    \bm{u}_k = \begin{bmatrix}
        \sin\theta_k \cos\phi_k & \sin\theta_k \sin\phi_k & \cos\theta_k
    \end{bmatrix}^{\mathrm{T}}.
\]

\begin{figure}[htb!]
    \centering
    \includegraphics[width=.4\textwidth]{./img/signal/2d_array_eg.tikz}
    \caption{二维均匀线性阵列示例}
    \label{fig_2d_array_eg}
\end{figure}

类似于一维阵列,二维阵列中的阵元接收到的信号,相较于位于原点处的参考阵元,会引入额外的时延。根据\cref{fig_2d_array_eg},可以发现,第\( (m,n) \)个阵元接收到的信号少走的距离为
\[
    \Delta d^{(m,n)}_k = \bm{u}_k^{\mathrm{T}} \bm{l}^{(m,n)} = m d_x \sin\theta_k \cos\phi_k + n d_y \sin\theta_k \sin\phi_k.
\]
记\( \vartheta_k = d_x \sin\theta_k \cos\phi_k \),\( \varphi_k = d_y \sin\theta_k \sin\phi_k \),则第\( (m,n) \)个阵元接收到的信号可以表示为
\[
    \begin{split}
                & r^{(m,n)}_k(t) = A_k s\left(t - \frac{2 R_k - m \vartheta_k - n \varphi_k}{c}\right)                                                                                                                                                           \\
        =       & A_k \operatorname{rect}\left(\frac{t - \frac{2 R_k - m \vartheta_k - n \varphi_k}{c}}{T}\right) p\left(t - \frac{2 R_k - m \vartheta_k - n \varphi_k}{c}\right) e^{j 2 \pi f_c \left(t - \frac{2 R_k - m \vartheta_k - n \varphi_k}{c}\right)} \\
        \approx & A_k \operatorname{rect}\left(\frac{t - \frac{2 R_k}{c}}{T}\right) p\left(t - \frac{2 R_k}{c}\right) e^{j 2 \pi f_c \left(t - \frac{2 R_k}{c}\right)} e^{j 2 \pi f_c \frac{m \vartheta_k + n \varphi_k}{c}}                                     \\
        =       & A_k s\left(t - \frac{2 R_k}{c}\right) e^{j 2 \pi f_c \frac{m \vartheta_k + n \varphi_k}{c}}                                                                                                                                                    \\
        =       & r_k(t) e^{j 2 \pi f_c \frac{m \vartheta_k + n \varphi_k}{c}}.
    \end{split}
\]
同样地,令\( \bm{r}_k \) 为第\( (0,0) \)个阵元接收到的信号对应的离散向量,则第\( (m,n) \)个阵元接收到的信号向量可以表示为
\[
    \bm{r}_k^{(m,n)} = \bm{r}_k e^{j 2 \pi f_c \frac{m \vartheta_k + n \varphi_k}{c}} = \bm{r}_k e^{j 2 \pi f_c \frac{m \vartheta_k}{c}} e^{j 2 \pi f_c \frac{n \varphi_k}{c}}.
\]

将所有阵元接收到的第\( k \)个目标反射的信号向量排列成一个\( L \times M \times N \)大小的张量\( \mathcal{X}_k \),则该张量中的元素有如下表达式
\[
    \left( \mathcal{X}_k \right)_{lmn} = \left( \bm{r}_k \right)_l e^{j 2 \pi f_c \frac{m \vartheta_k}{c}} e^{j 2 \pi f_c \frac{n \varphi_k}{c}}.
\]
根据\cref{def:outer-product}中的外积定义,不难发现\( \mathcal{X}_k \)是由三个向量的外积构成的:
\[
    \mathcal{X}_k = \bm{r}_k \circ \bm{a}_{\vartheta_k} \circ \bm{b}_{\varphi_k},
\]
其中\( \bm{a}_{\vartheta_k} \)和\( \bm{b}_{\varphi_k} \)为两个导向矢量,有如下表达式:
\[
    \begin{split}
        \bm{a}_{\vartheta_k} & = \begin{bmatrix}
                                     1 & e^{j 2 \pi f_c \frac{\vartheta_k}{c}} & \cdots & e^{j 2 \pi f_c \frac{(M-1) \vartheta_k}{c}}
                                 \end{bmatrix}^{\mathrm{T}}, \\
        \bm{b}_{\varphi_k}   & = \begin{bmatrix}
                                     1 & e^{j 2 \pi f_c \frac{\varphi_k}{c}} & \cdots & e^{j 2 \pi f_c \frac{(N-1) \varphi_k}{c}}
                                 \end{bmatrix}^{\mathrm{T}}.
    \end{split},
\]

因此,在理想情况下,二维均匀线性阵列的接收信号矩阵是由一系列秩一张量构成的:
\[
    \mathcal{X}
    = \sum_{k=1}^{K} \mathcal{X}_k
    = \sum_{k=1}^{K} \bm{r}_k \circ \bm{a}_{\vartheta_k} \circ \bm{b}_{\varphi_k}.
\]
进一步地,根据\cref{prop:diag-kron},张量\( \mathcal{X} \)可以写成如下形式:
\[
    \begin{split}
        \mathcal{X} = \mathcal{I}_K \times_1 \mathbf{R} \times_2 \mathbf{A} \times_3 \mathbf{B}.
    \end{split},
\]
其中\( \mathcal{I}_K \)为一个\( K \times K \times K \)的对角张量,其对角元素都为1,而\( \mathbf{R} \)、\( \mathbf{A} \)和\( \mathbf{B} \)分别为接收信号、导向矢量的矩阵表示:
\[
    \mathbf{R} = \begin{bmatrix}
        \bm{r}_1 & \cdots & \bm{r}_K
    \end{bmatrix}, \quad
    \mathbf{A} = \begin{bmatrix}
        \bm{a}_{\vartheta_1} & \cdots & \bm{a}_{\vartheta_K}
    \end{bmatrix}, \quad
    \mathbf{B} = \begin{bmatrix}
        \bm{b}_{\varphi_1} & \cdots & \bm{b}_{\varphi_K}
    \end{bmatrix}.
\]

当二维阵列的第二个维度\( N=1 \)时,此时阵列退化为一维阵列,接收张量可以表示为
\[
    \mathcal{X} = \mathcal{I}_K \times_1 \mathbf{R} \times_2 \mathbf{A}.
\]
根据\cref{prop:apx_kron_matmul},可以发现该接收张量的表示形式与一维阵列的情况等价,即
\[
    \mathcal{X} = \mathcal{I}_K \times_1 \mathbf{R} \times_2 \mathbf{A} = \mathbf{R} \mathbf{A}^{\mathrm{T}}.
\]

\section{多普勒接收信号模型}
天线阵列的空间展开赋予雷达``方向分辨力'',从而测量目标的方位;而时间维度的展开则提供了``运动分辨力'',用以估计目标的速度。相较于需要多阵元同步采样的空间维展开,时间维展开的实现更为直接,只需进行脉冲的重复发射与接收。

设雷达系统只有一个天线,且每隔 \(T_r\) 秒发射一个脉冲,即脉冲重复周期(Pulse Repetition Interval, PRI)为 \(T_r\),并持续发射 \(P\) 个脉冲。与此同时,设第\( k \)个目标的径向速度为\( v_k \),则第 \( p \) 脉冲发射的时刻为\( p T_r \)。如\cref{fig_v_k}所示,此时目标到雷达的距离近似为\( R_k -  p T_r v_k\),其中\( R_k \)为初始时刻目标到雷达的距离。

\begin{figure}[htb!]
    \centering
    \includegraphics[width=.35\textwidth]{./img/signal/v_k.tikz}
    \caption{目标的径向速度示意图}
    \label{fig_v_k}
\end{figure}

因此,对于第\( k \)个目标,第\( p \)个脉冲对应接收到的信号可以表示为
\[
    \bm{r}_k^{(p)} = A_k s\left(t - \frac{2 (R_k - p T_r v_k)}{c}\right).
\]
将发射信号的基本模型带入到上式中,有
\[
    \begin{split}
        \bm{r}_k^{(p)} & = A_k \text{rect}\left(\frac{t - \frac{2 (R_k - p T_r v_k)}{c}}{T}\right) p\left(t - \frac{2 (R_k - p T_r v_k)}{c}\right) e^{j 2 \pi f_c \left(t - \frac{2 (R_k - p T_r v_k)}{c}\right)}   \\
                       & \approx A_k \text{rect}\left(\frac{t - \frac{2 R_k}{c}}{T}\right) p\left(t - \frac{2 R_k}{c}\right) e^{j 2 \pi f_c \left(t - \frac{2 R_k}{c}\right)} e^{j 2 \pi f_c \frac{2 p T_r v_k}{c}} \\
                       & = A_k s\left(t - \frac{2 R_k}{c}\right) e^{j 2 \pi f_c \frac{2 p T_r v_k}{c}}                                                                                                              \\
                       & = r_k(t) e^{j 2 \pi f_c \frac{2 p T_r v_k}{c}},
    \end{split}
\]
其中,\( r_k(t) \)为第0次发射脉冲对应的接收信号。可以看出,第\( p \)次脉冲接收到的信号,相较于第0次脉冲接收到的信号,多了一个与\( p \)有关的相位因子,这个相位因子包含了目标的径向速度信息。

同样地,令\( \bm{r}_k \) 为第0次发射脉冲对应的接收信号向量,则第\( p \)次脉冲对应的接收信号向量可以表示为
\[
    \bm{r}_k^{(p)} = \bm{r}_k e^{j 2 \pi f_c \frac{2 p T_r v_k}{c}}.
\]
将所有的\( P \)次脉冲接收到的信号向量组合成一个矩阵,可以表示为
\[
    \begin{split}
        \mathbf{X}_k & = \begin{bmatrix}
                             \bm{r}_k^{(0)} & \bm{r}_k^{(1)} & \cdots & \bm{r}_k^{(P-1)}
                         \end{bmatrix}                                                             \\
                     & = \begin{bmatrix}
                             \bm{r}_k & \bm{r}_k e^{j 2 \pi f_c \frac{2 T_r v_k}{c}} & \cdots & \bm{r}_k e^{j 2 \pi f_c \frac{2 (P-1) T_r v_k}{c}}
                         \end{bmatrix} \\
                     & = \bm{r}_k \begin{bmatrix}
                                      1 & e^{j 2 \pi f_c \frac{2 T_r v_k}{c}} & \cdots & e^{j 2 \pi f_c \frac{2 (P-1) T_r v_k}{c}}
                                  \end{bmatrix}                 \\
                     & = \bm{r}_k \bm{p}_{v_k}^{\mathrm{T}},
    \end{split}
\]
其中
\[
    \bm{p}_{v_k} = \begin{bmatrix}
        1 & e^{j 2 \pi f_c \frac{2 T_r v_k}{c}} & \cdots & e^{j 2 \pi f_c \frac{2 (P-1) T_r v_k}{c}}
    \end{bmatrix}^{\mathrm{T}},
\]
同样也是一个导向矢量,其中包含了目标的径向速度信息。注意到,矩阵 \( \mathbf{X}_k \in \mathbb{C}^{L \times P} \) 的两个维度都与时间相关:其中,维度 \(L\) 对应雷达系统对回波信号的采样时刻,而维度 \(P\) 则对应脉冲的重复发射次数。由于前者的采样间隔远小于后者,通常将 \(L\) 所在的维度称为“快时间”,而将 \(P\) 所在的维度称为“慢时间”。

类似于一维线性阵列中的空间混叠现象,在多普勒体制的雷达中同样可能出现速度模糊。由于\(\bm{p}_k\) 也是一段复指数序列,其对应的“多普勒频率”为
\[
    f_k = f_c \frac{2 T_r v_k}{c}.
\]
由于脉冲在慢时间上以单位采样率(每个脉冲采样一次)采集,为避免混叠必须满足
\[
    |f_k| \leq \frac{1}{2}.
\]
因此,目标径向速度的可测范围为
\[
    |v_k| \leq \frac{c}{4 f_c T_r}
    = \frac{\lambda}{4 T_r}
    = \frac{\lambda f_r}{4},
\]
其中 \(\lambda = c/f_c\) 为载波波长,\(f_r = 1/T_r\) 为脉冲重复频率(Pulse Repetition Frequency, PRF)。除此之外,多普勒体制的雷达还可能出现距离模糊现象。由于脉冲重复周期为 \(T_r\),当目标距离 \(R_k\) 大于 \(\tfrac{c T_r}{2}\) 时,其回波信号至少要到下一次脉冲发射之后才能返回接收机,从而导致目标距离无法被正确估计。由此可得,多普勒雷达的无模糊测距范围为
\[
    R_k \leq \frac{c T_r}{2} = \frac{c}{2 f_r}.
\]
进一步,将无模糊速度与无模糊距离相乘,有
\[
    |v_k| \cdot R_k \;\leq\;
    \frac{c}{4 f_c T_r} \cdot \frac{c T_r}{2}
    = \frac{c^2}{8 f_c}.
\]
这表明,无模糊速度与无模糊距离之间存在乘积约束,其上限仅由载波频率 \(f_c\) 决定。换句话说,若希望同时增大可测速度与可测距离,就必须降低载波频率;然而频率降低又会使雷达系统的体积增大,分辨率下降,从而产生新的折中问题。

考虑所有的\( K \)个目标,最终总的接收信号可以表示为
\[
    \mathbf{X} = \sum_{k=1}^{K} \mathbf{X}_k = \sum_{k=1}^{K} \bm{r}_k \bm{p}_{v_k}^{\mathrm{T}} = \mathbf{R} \mathbf{P}^{\mathrm{T}},
\]
其中
\[
    \mathbf{R} = \begin{bmatrix}
        \bm{r}_1 & \bm{r}_2 & \cdots & \bm{r}_K
    \end{bmatrix}, \quad \mathbf{P} = \begin{bmatrix}
        \bm{p}_{v_1} & \bm{p}_{v_2} & \cdots & \bm{p}_{v_K}
    \end{bmatrix},
\]
分别表示目标的接收信号和导向矢量构成的矩阵。

\section{空时联合接收信号模型}

在前面的推导中,我们分别从空间维度(阵列接收)和时间维度(脉冲重复)出发,建立了目标回波的接收信号模型。实际上,现代雷达系统往往需要在空间与时间两个维度上同时处理数据:一方面,阵列天线提供了空域采样能力,可以用于角度估计与波束形成;另一方面,脉冲重复采样则提供了时域上的信息,可用于目标检测、速度估计以及杂波抑制。将这两个维度结合起来,可以得到所谓的``空时联合接收信号模型''(space-time signal model),为后续的空时自适应处理(Space-Time Adaptive Processing, STAP)奠定基础。本节将建立这一联合模型,并为后续算法的推导提供统一的数学表示。

设有一部多普勒二维阵列雷达系统,阵列规模为 \(M \times N\),在两个正交方向上的阵元间距分别为 \(d_x\) 和 \(d_y\)。系统的脉冲重复周期为 \(T_r\),一次观测中连续发射 \(P\) 个脉冲。考虑场景中存在 \(K\) 个目标,第 \(k\) 个目标的径向距离为 \(R_k\),径向速度为 \(v_k\),其相对于阵列坐标系的俯仰角为 \(\theta_k\),方位角为 \(\phi_k\)。则第 \((m,n)\) 个阵元接收到的第 \(p\) 个脉冲对应的回波信号可以表示为
\[
    \begin{split}
        r^{(m,n,p)}_k(t)
         & = A_k\, s\left(t - \frac{2 R_k - m\vartheta_k - n\varphi_k - p\nu_k}{c}\right) \\
         & \approx r_k(t)\,
        e^{j 2 \pi f_c \tfrac{m \vartheta_k}{c}}\,
        e^{j 2 \pi f_c \tfrac{n \varphi_k}{c}}\,
        e^{j 2 \pi f_c \tfrac{p \nu_k}{c}},
    \end{split}
\]
其中,\(r_k(t)\) 为参考信号,即第 \((0,0)\) 个阵元在初始时刻接收到的回波;而
\[
    \vartheta_k = d_x \sin\theta_k \cos\phi_k,
    \quad
    \varphi_k = d_y \sin\theta_k \sin\phi_k,
    \quad
    \nu_k = 2 T_r v_k
\]
分别表示目标在两个空间维度上的相位偏移量以及在慢时间维度上的多普勒相位偏移。令\( \bm{r}_k \in \mathbb{C}^{L} \) 为对 \(r_k(t)\) 进行离散采样所获得的向量,并构造如下导向矢量
\[
    \begin{split}
        \bm{a}_{\vartheta_k} & = \begin{bmatrix}
                                     1 & e^{j 2 \pi f_c \frac{\vartheta_k}{c}} & \cdots & e^{j 2 \pi f_c \frac{(M-1) \vartheta_k}{c}}
                                 \end{bmatrix}^{\mathrm{T}}, \\
        \bm{b}_{\varphi_k}   & = \begin{bmatrix}
                                     1 & e^{j 2 \pi f_c \frac{\varphi_k}{c}} & \cdots & e^{j 2 \pi f_c \frac{(N-1) \varphi_k}{c}}
                                 \end{bmatrix}^{\mathrm{T}}, \\
        \bm{p}_{\nu_k}       & = \begin{bmatrix}
                                     1 & e^{j 2 \pi f_c \frac{\nu_k}{c}} & \cdots & e^{j 2 \pi f_c \frac{(P-1) \nu_k}{c}}
                                 \end{bmatrix}^{\mathrm{T}}.
    \end{split}
\]
类似于二维阵列接收信号模型,多普勒二维阵列雷达系统整个接收到的第\( k \)个目标的回波数据可以排列为一个\( L \times M \times N \times P \) 的四维张量,有如下表达式
\[
    \mathcal{X}_k = \bm{r}_k \circ \bm{a}_{\vartheta_k} \circ \bm{b}_{\varphi_k} \circ \bm{p}_{\nu_k}.
\]
考虑所有的 \(K\) 个目标,最终总的接收信号张量可以表示为
\[
    \mathcal{X} = \sum_{k=1}^{K} \mathcal{X}_k = \sum_{k=1}^{K} \bm{r}_k \circ \bm{a}_{\vartheta_k} \circ \bm{b}_{\varphi_k} \circ \bm{p}_{\nu_k}.
\]

构造如下的接收信号与导向矢量的矩阵:
\[
    \begin{split}
        \mathbf{R} & = \begin{bmatrix}
                           \bm{r}_1 & \cdots & \bm{r}_K
                       \end{bmatrix}, \quad
        \mathbf{A} = \begin{bmatrix}
                         \bm{a}_{\vartheta_1} & \cdots & \bm{a}_{\vartheta_K}
                     \end{bmatrix}, \quad \\
        \mathbf{B} & = \begin{bmatrix}
                           \bm{b}_{\varphi_1} & \cdots & \bm{b}_{\varphi_K}
                       \end{bmatrix}, \quad
        \mathbf{P} = \begin{bmatrix}
                         \bm{p}_{\nu_1} & \cdots & \bm{p}_{\nu_K}
                     \end{bmatrix}.
    \end{split}
\]
那么根据\cref{prop:diag-kron},张量\( \mathcal{X} \)可以写成如下形式:
\[
    \begin{split}
        \mathcal{X} = \mathcal{I}_K \times_1 \mathbf{R} \times_2 \mathbf{A} \times_3 \mathbf{B} \times_4 \mathbf{P},
    \end{split}
\]
其中\( \mathcal{I}_K \)为一个\( K \times K \times K \times K \)的单位张量。

这一张量表示不仅与前文所述的阵列接收信号模型和多普勒接收信号模型兼容,而且还与\cref{sec_radar_signal_model} 中介绍的最基本的单天线接收信号模型相一致:
\[
    r(t) = \sum_{k=1}^{K} A_k s\!\left(t - \frac{2 R_k}{c}\right).
\]
在这种情况下,接收信号张量可写为
\[
    \mathcal{X} = \mathcal{I}_K \times_1 \mathbf{R},
\]
其中 \(\mathcal{I}_K\) 退化为一阶单位张量,即一个全 1 向量。于是,
\[
    \mathcal{X} = \mathbf{R}\bm{1} = \sum_{k=1}^{K} \bm{r}_k,
\]
恰好对应所有目标回波向量的叠加。

综上所述,引入张量这一表示方式后,原本复杂的接收信号模型得到了显著简化,并在形式上实现了高度统一。无论是最简单的单阵元单脉冲体制,还是复杂的多阵元多脉冲体制,甚至是包含了极化域信息的多维雷达系统,都能用同一套张量框架加以描述。这不仅为不同雷达体制提供了统一的数学表述,也为后续空时联合处理和多维信号处理方法奠定了坚实基础。

\section{噪声、杂波和多径模型}

在前面的讨论中,我们主要关注了理想情况下的雷达接收信号模型。然而,在实际应用中,雷达系统不可避免地会受到各种噪声、杂波和多径效应的影响,这些因素会显著干扰目标信号的检测与估计。因此,建立包含这些干扰因素的更为现实的接收信号模型,对于设计鲁棒的雷达信号处理算法至关重要。

本书中的噪声通常是指热噪声,是由接收机前端、信道特性和环境因素共同作用的结果,一般来说可以建模为加性高斯白噪声(Additive White Gaussian Noise, AWGN)。这里我们以一维均匀线性阵列为例,设一共有\( M \)个阵元,那么每个阵元在采集信号的过程中都不可避免地会被噪声干扰。令采样点数为\( L \),则接收数据为一个\( L \times M \)的矩阵,可以表示为
\[
    \hat{\mathbf{X}} = \mathbf{X} + \mathbf{N},
\]
其中,\(\mathbf{X}, \mathbf{N} \in \mathbb{C}^{L \times M}\) 分别为接收信号矩阵和接收噪声矩阵。假设噪声是独立同分布的复高斯随机变量,其均值为零,方差为\( \sigma^2 \)。那么,\(\mathbf{N}\) 的协方差矩阵可以近似为一个对角矩阵:
\[
    \mathbf{\Sigma} = \frac{1}{L} \mathbf{N}^{\mathrm{H}} \mathbf{N} \approx \sigma^2 \mathbf{I},
\]
其中,\(\mathbf{I}\) 为 \(M \times M\) 的单位矩阵。由于噪声与信号通常相互独立,因此信号与噪声的互协方差矩阵(信号与噪声的均值均为零)可以近似为全零矩阵:
\[
    \mathbf{\Sigma}_{\mathbf{XN}} = \frac{1}{L} \mathbf{X}^{\mathrm{H}} \mathbf{N} \approx \mathbf{0}.
\]
这一性质在实际应用中是非常重要的,因为它意味着在信号处理过程中可以将噪声视为与信号相互独立的干扰项,从而简化了许多算法的设计与分析。由于硬件滤波、带宽限制等因素,噪声也可能呈现出非高斯特性,对应的协方差矩阵有可能不再是对角矩阵,这类噪声被称为色噪声(Colored Noise)。在这种情况下,噪声的统计特性会更加复杂,需要采用更为精细的模型来描述其影响,本书不做深入探讨。

除了噪声以外,杂波也是影响雷达系统探测与成像性能的重要因素。杂波通常是指来自环境的非目标回波,例如地面或海面反射、雨滴、鸟类等引起的回波,这些杂波信号会与目标信号混叠在一起,增加了目标检测和定位的难度。

以地杂波为例,由于地面通常是相对固定的散射体,比如建筑、山体、树林等,且在较为小的范围内这些目标数量很多,因此产生的地杂波可以看作是大量的幅度微弱的回波信号的叠加。根据中心极限定理,这些微弱回波信号的叠加可以近似看作是一个高斯随机过程,因此地杂波通常也可以建模为加性高斯噪声。然而,与热噪声不同的是,地杂波往往具有较强的空间相关性和时间相关性,这使得其协方差矩阵不再是一个对角矩阵,反映了杂波在不同阵元和不同时间点之间的相关性。不同的地形特征和环境条件也会对杂波的特性产生影响,对应的杂波模型也完全有可能不再服从高斯分布,需要根据具体应用场景进行调整。

与地杂波相比,海杂波通常表现出更加复杂的特性。海面是一个动态变化的散射体,其反射特性不仅受到风速的影响,还与海洋温度、流场等多种环境因素密切相关。因此,在对海杂波进行建模时,往往需要引入更多环境变量,并采用更复杂的统计模型来刻画其特性。在海面较为平静、风速较小时,海面上主要由大量微小波浪构成,此时的杂波可以近似看作众多微弱回波的叠加,类似于地杂波,能够建模为加性高斯噪声。而当风速较大、海况较为复杂时,海杂波的幅度分布则会呈现明显的“尖峰”与“长尾”特征。这是因为杂波不再是单一高斯过程,而是由多个具有不同功率水平的高斯过程复合而成。换言之,尽管局部小波浪的回波叠加可以近似为高斯分布,但在大风浪作用下,各种尺度的海浪同时存在,最终的回波分布表现为多个高斯过程的混合。该类复合过程对应的幅值分布即为 K 分布(K-distribution),目前已被广泛应用于海杂波的建模与分析中。除此之外,Weibull 分布、Log-normal 分布等也常被用于描述海杂波的幅度特性。

除了噪声与杂波之外,多径效应也是雷达系统中普遍存在且不可忽视的干扰因素。所谓多径,是指电磁波在传播过程中,不仅沿着目标的直达路径到达接收端,还会在地面、海面、建筑物或其他障碍物上发生反射、散射或绕射,从而形成多条传播路径。接收机最终接收到的信号是这些不同路径回波的叠加。

设理想情况下接收到的第 \( k \) 个目标回波为 \( r_k(t) \),在考虑多径效应时,对应的接收信号可建模为
\begin{equation}
    \tilde{r}_k(t) = \sum_{q=1}^{Q} \alpha_q \, r_k\!\left(t - \tau_q\right) e^{j \phi_q},
\end{equation}
其中,\( Q \) 为多径条数,\(\alpha_q\)、\(\tau_q\) 与 \(\phi_q\) 分别表示第 \( q \) 条路径的衰减系数、时延和相位偏移。这些参数通常与传播环境密切相关,例如地形起伏、建筑物分布以及天气条件等。

多径效应可能导致接收信号出现展宽,使脉冲波形被拉长甚至畸变,从而降低系统的时延分辨率。不同路径分量在接收端叠加时,可能产生相长或相消干涉:若多径分量之间具有固定的相位关系,则会形成稳定的干涉条纹;若相位随时间快速变化,则表现为信号强度的随机起伏,即常说的``衰落''(Fading)。此外,在雷达成像与检测中,多径分量还可能以``伪目标''的形式出现。例如,海面或地面反射路径可能导致目标被``抬高''或``镜像化'',从而引发误判;在复杂环境下,甚至可能出现多个虚假目标与真实目标混叠,增加目标检测与识别的难度。
