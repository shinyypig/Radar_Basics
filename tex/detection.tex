\chapter{雷达信号检测}
在雷达系统中,目标检测是由接收信号处理走向目标识别的关键环节。其核心任务是从含有噪声、杂波与多径效应的复杂回波中,可靠地区分目标与干扰。为了提升检测性能,雷达信号处理往往需要经过脉冲压缩以提高距离分辨率,通过脉冲积累增强信噪比,最终在统计意义下实现目标检测。

本章将依次介绍:首先讨论脉冲压缩原理及其在提高雷达分辨率中的作用;随后分析非相干积累与相干积累两类脉冲积累方法的特点与适用场景;最后重点研究目标检测的统计理论,包括虚警率、检测率、门限选择与恒虚警率(CFAR)检测等经典方法。

\section{脉冲压缩}
如\cref{sec_radar_signal_model}所述,早期雷达系统往往难以同时兼顾距离分辨率与目标检测性能。为在保持较高距离分辨率的同时提升检测效能,现代雷达通常在低频包络中引入时间调制以拓展信号带宽,从而增强系统的抗干扰能力与目标检测能力。以线性调频信号为例,其回波信号的实部如\cref{fig_chirp_eg}所示。在实际应用中,由于噪声等因素的存在,往往难以直接确定回波脉冲的上升沿;此外,若仅依靠上升沿来估计脉冲时延,也无法有效解决距离分辨率与检测性能之间的固有矛盾。

\begin{figure}[htb!]
    \centering
    \includegraphics[width=.6\textwidth]{./img/detection/chirp_eg.tikz}
    \caption{线性调频回波信号示意图}
    \label{fig_chirp_eg}
\end{figure}

需要注意的是,\cref{fig_chirp_eg} 中仅绘制了回波信号的低频包络部分。这是因为在实际应用中,回波信号通常会先经过下变频处理,再进行采样。经过这一过程,高频载波成分被滤除,仅保留下变频后的低频信息。因此,本章讨论的信号均不再考虑高频载波。

为克服这一难题,雷达信号处理中引入了脉冲压缩(Pulse Compression)技术。该方法通过在发射端发送宽脉冲、在接收端对回波进行匹配滤波,实现了``能量积累''与``时间压缩''的统一。由此,雷达既能在保持高能量积累以提高检测概率的同时,又获得等效的窄脉冲,从而显著提升距离分辨率。

设\( \bm{s} \in \mathbb{C}^{W \times 1} \)为发射信号对应的离散时间序列向量,也称作参考向量,\( \bm{x} \in \mathbb{C}^{L \times 1} \)为接收信号对应的离散时间序列向量。此时,问题就转换成从回波向量\( \bm{x} \)中找到与发射向量相似的序列。不妨将\( \bm{x} \)排列成如下的矩阵形式
\[
    \mathbf{X} = \begin{bmatrix}
        x_1    & x_2     & \cdots & x_{L-W+1} \\
        x_2    & x_3     & \cdots & x_{L-W+2} \\
        \vdots & \vdots  & \ddots & \vdots    \\
        x_W    & x_{W+1} & \cdots & x_L
    \end{bmatrix} \in \mathbb{C}^{W \times (L-W+1)},
\]
其中矩阵\( \mathbf{X} \)的第\( i \)列记为\( \bm{x}_i \in \mathbb{C}^{W \times 1} \)。显然,我们只需要一一对比所有的\( \bm{x}_i \)与\( \bm{s} \)的相似性,如果相似度比较高,那么就可以认为在该位置存在回波信号,自然就能确定对应的时延了。最常见的衡量相似度的指标是余弦相似度(Cosine Similarity),其定义如下
\begin{definition}
    设有两个向量\( \bm{a}, \bm{b} \in \mathbb{C}^{n \times 1} \),则它们的余弦相似度有如下计算公式
    \[
        S(\bm{a}, \bm{b}) = \frac{\bm{a}^{\mathrm{H}} \bm{b}}{\|\bm{a}\| \|\bm{b}\|}
    \]
    其中\( \bm{a}^{\mathrm{H}} \)为\( \bm{a} \)的共轭转置。余弦相似度越大,一定程度上表示两个向量越相似。
\end{definition}

因此,依次计算所有的
\[
    y_i = S(\bm{x}_i, \bm{s}),
\]
即可得到脉冲压缩后的输出结果:
\[
    \bm{y} = \begin{bmatrix} y_1 & y_2 & \cdots & y_{L-W+1} \end{bmatrix}^{\mathrm{T}}
    \in \mathbb{C}^{(L-W+1) \times 1}.
\]
然而,余弦相似度的计算在硬件实现中相对复杂,因此在实际应用中往往需要进一步简化。注意到参考向量 \( \bm{s} \) 是已知的,不妨提前归一化\footnote{后文均默认\( \bm{s} \)为归一化后的向量},令其模长\( \left\| \bm{s} \right\| = 1 \)。于是余弦相似度可简化为
\[
    y_i = \frac{\bm{x}_i^{\mathrm{H}} \bm{s}}{\|\bm{x}_i\|}.
\]
但在工程实现中,计算 \( \|\bm{x}_i\| \) 仍然存在一定开销。

为克服这一问题,可以预先对接收向量 \( \bm{x} \) 进行归一化,使得不同距离单元处的信号能量保持在相近水平。根据雷达方程\cref{eq:radar_equation},接收信号的功率随目标距离的四次方衰减。因此,可以利用这一先验规律,对回波信号施加随时间(或距离)变化的自动增益控制:距离越远,信号衰减越严重,所需增益越大。这样便能在不显著增加硬件复杂度的前提下,近似实现归一化处理。在这种情况下,余弦相似度可近似简化为一个内积形式:
\[
    y_i =  \bm{x}_i^{\mathrm{H}} \bm{s},
\]
并可以进一步写成矩阵乘法的形式:
\[
    \bm{y} = \mathbf{X}^{\mathrm{H}} \bm{s}.
\]

由\cref{apx.conv-corr-mat}可知,该形式实质上对应一次相关运算,即将原始回波信号与参考信号进行相关处理。进一步地,相关运算可以等价地转化为卷积运算,而卷积又可视为一个有限冲激响应(Finite Impulse Response, FIR)滤波器,其滤波器系数正是参考信号的共轭翻转序列。因此,该处理过程通常被称为\emph{匹配滤波}。方便起见,我们直接称\( \bm{s} \)为匹配滤波器的滤波器系数。

\begin{example}\label{eg_pc}
    对\cref{fig_pc_eg1}中的接收信号进行匹配滤波处理。
    \begin{figure}[htb!]
        \centering
        \includegraphics[width=.45\textwidth]{./img/detection/pc_eg1.tikz}
        \caption{接收信号示意图(仅绘制了实部)}
        \label{fig_pc_eg1}
    \end{figure}
\end{example}
\begin{solution}
    从原始回波信号可以看出,噪声干扰较为明显。若仅依靠脉冲的上升沿进行判别,只能大致识别出两个目标回波。而经过匹配滤波处理后,不仅能够清晰地区分出三个目标回波,而且噪声得到了有效抑制,其结果如\cref{fig_pc_eg2}所示。由于匹配滤波能够将宽脉冲在时域中压缩为尖锐的脉冲峰,这一过程通常也被称为脉冲压缩。
    \begin{figure}[htb!]
        \centering
        \includegraphics[width=.45\textwidth]{./img/detection/pc_eg2.tikz}
        \caption{脉冲压缩结果}
        \label{fig_pc_eg2}
    \end{figure}

    令\( A \)为回波对应的幅度与相位信息,并假设噪声的均值\( \mu_n \)为零,方差为\( \sigma_n^2 \),则对于回波处的采样点,其均值为
    \[
        \mu_s = \operatorname{E}[A s_k + n] = A s_k,
    \]
    其中\( s_k \)为参考信号的第\( k \)个采样点, \( n \) 为噪声项。注意到,对于线性调频信号而言,归一化后的参考向量\( \bm{s} \in \mathbb{C}^{W \times 1} \)中的元素\( s_k \)的模值均为\( \frac{1}{\sqrt{W}} \)。因此,根据信噪比的定义\footnote{本书中,信噪比均采用偏转系数(Deflection Coefficient)定义,即\( SNR = \frac{|\mu_s - \mu_n|^2}{\sigma_n^2} \),其中\( \mu_s \) 和 \( \mu_n \) 分别为目标和噪声的均值,\( \sigma_n^2 \) 为噪声的方差。当噪声均值为零时,偏转系数与经典信噪比形式一致。},可以得到原始回波信号的信噪比为
    \[
        \mathrm{SNR}_{a} = \frac{(\mu_s - \mu_n)^2}{\sigma^2} = \frac{|A s_k - 0|^2}{\sigma^2} = \frac{|A|^2}{\sigma^2 W}.
    \]

    不妨令向量\( \bm{x}_i = A \bm{s} + \bm{n}_i \)对应了目标的回波信号,其中\(\bm{n}_i\) 为噪声向量。那么脉压后,目标对应的尖值取值的均值为
    \[
        \mu_s = \operatorname{E}[\bm{x}_i^{\mathrm{H}} \bm{s}] = \operatorname{E}[(A \bm{s} + \bm{n}_i)^{\mathrm{H}} \bm{s}] = A \bm{s}^{\mathrm{H}} \bm{s} + \operatorname{E}[\bm{n}_i^{\mathrm{H}} \bm{s}] = A.
    \]
    与此同时,令\( \bm{x}_j = \bm{n}_j \)为噪声对应的向量,其不包含目标回波,则对应的脉压结果的均值为
    \[
        \mu_n = \operatorname{E}[\bm{n}_j^{\mathrm{H}} \bm{s}] = 0,
    \]
    方差为
    \[
        \sigma^2 = \operatorname{E}[|\bm{n}_j^{\mathrm{H}} \bm{s} - 0|^2] = \operatorname{E}[\bm{s}^{\mathrm{H}} \bm{n}_j \bm{n}_j^{\mathrm{H}} \bm{s}] = \sigma^2 \bm{s}^{\mathrm{H}} \mathbf{I} \bm{s} = \sigma^2.
    \]
    因此,脉冲压缩后的信噪比为
    \[
        \mathrm{SNR}_{b} = \frac{|A-0|^2}{\sigma^2} = \frac{|A|^2}{\sigma^2}.
    \]
    综上,脉冲压缩的信噪比提升倍数为
    \[
        \frac{\mathrm{SNR}_{b}}{\mathrm{SNR}_{a}} = W.
    \]
\end{solution}

既然脉冲压缩本质上是对原始回波信号的一种滤波,而其对应的滤波器系数通常取为参考信号所对应的向量,那么一个自然的问题是:最优的滤波系数是否必然就是参考信号向量?不妨将期望的滤波器系数记为 \( \bm{w} \)。显然,理想的滤波器应在凸显目标回波信号的同时,尽可能抑制非目标分量,例如噪声、干扰和杂波等。基于这一考虑,可以建立如下的优化模型:
\[
    \begin{cases}
        \min\limits_{\bm{w}} \left\| \mathbf{X}^{\mathrm{H}} \bm{w} \right\|^2, \\[6pt]
        \text{s.t. } \bm{s}^{\mathrm{H}} \bm{w} = 1 ,
    \end{cases}
\]
其中,目标函数刻画了滤波器输出的总能量,约束条件保证了参考信号方向上的响应被固定为 1。这样,滤波器在保持对目标信号敏感的同时,能够最大程度地抑制非目标分量的输出。利用\cref{apx.lagrange-multiplier}介绍的拉格朗日乘数法,可以获得该优化问题的解析解。首先,构建拉格朗日函数
\[
    \mathcal{L}(\bm{w}, \lambda) = \frac{1}{2}\left\| \mathbf{X}^{\mathrm{H}} \bm{w} \right\|^2 - \lambda \left( \bm{s}^{\mathrm{H}} \bm{w} - 1 \right).
\]
对\( \bm{w} \)和\( \lambda \)分别求导,并令其为0,得到方程组
\[
    \begin{cases}
        \frac{\partial \mathcal{L}}{\partial \bm{w}} = \mathbf{X} \mathbf{X}^{\mathrm{H}} \bm{w} - \lambda \bm{s} = 0 \\
        \frac{\partial \mathcal{L}}{\partial \lambda} = \bm{s}^{\mathrm{H}} \bm{w} - 1 = 0
    \end{cases}.
\]
联立上述方程组,得到
\[
    \begin{cases}
        \lambda = \frac{1}{\bm{s}^{\mathrm{H}} \mathbf{X} \mathbf{X}^{\mathrm{H}} \bm{s}} \\
        \bm{w} = \frac{\mathbf{X} \mathbf{X}^{\mathrm{H}} \bm{s}}{\bm{s}^{\mathrm{H}} \mathbf{X} \mathbf{X}^{\mathrm{H}} \bm{s}}
    \end{cases}.
\]
从中可以看出,最优的滤波器系数 \( \bm{w} \) 并不等于参考信号 \( \bm{s} \)。但注意到,当我们假设信号中只存在高斯白噪声时,数据的协方差矩阵近似为一个对角矩阵\( \frac{1}{L-W+1} \mathbf{X} \mathbf{X}^{\mathrm{H}} \approx \sigma^2 \mathbf{I} \),其中\( \mathbf{I} \)为单位矩阵。此时,最优的滤波器系数 \( \bm{w} \) 近似为
\[
    \bm{w} \approx \frac{ \mathbf{I} \bm{s}}{\bm{s}^{\mathrm{H}} \mathbf{I} \bm{s}} = \frac{\bm{s}}{\bm{s}^{\mathrm{H}} \bm{s}} = \bm{s}.
\]
这表明,在仅含白噪声的理想环境下,最优的滤波器系数近似等于参考向量。

此外需要注意,脉冲压缩后的输出脉冲实际上对应于参考向量的自相关,因此其波形形状与参考向量(低频包络)密切相关。注意到,发射信号的低频包络由两部分组成
\[
    s(t) = \operatorname{rect}\left( \frac{t}{T} \right) p(t).
\]
其自相关函数为
\[
    R_s(\tau) = \operatorname{E}\left[ s(t)s^*(t-\tau)\right]
    = \lim_{T \rightarrow \infty } \frac{1}{T} \int_{-T/2}^{T/2} \operatorname{rect}\left( \frac{t}{T} \right)
    \operatorname{rect}\left( \frac{t-\tau}{T} \right)
    p(t)p^*(t-\tau)dt.
\]
不妨设 \(p(t)\) 为宽平稳随机过程,则其自相关函数\( R_p(\tau) = \mathrm{E}[p(t)p^*(t-\tau)] \) 仅与时间差 \(\tau\) 有关。由此可得近似
\[
    \begin{split}
        R_s(\tau) & \approx R_p(\tau)
        \lim_{T \rightarrow \infty } \frac{1}{T} \int_{-T/2}^{T/2}
        \operatorname{rect}\left( \frac{t}{T} \right)
        \operatorname{rect}\left( \frac{t-\tau}{T} \right) dt                    \\
                  & = R_p(\tau) \operatorname{Tri}\left( \frac{\tau}{T} \right),
    \end{split}
\]
其中 \(\operatorname{Tri}(\cdot)\) 为三角函数,定义为
\[
    \operatorname{Tri}(x) =
    \begin{cases}
        1-|x|, & |x|\leq 1, \\
        0,     & |x|>1.
    \end{cases}
\]

这表明,脉冲压缩输出的脉冲形状近似为\( p(t) \)自相关函数与三角窗的乘积。而一个信号的自相关函数宽度取决于其带宽:带宽越大,自相关函数越窄,反之亦然。若假设 \(p(t)\) 的带宽为 \(B\),则其自相关函数宽度约为 \(1/B\)。另一方面,若两个目标间的距离差为 \(\Delta R\),则其回波经脉冲压缩后的时间间隔为 \(\frac{2\Delta R}{c}\)。因此,只有当\(\frac{2\Delta R}{c} > \frac{1}{B}\)时,两个脉冲才能有效区分。由此可见,脉冲压缩技术能够将雷达的距离分辨率提升至
\[
    \Delta R = \frac{c}{2B}.
\]

\cref{fig_mf} 展示了不同低频包络信号及其脉冲压缩结果。可以看到,随着信号带宽的增加,压缩后的波形变得更加尖锐。但需要注意的是,脉冲压缩输出的尖峰仍具有一定的宽度,其中最高的部分称为主瓣(Mainlobe),而其余较低的部分则称为旁瓣(Sidelobe)。

\begin{figure}[htb!]
    \centering
    \begin{subfigure}{.3\textwidth}
        \centering
        \includegraphics[width=.9\textwidth]{./img/detection/mf1.tikz}
        \includegraphics[width=.9\textwidth]{./img/detection/mf4.tikz}
        \caption{复指数信号}
        \label{fig_mf_1}
    \end{subfigure}
    \begin{subfigure}{.3\textwidth}
        \centering
        \includegraphics[width=.9\textwidth]{./img/detection/mf2.tikz}
        \includegraphics[width=.9\textwidth]{./img/detection/mf5.tikz}
        \caption{线性调频信号\( \kappa = 20 \)}
        \label{fig_mf_2}
    \end{subfigure}
    \begin{subfigure}{.3\textwidth}
        \centering
        \includegraphics[width=.9\textwidth]{./img/detection/mf3.tikz}
        \includegraphics[width=.9\textwidth]{./img/detection/mf6.tikz}
        \caption{线性调频信号\( \kappa=100 \)}
        \label{fig_mf_3}
    \end{subfigure}
    \caption{不同低频包络信号及其脉冲压缩结果}
    \label{fig_mf}
\end{figure}

在\cref{eg_pc}中,我们已经证明,脉冲压缩所带来的信噪比提升倍数等于参考向量的长度 \(W\)。根据香农采样定理,采样率 \(f_s\) 至少应满足 \(f_s \geq 2B\),其中 \(B\) 为信号带宽。因此,对于时宽为 \(T\) 的发射信号,其参考向量的采样点数至少为
\[
    W = f_s T \geq 2BT.
\]
这表明,脉冲压缩所获得的处理增益与发射信号的时宽和带宽的乘积(Time-Bandwidth Product)成正比。

\section{脉冲积累}
在实际应用中,对于部分微弱目标,即便采用脉冲压缩技术,单次脉冲的信噪比仍可能不足以实现有效检测。此时,可以借助脉冲积累(Pulse Integration)进一步提升信噪比。根据积累过程中是否保留信号的相位信息,脉冲积分类方法可分为两类:其一是对回波进行相位对齐后再叠加的相干积累(Coherent Integration);其二是直接对回波幅度或功率进行统计平均的非相干积累(Non-coherent Integration)。前者能够充分利用信号能量,但对相位同步精度要求较高;后者实现简便,但积累增益有限。

设多普勒雷达系统共发射 \( P \) 个脉冲,并且场景中仅存在一个目标。则脉冲压缩后第 \( p \) 个脉冲对应的尖峰输出为
\[
    y_p = A e^{j \varphi_p} + n_p,
\]
其中 \( A \) 表示目标的幅度与相位信息,\(\varphi_p\) 为第 \( p \) 个脉冲带来的额外相位,\( n_p \) 为服从方差为\( \sigma^2 \)的复高斯分布的随机变量,即\( n_p \sim \mathcal{CN}(0,\sigma^2) \)。此外,噪声处对应的脉冲压缩输出记为 \( w_p \),其与 \( n_p \) 独立同分布。则单脉冲信噪比为
\[
    \mathrm{SNR}_s
    = \frac{\left| \operatorname{E}(y_p) - \operatorname{E}(w_p) \right|^2}{\sigma^2}
    = \frac{|A|^2}{\sigma^2}.
\]

相干积累的基本思想是对各脉冲进行相位补偿后再求平均,因此目标对应的输出为
\[
    y_{\mathrm{ci}} = \frac{1}{P} \sum_{p=1}^P y_p e^{-j \varphi_p} = A + \frac{1}{P} \sum_{p=1}^P n_p e^{-j \varphi_p}.
\]
相应的噪声输出为
\[
    w_{\mathrm{ci}} = \frac{1}{P} \sum_{p=1}^P w_p e^{-j \varphi_p}.
\]
因此,相干积累后目标的均值为
\[
    \operatorname{E}[y_{\mathrm{ci}}] = A + \operatorname{E}\left[ \frac{1}{P} \sum_{p=1}^P n_p e^{-j \varphi_p} \right] = A,
\]
噪声的均值为
\[
    \operatorname{E}[w_{\mathrm{ci}}] = \frac{1}{P} \sum_{p=1}^P \operatorname{E}[w_p] e^{-j \varphi_p} = 0.
\]
由于各个噪声项相互独立,因此其方差为
\[
    \operatorname{Var}[w_{\mathrm{ci}}] =\operatorname{Var}\left[\frac{1}{P}\sum_{p=1}^P w_p e^{-j \varphi_p} \right] = \frac{1}{P^2}\sum_{p=1}^P \operatorname{Var}[w_p] = \frac{\sigma^2}{P}.
\]
由此可得相干积累后的信噪比为
\[
    \mathrm{SNR}_{\mathrm{ci}} = \frac{|A - 0|^2}{\frac{\sigma^2}{P}} = \frac{P|A|^2}{\sigma^2}.
\]

这表明,当有\( P \)个脉冲时,通过相干积累可以将信噪比提升至原来的\( P \)倍。但相干积累要求能够精确地对齐各个脉冲的相位,这在实际应用中可能会难以实现。此时,可以考虑非相干积累的方法,其思想是直接对回波的模值平方取平均,即
\[
    y_{\mathrm{nci}} = \frac{1}{P} \sum_{p=1}^P |y_p|^2 = \frac{1}{P} \sum_{p=1}^P \left| A e^{j \varphi_p} + n_p \right|^2.
\]
对应的噪声输出为
\[
    w_{\mathrm{nci}} = \frac{1}{P} \sum_{p=1}^P |w_p|^2.
\]
计算可得
\[
    \begin{split}
        \operatorname{E}[y_{\mathrm{nci}}] & = \frac{1}{P} \sum_{p=1}^P \operatorname{E}\left[ |A e^{j \varphi_p} + n_p|^2 \right] = \frac{1}{P} \sum_{p=1}^P \operatorname{E} \left[ |A|^2 + |n_p|^2 + A e^{j \varphi_p} \overline{n}_p + \overline{A} e^{-j \varphi_p} n_p \right] \\
                                           & = \frac{1}{P} \sum_{p=1}^P \left( |A|^2 + \operatorname{E}[|n_p|^2] \right)                                                                                                                                                             \\
                                           & = |A|^2 + \sigma^2.
    \end{split}
\]
由于复高斯噪声的模平方服从均值为\( \sigma^2 \)、方差为\( \sigma^4 \)的指数分布(Exponential Distribution)。因此,非相干积累后噪声的均值为
\[
    \begin{split}
        \operatorname{E}[w_{\mathrm{nci}}] & =\frac{1}{P}  \sum_{p=1}^P \operatorname{E}\left[ |w_p|^2 \right] = \sigma^2.
    \end{split}
\]
同样,因为\( |w_p|^2 \)之间互相独立,所以噪声的方差为
\[
    \operatorname{Var}[w_{\mathrm{nci}}] = \operatorname{Var}\left[ \frac{1}{P} \sum_{p=1}^P |w_p|^2\right] = \frac{1}{P^2}\sum_{p=1}^P \operatorname{Var}\left[ |w_p|^2\right] = \frac{\sigma^4}{P}.
\]
由此可得非相干积累后的信噪比为
\[
    \mathrm{SNR}_{\mathrm{nci}} = \frac{\left||A|^2 + \sigma^2 - \sigma^2\right|^2}{\frac{\sigma^4}{P}} = \frac{P |A|^4}{\sigma^4}.
\]
注意到,非相干积累的输出实质上是各脉冲回波幅度平方的累加,因此在定义信噪比时,通常再取平方根作为最终结果:
\[
    \mathrm{SNR}_{\mathrm{nci}} = \frac{\sqrt{P} |A|^2}{\sigma^2}.
\]
与此同时,也往往令非相干积累的输出为
\[
    y_{\mathrm{nci}} = \sqrt{\frac{1}{P} \sum_{p=1}^P |y_p|^2},
\]
从而保持与原始脉冲以及相干积累结果的可比性。

综上所述,相干积累的增益为 \( P \) 倍,而非相干积累的增益仅为 \(\sqrt{P}\) 倍。前者理论增益更大,但实现上对相位同步的依赖更为苛刻;后者实现简单,因而在工程上被广泛采用。

\cref{fig_ci_eg}展示了 10 个脉冲和 100 个脉冲积累的结果。可以看到,相干积累能够有效抑制噪声,使目标更加突出;而非相干积累虽然同样降低了噪声的方差,但会同时抬高噪声的均值。当脉冲数量增加时,脉冲积累的效果愈发明显。总体而言,相较于单个脉冲,这两种积累方式均能显著削弱噪声干扰,从而凸显目标回波。

\begin{figure}[htb!]
    \centering
    \begin{subfigure}{.7\textwidth}
        \centering
        \includegraphics[width=.9\textwidth]{./img/detection/ci_eg.tikz}
        \caption{\( P = 10 \)}
        \label{fig_ci_eg_1}
    \end{subfigure}
    \begin{subfigure}{.7\textwidth}
        \centering
        \includegraphics[width=.9\textwidth]{./img/detection/ci_eg2.tikz}
        \caption{\( P = 100 \)}
        \label{fig_ci_eg_2}
    \end{subfigure}
    \caption{脉冲积累示意图}
    \label{fig_ci_eg}
\end{figure}

\section{目标检测}

在完成脉冲压缩与脉冲积累后,雷达系统需要对处理后的信号进行目标检测,以区分目标回波与噪声或杂波。目标检测通常被建模为一个二选一的假设检验问题,即在给定观测数据的情况下,判断其是来自仅含噪声(记作\( H_0 \))还是包含目标(记作 \( H_1 \))。基于统计决策理论,可以设计各种检测算法以优化检测性能。

\subsection{门限选择与检测性能}
在介绍检测算法之前,有必要先定义一些关键指标,以便更好地评估检测性能。对于二元假设检验问题,我们可以将检测结果分为四类:
\begin{enumerate}
    \item 真阳性(True Positive,TP):正确地将目标判定为目标。
    \item 假阳性(False Positive,FP):错误地将噪声判定为目标。
    \item 真阴性(True Negative,TN):正确地将噪声判定为噪声。
    \item 假阴性(False Negative,FN):错误地将目标判定为噪声。
\end{enumerate}
不妨令\( P_{tp}, P_{fp}, P_{tn}, P_{fn} \) 分别表示检测器的真阳性、假阳性、真阴性和假阴性的概率。不难推导,目标出现的概率为
\[
    P_{1} = P_{tp} + P_{fn},
\]
而噪声出现的概率为
\[
    P_{0} = P_{fp} + P_{tn}.
\]
并且有\( P_{0} + P_{1} = P_{tp} + P_{fp} + P_{tn} + P_{fn} = 1 \)。

检测率(Detection Rate,DR)是指在存在目标的情况下,正确地检测到目标的概率,其定义为
\[
    P_d = \frac{P_{tp}}{P_{tp} + P_{fn}}
\]
虚警率(False Alarm Rate,FAR)是指在仅含噪声的情况下,错误地将噪声判定为目标的概率,其定义为
\[
    P_{fa} = \frac{P_{fp}}{P_{fp} + P_{tn}}
\]

在大部分情况下,检测率(DR)与虚警率(FAR)之间存在一定的权衡关系。降低门限可以提高检测率,但噪声会更容易地被误判为目标,从而导致虚警率上升。因此,在实际应用中,需要根据具体场景和需求,合理选择检测门限,以达到最佳的检测性能。

以高斯分布为例,考虑如下二元假设检验模型:
\[
    \begin{cases}
        H_0:~ x \sim \mathcal{N}(\mu_n, \sigma^2), & \text{仅含噪声的情形},   \\
        H_1:~ x \sim \mathcal{N}(\mu_s, \sigma^2), & \text{包含目标回波的情形}.
    \end{cases}
\]
对应的概率密度函数如\cref{fig_dect_prob}所示,其中虚线代表检测门限,蓝色区域面积为虚警率,红色区域面积为检测率。

\begin{figure}[htb!]
    \centering
    \includegraphics[width=.4\textwidth]{./img/detection/roc_1.tikz}
    \caption{二元假设检验的概率密度函数}
    \label{fig_dect_prob}
\end{figure}

注意到,每一个检测门限都对应一对虚警率和检测率,因此通过连续调整检测门限,可以得到一条虚警率与检测率之间的关系曲线,称为接收操作特性(Receiver Operating Characteristic,ROC)曲线。当\( \mu_s - \mu_n = 1 \),\( \sigma^2 = 1 \)时,ROC曲线如\cref{fig_roc_curve_1}所示。

\begin{figure}[htb!]
    \centering
    % \includegraphics[width=.4\textwidth]{./img/detection/}
    \begin{tikzpicture}[scale=0.7]
        \begin{axis}[
                xlabel=虚警率, ylabel=检测率,
                ticklabel style={font=\small},
                label style={font=\small},
                grid, xmin=0, xmax=1, ymin=0, ymax=1, axis equal image,
                legend cell align=left,
                legend style={
                        anchor=north east,
                        font=\tiny,
                        draw=none,
                        fill=none
                    }
            ]
            \addplot[
                c1,
                thick,
            ] table[x=x1, y=y1, col sep=comma] {./img/detection/roc_curves.csv};
            \addplot[black, dashed, thick] coordinates {(0,0) (1,1)};
        \end{axis}
    \end{tikzpicture}
    \caption{ROC曲线示例}
    \label{fig_roc_curve_1}
\end{figure}

不难看出,ROC 曲线越靠近左上角,检测性能就越优。这意味着在较低虚警率的条件下,仍能获得较高的检测率。理想状态下,最优的检测器应当在虚警率为零时即可实现检测率为一,即达到完美检测的效果。相反,最差的情形是 ROC 曲线与 \cref{fig_roc_curve_1} 中的虚线完全重合,此时检测器的表现与随机猜测无异,说明目标与噪声的概率分布完全重叠。

进一步地,对于目标和噪声都服从相同方差的高斯分布来说,检测性能主要受均值差\( \mu_s - \mu_n \)和方差\(\sigma^2\)的影响。均值差越大,方差越小,越容易区分目标和噪声,检测性能也就越好。

\begin{figure}[htb!]
    \centering
    \begin{subfigure}{.4\textwidth}
        \centering
        \includegraphics[width=.9\textwidth]{./img/detection/roc_2.tikz}
        \caption{\( \mu_s - \mu_n = 2, \sigma = 1 \)}
        \label{fig_dect_prob_2_1}
    \end{subfigure}
    \begin{subfigure}{.4\textwidth}
        \centering
        \includegraphics[width=.9\textwidth]{./img/detection/roc_3.tikz}
        \caption{\( \mu_s - \mu_n = 1, \sigma = 0.5 \)}
        \label{fig_dect_prob_2_2}
    \end{subfigure}
    \caption{不同均值差和方差下的二元假设检验概率密度函数}
    \label{fig_dect_prob_2}
\end{figure}

事实上,可以证明,在高斯分布的假设下,ROC曲线仅受\( \frac{(\mu_s - \mu_n)^2}{\sigma^2} \)的影响,而\( \frac{(\mu_s - \mu_n)}{\sigma^2} \)正是本书所采用的信噪比定义。由此可见信噪比越大,ROC曲线越靠近左上角,检测性能越好,如\cref{fig_roc_curves}所示。

\begin{figure}[htb!]
    \centering
    \begin{tikzpicture}[scale=0.7]
        \begin{axis}[
                xlabel=虚警率, ylabel=检测率,
                ticklabel style={font=\small},
                label style={font=\small},
                grid, xmin=0, xmax=1, ymin=0, ymax=1, axis equal image,
                legend cell align=left,
                legend style={
                        at={(1.4,0.5)},
                        font=\tiny,
                        draw=none,
                        fill=none
                    }
            ]
            \addplot[
                c1,
                thick,
            ] table[x=x1, y=y1, col sep=comma] {./img/detection/roc_curves.csv};
            \addlegendentry{SNR = 1}
            \addplot[
                c2,
                thick,
            ] table[x=x2, y=y2, col sep=comma] {./img/detection/roc_curves.csv};
            \addlegendentry{SNR = 4}
            \addplot[
                c3,
                thick,
            ] table[x=x3, y=y3, col sep=comma] {./img/detection/roc_curves.csv};
            \addlegendentry{SNR = 9}
            \addplot[
                c4,
                thick,
            ] table[x=x4, y=y4, col sep=comma] {./img/detection/roc_curves.csv};
            \addlegendentry{SNR = 16}
            \addplot[black, dashed, thick] coordinates {(0,0) (1,1)};
        \end{axis}
    \end{tikzpicture}
    \caption{不同信噪比下的ROC曲线}
    \label{fig_roc_curves}
\end{figure}

在信噪比固定的情况下,该如何选择最优的检测门限呢?一种常见的方法是令总体检测准确率(Accuracy)\( P_{tp} + P_{tn} \)最大化。不妨令检测门限为\( \gamma \),则有检测准确率为
\[
    a(\gamma) = P_{tp}(\gamma) + P_{tn}(\gamma) =  P_{0} \int_{-\infty}^{\gamma} p_{0}(x) dx + P_{1} \int_{\gamma}^{\infty} p_{1}(x) dx,
\]
其中\( p_0 \)和\( p_1 \)分别是目标和噪声的概率密度函数。以高斯分布为例,\( a(\gamma) \)可以表示为
\[
    a(\gamma) = P_{0} \Phi\left( \frac{\gamma - \mu_n}{\sigma} \right) + P_{1} \left[ 1 - \Phi\left( \frac{\gamma - \mu_s}{\sigma} \right) \right],
\]
其中\( \Phi(\cdot) \)为标准正态分布的累积分布函数
\[
    \Phi(x) = \frac{1}{\sqrt{2\pi}} \int_{-\infty}^{x} e^{-\frac{t^2}{2}} dt.
\]
方便起见,记标准正态分布的概率密度函数为
\[
    \phi(x) = \Phi'(x) = \frac{1}{\sqrt{2\pi}} e^{-\frac{x^2}{2}}.
\]
此时,目标和噪声的概率密度函数分别为
\[
    p_0(x) = \frac{1}{\sigma}\phi\left( \frac{x - \mu_n}{\sigma} \right), \quad p_1(x) = \frac{1}{\sigma}\phi\left( \frac{x - \mu_s}{\sigma} \right).
\]
注意到对\( \Phi\left( \frac{\gamma - \mu_n}{\sigma} \right) \)求导有
\[
    \frac{d}{d\gamma} \Phi\left( \frac{\gamma - \mu_n}{\sigma} \right) = \frac{1}{\sigma}\phi\left( \frac{\gamma - \mu_n}{\sigma} \right) = p_0(\gamma),
\]
因此\( a(\gamma) \)关于\( \gamma \)的导数为
\[
    a'(\gamma) = P_{0} p_{0}(\gamma) - P_{1} p_{1}(\gamma).
\]
并令其为零,得到
\[
    \gamma = \frac{\mu_s^2 - \mu_n^2 + 2\sigma^2 \ln\frac{P_0}{P_1}}{2(\mu_s - \mu_n)}.
\]
如\cref{fig_dect_th}所示,最优检测门限对应于两条概率密度函数在出现概率加权后的交点,其横坐标即为最优阈值。此外,图中红色区域面积等于\( P_{tp} \),蓝色区域面积等于\( P_{tn} \)。

\begin{figure}[htb!]
    \centering
    \begin{tikzpicture}
        \begin{axis}[
                xmin=-6, xmax=6,
                ymin=0, ymax=0.4,
                xtick=\empty, ytick=\empty,
                width=8cm, height=4cm,
                grid=both,
                clip=false,
            ]

            % 第一条曲线 (加 name path=A)
            \addplot[name path=a, domain=-6:6, samples=200, thick, c1] {exp(-(x + 1)^2/2)/sqrt(2*pi)};

            % 第二条曲线 (加 name path=B)
            \addplot[name path=b, domain=-6:6, samples=200, thick, c2] {0.5*exp(-(x-1)^2/2)/sqrt(2*pi)};

            % % 基准线 (x 轴右侧部分)
            \path[name path=axis1] (axis cs:0.35,0) -- (axis cs:6,0);
            \path[name path=axis2] (axis cs:-6,0) -- (axis cs:0.35,0);

            % % 填充曲线 B 下方 (x>=0.35)
            \addplot[c2!60, opacity=0.5] fill between[of=b and axis1, soft clip={domain=0.35:6}];

            % % 填充曲线 A 下方 (x>=1.5)
            \addplot[c1!60, opacity=0.5] fill between[of=a and axis2, soft clip={domain=-6:0.35}];

            % % 竖线
            \addplot[dashed, thick, black] coordinates {(0.35,0) (0.35,0.15)};
        \end{axis}
    \end{tikzpicture}
    \caption{最优检测门限示意图}
    \label{fig_dect_th}
\end{figure}

\subsection{恒虚警率检测}
在雷达信号检测中,单纯追求检测准确率最大化并不一定合理。因为在实际应用中,目标出现的概率往往极低,如果仅以准确率为准则,检测器可能会倾向于判定“无目标”,从而错失真实目标。因此,工程实践中更希望在容忍一定虚警的前提下,尽可能保证目标被探测到。然而,若虚警率过高,又会导致大量无效告警,造成资源浪费。为此,引入了恒虚警率(Constant False Alarm Rate, CFAR)检测方法:通过预先设定一个可接受的虚警率,并自适应地调整检测门限,使得虚警率始终维持在设定水平以内。

设噪声(或干扰)的概率密度函数为\( p_n(x) \),则给定门限\( \gamma \)时,虚警率为
\[
    P_{fa}(\gamma) = \int_{\gamma}^{\infty} p_n(x) dx.
\]
为了实现恒虚警率检测,需要根据预设的虚警率\( \hat{P}_{fa} \)求解门限\( \gamma \)。具体而言,需解方程
\[
    P_{fa}(\gamma) = \hat{P}_{fa}.
\]

注意到,脉冲压缩后的输出为复数信号。为了便于目标检测,通常需要先取其模长平方,再进行后续判决。因此,若假设噪声服从均值为零、方差为\( \sigma^2 \)的复高斯分布,则其模长平方服从指数分布(均值为\( \sigma^2 \),方差为\( \sigma^4 \)),概率密度函数为
\[
    p_n(x) = \frac{1}{\sigma^2} e^{-\frac{x}{\sigma^2}}, \quad x \geq 0.
\]
此时,虚警率关于门限的函数为
\[
    P_{fa}(\gamma) = \int_{\gamma}^{\infty} \frac{1}{\sigma^2} e^{-\frac{x}{\sigma^2}} dx = e^{-\frac{\gamma}{\sigma^2}}.
\]
因此,给定虚警率\( \hat{P}_{fa} \),可以求解门限\( \gamma \)为
\[
    \gamma = -\sigma^2 \ln \hat{P}_{fa}.
\]

然而,在实际应用中,噪声的统计特性无法提前获得,且可能随时间和空间发生变化。因此,需要自适应地对噪声进行建模和估计,以便动态调整检测门限。给定脉压并取模平方后的噪声观测样本 \( N \) 个,记作
\[
    X_1, X_2, \ldots, X_N,
\]
它们相互独立并且均服从均值为 \( \sigma^2 \) 的指数分布。令门限取值与观测均值成正比:
\[
    \gamma = \alpha \overline{X} = \alpha \frac{1}{N}\sum_{i=1}^{N} X_i,
\]
其中 \( \alpha \) 为可调节的比例系数。对于一个同样服从指数分布的随机变量 \( X_0 \),以\( \overline{X} \)为先验的情况下,将其判为目标的概率(即虚警率)为
\[
    P_{fa}(\gamma \mid \overline{X}) = P(X_0 > \gamma \mid \overline{X})
    = P(X_0 > \alpha \overline{X})
    = e^{-\tfrac{\alpha \overline{X}}{\sigma^2}}.
\]

注意到 \( \overline{X} \) 是若干独立指数分布的均值,因此其服从均值为 \( \sigma^2 \) 的伽马分布,即
\[
    \overline{X} \sim \mathrm{Gamma}\!\left(N, \tfrac{N}{\sigma^2}\right),
\]
对应的概率密度函数为
\[
    p_{\overline{X}}(x) = \frac{\bigl(\tfrac{N}{\sigma^2}\bigr)^N}{\Gamma(N)}\,x^{N-1}
    e^{-\tfrac{N}{\sigma^2}x}, \quad x>0,
\]
其中 \( \Gamma(N)=(N-1)! \) 为伽马函数。于是虚警率可写为
\[
    \begin{aligned}
        P_{fa}(\gamma)
         & = \int_{0}^{\infty} P_{fa}(\gamma \mid x)\,p_{\overline{X}}(x)\,dx \\
         & = \frac{\bigl(\tfrac{N}{\sigma^2}\bigr)^N}{\Gamma(N)}
        \int_{0}^{\infty} x^{N-1} e^{-\tfrac{N+\alpha}{\sigma^2}x}\,dx.
    \end{aligned}
\]
利用 Gamma 积分公式
\[
    \int_{0}^{\infty} x^{N-1} e^{-\beta x}\,dx = \frac{\Gamma(N)}{\beta^N},
\]
可得
\[
    P_{fa}(\gamma) = \frac{\left(\frac{N}{\sigma^2}\right)^{N}}{\Gamma(N)} \frac{\Gamma(N)}{\left(\frac{N+\alpha}{\sigma^2}\right)^{N}} = \left(1+\frac{\alpha}{N}\right)^{-N}.
\]

因此,当比例系数为 \( \alpha \) 时,虚警率为
\[
    P_{fa}(\gamma) = \left(1+\frac{\alpha}{N}\right)^{-N}.
\]
反之,若预设虚警率为 \( \hat{P}_{fa} \),则门限可写为
\[
    \gamma = \alpha \overline{X}
    = N\!\left(\hat{P}_{fa}^{-1/N}-1\right)\overline{X}.
\]

给定一段脉冲压缩并取模平方后的离散序列
\[
    \cdots, X_{i-1},\, X_i,\, X_{i+1}, \cdots ,
\]
其中 \(X_i\) 为待检测样本(Cell Under Test, CUT)。最常见的 CFAR 算法是基于滑动窗口的单元平均恒虚警率检测(Cell Averaging CFAR, CA-CFAR)。该方法通过在 CUT 周围选取训练单元估计背景噪声功率,从而自适应调整检测门限。

设滑动窗口长度为 \(2N+1\),保护单元长度为 \(2G+1\)(其中 \(G<N\))。则 CUT 左右两侧各有 \(N-G\) 个训练单元。训练单元的均值估计为
\[
    \overline{X}_i = \frac{1}{2(N-G)}
    \left( \sum_{j=i-N}^{\,i-G-1} X_j \;+\; \sum_{j=i+G+1}^{\,i+N} X_j \right).
\]
在给定目标虚警率 \(\hat{P}_{fa}\) 的条件下,CA-CFAR 的检测门限可写为
\[
    \gamma_i = N\!\left(\hat{P}_{fa}^{-1/N}-1\right)\overline{X}_i.
\]
当 \(X_i > \gamma_i\) 时判定存在目标,否则判定为噪声。此外,之所以要设置保护单元,是因为脉压后的脉冲仍有一定的宽度,且往往存在旁瓣。如果不设置保护单元,目标主瓣或旁瓣的能量就会被误算进训练单元,导致背景噪声估计被抬高,阈值过高,从而削弱检测灵敏度甚至漏检目标。

\begin{figure}[htb!]
    \centering
    \begin{subfigure}{.7\textwidth}
        \centering
        \begin{tikzpicture}
            \begin{axis}[
                    xmin=-5, xmax=5,
                    ymin=0,
                    width=10cm, height=3cm,
                    grid=none,
                    xtick=\empty,ytick=\empty,
                    clip=false,
                ]
                \addplot[
                    c1,
                ] table[x=t, y=y, col sep=comma] {./img/detection/cfar.csv};
                \addplot[
                    c2,
                ] table[x=t, y=t1, col sep=comma] {./img/detection/cfar.csv};
            \end{axis}
        \end{tikzpicture}
        \caption{\( \hat{P}_{fa} = 10^{-1} \)}
        \label{fig_cfar_test_1}
    \end{subfigure}
    \begin{subfigure}{.7\textwidth}
        \centering
        \begin{tikzpicture}
            \begin{axis}[
                    xmin=-5, xmax=5,
                    ymin=0,
                    width=10cm, height=3cm,
                    grid=none,
                    xtick=\empty,ytick=\empty,
                    clip=false,
                ]
                \addplot[
                    c1,
                ] table[x=t, y=y, col sep=comma] {./img/detection/cfar.csv};
                \addplot[
                    c2,
                ] table[x=t, y=t2, col sep=comma] {./img/detection/cfar.csv};
            \end{axis}
        \end{tikzpicture}
        \caption{\( \hat{P}_{fa} = 10^{-2} \)}
        \label{fig_cfar_test_2}
    \end{subfigure}
    \begin{subfigure}{.7\textwidth}
        \centering
        \begin{tikzpicture}
            \begin{axis}[
                    xmin=-5, xmax=5,
                    ymin=0,
                    width=10cm, height=3cm,
                    grid=none,
                    xtick=\empty,ytick=\empty,
                    clip=false,
                ]
                \addplot[
                    c1,
                ] table[x=t, y=y, col sep=comma] {./img/detection/cfar.csv};
                \addplot[
                    c2,
                ] table[x=t, y=t3, col sep=comma] {./img/detection/cfar.csv};
            \end{axis}
        \end{tikzpicture}
        \caption{\( \hat{P}_{fa} = 10^{-3} \)}
        \label{fig_cfar_test_3}
    \end{subfigure}
    \caption{CA-CFAR检测示意图}
    \label{fig_cfar_test}
\end{figure}

如\cref{fig_cfar_test} 所示,蓝色曲线表示脉压后信号的模平方,红色曲线表示 CA-CFAR 算法计算得到的检测门限,蓝色曲线高于红色曲线的部分则会被识别为目标。可以看到,蓝色曲线中的噪声水平随距离逐渐升高,若采用固定门限,则会产生大量误警。相比之下,CA-CFAR 能够根据背景噪声的变化自适应调整门限,从而有效抑制噪声影响。同时,通过合理设定虚警率参数,可以在保证检测灵敏度的前提下,将虚警率控制在可接受范围内,实现检测性能与可靠性的平衡。
